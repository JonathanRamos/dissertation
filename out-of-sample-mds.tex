\documentclass{article}
\usepackage{amsmath}
\usepackage{amssymb}
\usepackage{amsthm}
\usepackage{bm}
\usepackage[ntheorem]{empheq}
\usepackage{graphics}
\usepackage[colorlinks=true,pagebackref,linkcolor=magenta]{hyperref}
%\usepackage{hyperref}
\usepackage[colon,sort&compress]{natbib}
\usepackage{natbib}
\bibliographystyle{plainnat}
\begin{document}
\title{On the out-of-sample extension to classical MDS}
\maketitle
\section{Nystr\"{o}m method for kernel PCA}
Let $\mathcal{X} = \{x_1, x_2, \dots, x_n\}$ be $n$ sampled data
points from some feature space $\Omega$.  Let $k$ be a symmetric,
function on $\Omega \times \Omega$. Denote by
$\mathbf{K}_{\mathcal{X}}$ the $n \times n$ matrix whose $ij$ entry is
$k(x_i,x_j)$. We assume that $k$ is a positive semidefinite function,
i.e., the matrix $\mathbf{K}_{\mathcal{X}}$ is positive semidefinite
for any arbitrary $\mathcal{X}$. 
Define $\tilde{k}_{\mathcal{X}}$, the centering of $k$
with respect to $\mathcal{X}$ as
\begin{equation}
  \label{eq:1}
  \tilde{k}_{\mathcal{X}}(x,y) = k(x,y) -
  \frac{1}{n}\sum_{i=1}^{n}k(x,x_i) -
  \frac{1}{n}\sum_{i=1}^{n}k(x_i,y) + 
\frac{1}{n^2}\sum_{i=1}^{n}\sum_{j=1}^{n}k(x_i,x_j)
\end{equation}
Consider the sample centered inner product matrix
$\tilde{\mathbf{K}}_{\mathcal{X}} =
(\tilde{k}_{\mathcal{X}}(x_i,x_j))_{i,j=1}^{n}$, i.e.,
$\tilde{\mathbf{K}}_{\mathcal{X}} = (\mathbf{I} -
\tfrac{\mathbf{J}}{n})\mathbf{K}_{\mathcal{X}}(\mathbf{I} -
\tfrac{\mathbf{J}}{n})$. Kernel PCA
\citet{scholkopf97:_lectur_notes_comput_scien} embeds the data points
$x_i \in \mathcal{X}$ into $\mathbb{R}^{d}$ by using $d$ eigenvectors
corresponding to the $d$ largest eigenvalues of
$\tilde{\mathbf{K}}_{\mathcal{X}}$. Suppose that $\mathbf{V}
\bm{\Lambda} \mathbf{V}^{T}$ is the spectral decomposition of
$\tilde{\mathbf{K}}_{\mathcal{X}}$. Let $\lambda_1 \geq \lambda_2 \geq
\dots \geq \lambda_n$ be the eigenvalues of
$\tilde{\mathbf{K}}_{\mathcal{X}}$ and $\mathbf{v}_1, \mathbf{v}_2,
\dots, \mathbf{v}_n$ the corresponding eigenvectors. Given a new data
point $x \not \in \mathcal{X}$, \citet{williams01:_using_nystr} embed
$x$ as $\mathbf{f}(x) = (f_{1}(x), f_{2}(x), \dots, f_{d}(x))^{T} \in
\mathbb{R}^{d}$ by
\begin{equation}
  \label{eq:2}
  f_{r}(x) = \frac{1}{\sqrt{\lambda_r}} \sum_{i=1}^{n}{\mathbf{v}_{r}(i) 
\tilde{k}_{\mathcal{X}}(x_i,x)}
\end{equation}
Written in matrix vector form, $\mathbf{f}_x$ is
\begin{equation}
  \label{eq:3}
  \mathbf{f}(x) = \bm{\Lambda}_{d}^{-1/2} \mathbf{V}^{T}_d \mathbf{b}(x)
\end{equation}
where $\mathbf{V}_d$ and $\bm{\Lambda}_d$ are the $d \times d$
submatrix of $\mathbf{V}$ and $\bm{\Lambda}$ and $\mathbf{b}(x)$ is
the column vector whose entries are the
$\tilde{k}_{\mathcal{X}}(x_i,x)$. Eq.~(\ref{eq:3}) is the main idea
behind the out-of-sample extension of classical MDS of 
\citet{bengio04:_out_lle_isomap_mds_eigen} (see also
\cite{bengio06:_featur_extrac}). 
\section{Out-of-sample extension of classical MDS}
\cite{bengio04:_out_lle_isomap_mds_eigen} out-of-sample extension of
classical MDS follows the Nystr\"{o}m model for kernel PCA of the
preceding section. The kernel $k$ in kernel PCA is replaced by a
dissimilarity measure $d \colon \Omega \times \Omega \mapsto
\mathbb{R}^{\geq 0}$. \cite{bengio04:_out_lle_isomap_mds_eigen} then
transform the dissimilarity measure $d$ into an approximation of the
inner product by
\begin{equation}
  \label{eq:4}
  \tilde{k}_{\mathcal{X}}(x,y) = -\frac{1}{2}\Bigl(d^{2}(x,y) -  \frac{1}{n}\sum_{i=1}^{n}d^{2}(x,x_i) -
  \frac{1}{n}\sum_{i=1}^{n}d^{2}(x_i,y) + 
\frac{1}{n^2}\sum_{i=1}^{n}\sum_{j=1}^{n}d^{2}(x_i,x_j)\Bigr)
\end{equation}
Note that Eq.~(\ref{eq:4}) is equivalent to the centering using the
$\tau$ transform in \citet{trosset08} with respect
to the original set of points in $\mathcal{X}$. The out-of-sample
embedding of a new data point $x \not \in \mathcal{X}$ is then given
by Eq.~(\ref{eq:3}) where the matrices $\mathbf{V}$ and $\bm{\Lambda}$
generate the eigendecomposition of $
\tilde{\mathbf{K}}_{\mathcal{X}}$ with the kernel function $\tilde{k}$
being given by Eq.~(\ref{eq:4}). Note that
$\tilde{\mathbf{K}}_{\mathcal{X}}$ is not necessarily positive
definite, and thus $\mathbf{V}_d$ and $\bm{\Lambda}_{d}$ must
be restricted to the non-negative eigenvalues of
$\tilde{\mathbf{K}}_{\mathcal{X}}$. \\ \\
%
\cite{anderson03:_gener} solved a least square problem and embed a new
data point $x \not \in \mathcal{X}$ by 
\begin{equation}
  \label{eq:6}
   \mathbf{g}(x) = (\mathbf{X}^{T}\mathbf{X})^{-1}\mathbf{X}^{T} \mathbf{b}(x)
\end{equation}
where $\mathbf{b}(x)$ is the column matrix whose entries are the
$\tilde{k}_{\mathcal{X}}(x_i,x)$ as defined in Eq.~(\ref{eq:4}) and
$\mathbf{X}$ is the classical MDS embedding of the data points in
$\mathcal{X}$. We now relate Eq.~(\ref{eq:3}) and
Eq.~(\ref{eq:6}). Let $\mathbf{U}\bm{\Sigma}\mathbf{W}^{T}$ be the
singular value decomposition of $\mathbf{X}$. Then
$(\mathbf{X}^{T}\mathbf{X})^{-1}\mathbf{X}^{T} = \mathbf{X}^{\dagger}
= \mathbf{W}\bm{\Sigma}^{\dagger}\mathbf{U}^{T}$ is the Moore-Penrose
pseudoinverse of $\mathbf{X}$. Since $\mathbf{X}$ is the rank-d
solution of classical MDS with the approximated inner product matrix
$\tilde{\mathbf{K}}_{\mathcal{X}}$, $\mathbf{X}\mathbf{X}^{T} =
\mathbf{V}_d\bm{\Lambda}_{d}\mathbf{V}_{d}^{T}$. We can thus also chose
$\mathbf{U}$, $\bm{\Sigma}$ and $\mathbf{W}$ in the SVD for
$\mathbf{X}$ such that $\mathbf{U}_d = \mathbf{V}_d$ and
$(\bm{\Sigma}\bm{\Sigma}^{T})_{d} = \bm{\Lambda}_d$. Furthermore, we
also have that
\begin{equation}
  \label{eq:7}
  \mathbf{g}(x) = \mathbf{X}^{\dagger} \mathbf{b}(x) = \mathbf{W} \bm{\Sigma}^{\dagger}\mathbf{U}^{T}
  \mathbf{b}(x) = \mathbf{W} \bm{\Sigma}_d^{-1} \mathbf{U}_d^{T}
  \mathbf{b}(x) = \mathbf{W} \bm{\Lambda}_d^{-1/2} \mathbf{V}_d^{T}
  \mathbf{b}(x)
\end{equation}
where $\bm{\Sigma}_d$ is the appropriate $d \times d$ submatrix of
$\bm{\Sigma}$. Comparing Eq.~(\ref{eq:7}) and Eq.~(\ref{eq:3}), we see
that the solution of \cite{anderson03:_gener} is a unitary
transformation of that of
\cite{bengio04:_out_lle_isomap_mds_eigen}. However 
\begin{equation*}
  \mathbf{X}g(x) =
  \mathbf{U} \bm{\Sigma}
  \mathbf{W}^{T}\mathbf{W}\bm{\Sigma}_d^{-1}\mathbf{U}_d^{T}\mathbf{b}(x)
  = \mathbf{U}_d\mathbf{U}_d^{T}\mathbf{b}_x =
  \mathbf{V}_d\mathbf{V}_d^{T} \mathbf{b}(x) = \mathbf{V}_d \bm{\Lambda}_d^{1/2}\mathbf{f}(x)
\end{equation*}
The two solutions are thus equivalent. \cite{trosset08} noted that the solution
of \cite{anderson03:_gener} is a minimizer for one of the term in
their objective function.  \bibliography{dissertation}
\end{document}
%%% Local Variables: 
%%% mode: latex
%%% TeX-master: t
%%% End: 
