\chapter{Introduction}
\label{cha:introduction}

\section{Hessian eigenmaps}
\label{sec:hessian-eigenmaps}
Hessian eigenmaps \citet{donoho03:_hesian} was described as a locally
linear embedding technique for high-dimensionality data. Hessian
eigenmaps assume that the data points lies on a  
Riemannian manifold $\mathcal{M} \subset \mathbb{R}^{n}$ such
that $\mathcal{M}$ is locally isometric to an open connected subset
$\Omega \subset \mathbb{R}^{d}$. For $f \colon \mathcal{M} \mapsto
\mathbb{R}$, define a quadratic form $\mathcal{H}(f)$ associated with
$f$ by $\mathcal{H}(f) = \int_{M}{ \| H_f(m) \|_{F}^{2} dm }$ where
$H_f(m)$ is the Hessian of $f$ at $m$ and $\| \cdot \|_F$ is the
Frobenius norm. $\mathcal{H}(f)$ averages the Frobenius norm of the
Hessian of $f$ over $\mathcal{M}$. The embedding coordinates
corresponds to the basis for the null space of $\mathcal{H}(f)$. The
procedure can be described as follows. 

Let $\{m_i\}_{i=1}^{N}$ be a collection of $N$ points in
$\mathbb{R}^{n}$. Suppose that $K$ is a parameter chosen to be the
size of the nearest neighborhoods and $d$ is a parameter chosen to be
the dimension of the embedding coordinates. The Hessian eigenmaps
procedure proceed as follows \citet{donoho03:_hesian}
\begin{enumerate}
\item Identify neighbors: For each $m_i$, set $\mathcal{N}_i$ to be
  the set of the $K$ nearest neighbours of $m_i$. For each
  neighbourhood $\mathcal{N}_i$, set $\mathbf{M}_i$ to be the $K \times n$
  matrix whose rows are the points $m_j \in \mathcal{N}_i$. Define
  $\tilde{\mathbf{M}_i}$ to be $(\mathbf{I} -
  \mathbf{J}/K)\mathbf{M}_i$, i.e. $\tilde{\mathbf{M}_i}$ is
  the row centering of $\mathbf{M}_i$.
\item Do principal component analysis (PCA) on
  $\tilde{\mathbf{M}_i}$. This is equivalent to finding the singular
  value decomposition (SVD) of $\tilde{\mathbf{M}_i} = \mathbf{U}_i
  \bm{\Lambda} \mathbf{V}_i$ and using the first $d$ columns of
  $\mathbf{U}_i$ as the tangent coordinates of points in
  $\mathcal{N}_i$.
\item Estimate the entries of the Hessian coordinates by fitting a
  polynomial of degree two 
\end{enumerate}


%%% Local Variables: 
%%% mode: latex
%%% TeX-master: "dissertation"
%%% End: 
