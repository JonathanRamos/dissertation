\documentclass[professionalfonts,hyperref={pdfpagelabels=false,colorlinks=true,linkcolor=blue}]{beamer}

\mode<presentation>{
  \usetheme{Boadilla}
  \useinnertheme{rectangles}
}
\usepackage[english]{babel}
\usepackage[latin1]{inputenc}
\usepackage{subfigure}
\usepackage{bm}
\usepackage[all]{xy}
\bibliographystyle{plainnat}
%\usepackage{pgfpages}
%\pgfpagesuselayout{4 on 1}[letterpaper, landscape, border shrink=5mm]

\newtheorem{question}[theorem]{Question}
\newtheorem{openquestion}[theorem]{Open Question}
\setbeamercolor{question title}{bg = red}
\setbeamercolor{block body question}{bg=blue!60}

\title[]{On a relationship between Laplacian eigenmaps and diffusion
  maps.}
\author[Tang \& Trosset]{Minh Tang\inst{1} \and Michael
  Trosset\inst{2}}
\institute[Indiana University]{
  \inst{1} School of Informatics and Computing \\
  Indiana University, Bloomington
  \and \inst{2} Department of Statistics \\ Indiana University,
  Bloomington
}
\date[]{This work was funded by a grant from the Office of Naval Research.}
AtBeginSection[]{
  \begin{frame}<beamer>
    \frametitle{Outline}
    \tableofcontents[currentsection,currentsubsection]
    \end{frame}
}
\begin{document}
\begin{frame}
\titlepage
\end{frame}

\begin{frame}{Problem Description}
  We consider the problem of constructing a low-dimensional Euclidean
  representation of data described by pairwise similarities. 
  
 \vskip10pt The low-dimensional representation can served as the basis
 for other exploitation tasks, e.g., visualization, clustering, or 
 classification.  
 
 \vskip10pt Our basic strategy is:

  \begin{enumerate}
  \item Transform the similarities into some notion of dissimilarities;
  \item Embed the derived dissimilarities, e.g., by classical
    multidimensional scaling \cite{torgesen52:_multid},
    \cite{gower66:_some}.
  \end{enumerate}
  
  Our concerns are closely related to the concerns of \alert{manifold
    learning}. Various manifold learning techniques can be interpreted
  as transformations from similarities to dissimilarities.
\end{frame}
\begin{frame}{Some Terminologies}
  A \alert{dissimilarity matrix} $\bm{\Delta} = (\delta_{ij})$
  is a hollow, symmetric, non-negative matrix. Larger values indicate
  that the objects are more dissimilar.  
  \vskip10pt A \alert{similarity matrix} $\bm{\Gamma} =
  (\gamma_{ij})$ is a symmetric, non-negative matrix. Larger values
  indicate that the objects are more similar. 
  \vskip10pt A $n \times n$ dissimilarity matrix $\bm{\Delta} =
  (\delta_{ij})$ is a \alert{Type-2 Euclidean distance matrix}
  (EDM-2) iff there some $x_1,\dots, x_n \in
  \mathbb{R}^{p}$ such that $\delta_{ij} = \|x_i - x_j\|^{2}$.  
 \vskip10pt There is an equivalence between EDM-2 and psd matrices.
    \begin{itemize}
    \item If $\bm{\Delta}$ is EDM-2, then 
        $\mathbf{B} =
        \tau(\bm{\Delta}) = - \frac{1}{2} \mathbf{P} \bm{\Delta}
       \mathbf{P}$ is psd, where $\mathbf{P} = (\mathbf{I} - \bm{1}\bm{1}^{T}/n)$. 
 \item If $\bm{B}$ is psd then 
        $\bm{\Delta} =
        \kappa(\mathbf{B}) = \mathrm{diag}(\mathbf{B})\bm{1}\bm{1}^{T} -
        2\mathbf{B} + \bm{1}\bm{1}^{T}\mathrm{diag}(\mathbf{B})$       
      is EDM-2.
    \end{itemize}
\end{frame}

\begin{frame}
  \frametitle{Isomap} 
  Isomap \cite{tenebaum00:_global_geomet_framew_nonlin_dimen_reduc} is
  one of the best known manifold learning algorithm. Suppose that
  $y_1, y_2, \dots, y_n \in \mathbb{R}^{q}$ lie on a $d$-dimensional
  manifold. To represent $y_1, y_2, \dots, y_n$ as $x_1, x_2,
  \dots, x_n \in \mathbb{R}^{d}$, Isomap replaces Euclidean distance
  in $\mathbb{R}^{q}$ with a clever approximation of geodesic distance
  on the manifold as follows: 
  \begin{enumerate}
  \item Replace Euclidean distance with approximate geodesic
    distance.
    \begin{enumerate}
    \item[(a)] Construct a weighted graph $G = (V,E,\omega)$ with $n$
      vertices. Fix some $\epsilon \geq 0$ and let $v_i \sim v_j$ iff
      $\|y_i - y_j\| \leq \epsilon$. If $v_i \sim v_j$, set
      $\omega_{ij} = \|y_i - y_j\|$.
    \item[(b)] Compute $\bm{\Delta} = (\delta_{ij})$ where
      $\delta_{ij}$ is the shortest path distance between $v_i$ and
      $v_j$ in $G$.
    \end{enumerate}
   \item Embed $\bm{\Delta}$ by CMDS.
  \end{enumerate}

\end{frame}
% \begin{frame}[label=isomap_example]
%   \frametitle{Isomap Example: Embedding of a Swiss Roll}
%   \subfiglabelskip=0pt
%   \begin{figure}[htbp]
%     \label{fig:swissroll}
%     \centering
%     \subfigure[][]{
%       \includegraphics[width=40mm]{swissroll.pdf}
%     }
%     \hspace{3pt}
%     \subfigure[][]{
%       \includegraphics[width=40mm]{isomap_swissroll.pdf}
%     }
%     \caption{Isomap example. Shortest path distances approximate
%       geodesic distances.}
%   \end{figure}
% \end{frame}

\begin{frame}{From Similarities to Distances on Graphs}
 The Isomap recipe can be adapted to work with similarities as
  follows.
  \vskip10pt Given a $n \times n$ similarities matrix $\bm{\Gamma} = (\gamma_{ij})$:
  \begin{enumerate}
  \item Transform the similarities to distances. (Isomap
    starts off with dissimilarties).
    \begin{enumerate}
    \item[(a)]Construct a weighted graph $G = (V,E,\omega)$ with $n$
      vertices and edge weights $\omega_{ij} = \gamma_{ij}$.
    \item[(b)] Construct a matrix $\bm{\Delta} = (\delta_{ij})$
      that measures some suitable distance on $G$. 
    \end{enumerate}
  \item Embed $\bm{\Delta}$. 
  \end{enumerate}
  Several popular approaches to transform from similarities to
   distances relies on the concept of a \alert{random walk}.
    
    \vskip10pt Assume that $G$ is connected. Let $\bm{s} =
    \bm{\Gamma}\bm{1}$ and $\mathbf{S} = \mathrm{diag}(\bm{s})$. Then
    the random walk on $G = (V,E,\omega)$ is the Markov chain with
    state space $V$ and transition probabilities $\mathbf{P} =
    \mathbf{S}^{-1}\bm{\Gamma}$. The stationary distribution
    $\bm{\pi}$ of $\mathbf{P}$ exists and is unique, and furthermore,
    $\lim_{k \rightarrow \infty} \mathbf{P}^{k} = \bm{1}\bm{\pi}^{T}
    := \mathbf{Q}$.
\end{frame}

\begin{frame}{Expected Commute Time}
  Following \cite{kemeny83:_finit_markov_chain}, let
  \begin{equation*}
    \bm{\Pi} = \mathrm{diag}(\bm{\pi}) \quad \text{and} \quad
    \mathbf{Z} = (\mathbf{I} - \mathbf{P} + \mathbf{Q})^{-1}.
  \end{equation*}
  The expected first passage times are given by
  \begin{equation*}
    \mathbf{M} = (\mathbf{1}\mathbf{1}^{T}\mathrm{diag}(\mathbf{Z}) -
    \mathbf{Z})\bm{\Pi}^{-1} 
  \end{equation*}
  and the expected commute times are
  \begin{equation*}
    \bm{\Delta}_{\mathrm{ect}} = \mathbf{M} + \mathbf{M}^{T} =
    \kappa(\mathbf{Z}\bm{\Pi}^{-1})
  \end{equation*}

 It turns out that $\mathbf{Z}\bm{\Pi}^{-1} \succeq
  0$. $\bm{\Delta}_{\mathrm{ect}}$ is thus \alert{EDM-2}.
\end{frame}

\begin{frame}{Diffusion Distances}
  Let $\bm{e}_i$ and $\bm{e}_j$ denote point masses at vertices $v_i$ and
  $v_j$. After $r$ time steps, under the random walk model with
  transition matrix $\mathbf{P}$, these distributions had diffused to
  $\bm{e}_i^{T} \mathbf{P}^{r}$ and $\bm{e}_j^{T}\mathbf{P}^{r}$. 
  
  \vskip10pt The diffusion distance \cite{coifman06:_diffus_maps} at
  time $r$ between $v_i$ and $v_j$ is
    \begin{equation*}
      \rho_{r}(v_i,v_j) = \| \bm{e}_i^{T} \mathbf{P}^{r} -
      \bm{e}_j^{T}
      \mathbf{P}^{r} \|_{1/\bm{\pi}}
    \end{equation*}
    where the inner product $\langle \cdot, \cdot
    \rangle_{1/\bm{\pi}}$ is defined as
    \begin{equation*}
      \langle \bm{u}, \bm{v} \rangle_{1/\bm{\pi}} = \sum_{k} u(k)
      v(k)/\pi(k)
    \end{equation*}
    
      \vskip10pt
      It turns out that $\Delta_{\rho_{r}^{2}} =
      \kappa(\mathbf{P}^{2r}\bm{\Pi}^{-1})$. 
      Because $\mathbf{P}^{2r}\bm{\Pi}^{-1} \succeq 0$,
      $\Delta_{\rho_{r}^{2}}$ is EDM-2.  
\end{frame}

\begin{frame}{Some Remarks on ECT and Diffusion
  Distances}
  \begin{enumerate}
  \item $\bm{\Delta}_{\mathrm{ect}}$ can be written as
    \begin{equation*}
      \bm{\Delta}_{\mathrm{ect}} = \kappa(\mathbf{Z}\bm{\Pi}^{-1}) =
      \kappa\Bigl( \sum_{k=0}^{\infty}(\mathbf{P} -
      \mathbf{Q})^{k}\bm{\Pi}^{-1}\Bigr).
    \end{equation*}
    The expected commute time between $v_i$ and $v_j$ take into account
    paths of all length between $v_i$ and $v_j$.
  \item Even though $(\mathbf{P} - \mathbf{Q})^{k} =
    \mathbf{P}^{k} - \mathbf{Q}$ for $k \geq 1$,
    $\mathbf{Q}\bm{\Pi}^{-1} = \bm{1}\bm{1}^{T}$ and
    $\kappa(\bm{1}\bm{1}^{T}) = \bm{0}$, one cannot write
    $\bm{\Delta}_{\mathrm{ect}} =
    \kappa\Bigl(\sum_{k=0}^{\infty}\mathbf{P}^{k}\bm{\Pi}^{-1}\Bigr)$
    because $\sum_{k=0}^{\infty}\mathbf{P}^{k}\bm{\Pi}^{-1}$
    doesn't necessarily converge.
  \item $\bm{\Delta}_{\rho_{r}^{2}} =
    \kappa(\mathbf{P}^{2r}\bm{\Pi}^{-1}) = \kappa\bigl((\mathbf{P} -
    \mathbf{Q})^{2r}\bm{\Pi}^{-1}\bigr)$. Diffusion distance between
    $v_i$ and $v_j$ at time $r$ take into account only paths of length
    $2r$.
  \end{enumerate}
\end{frame}

\begin{frame}{General Framework for Euclidean Distances on Graphs}
  We now introduce a general family of Euclidean distances constructed
  from random walks on graphs. 
  
  \vskip10pt Let $f$ be a real-valued function with a
    series expansion
    \begin{equation*}
      f(x) = a_0 + a_1 x + a_2 x^2 + \cdots
    \end{equation*}
    and radius of convergence $R \geq 1$. 
  
    \vskip10pt
    If $f(x) \geq 0$ for $x \in (-1,1)$ (and $\mathbf{P}$
    is irreducible and aperiodic), then
    \begin{equation*}
      \bm{\Delta} = \kappa(f(\mathbf{P} - \mathbf{Q}) \bm{\Pi}^{-1}) =
      \kappa\Bigl((a_0
      \mathbf{I} + a_1 (\mathbf{P} - \mathbf{Q}) + a_2 (\mathbf{P} -
      \mathbf{Q})^2 + \cdots)\bm{\Pi}^{-1}\Bigr)
    \end{equation*}
    is well-defined and EDM-2. In the above equation, $f$ acts on the
   matrix $\mathbf{P} - \mathbf{Q}$ and not on the entries of
    $\mathbf{P} - \mathbf{Q}$. 
\end{frame}

\begin{frame}{Euclidean Distances on Graphs: Some Examples}
 \begin{equation*}
      \bm{\Delta} = \kappa(f(\mathbf{P} - \mathbf{Q}) \bm{\Pi}^{-1}) =
      \kappa\Bigl((a_0
      \mathbf{I} + a_1 (\mathbf{P} - \mathbf{Q}) + a_2 (\mathbf{P} -
      \mathbf{Q})^2 + \cdots)\bm{\Pi}^{-1}\Bigr)
    \end{equation*}

  \vskip 10pt The following functions generate $\bm{\Delta}$ that are
  EDM-2.
  \begin{itemize}
  \item $f(x) = 1/(1-x)$ gives expected commute time.
  \item $f(x) = 1/(1-x)^2$ gives a distance that, in comparison to
    expected commute time, assign longer paths higher weights.
  \item $f(x) = x^{2r}$ gives diffusion distance at time $r$.
  \item $f(x) = - \log{(1-x^2)}$ gives a distance that 
    take into account only paths of even lengths, with longer paths having
    lower weights.
  \item $f(x) = \exp(x)$ gives a distance that take into
    account paths of short length only, i.e. long paths have almost
    no weights.
  \end{itemize}
  \end{frame}

begin{frame}{Embedding $\bm{\Delta} = \kappa(f(\mathbf{P} -
    \mathbf{Q})\bm{\Pi}^{-1})$ in $\mathbb{R}^{d}$: Method 1}
  
  Embed $\bm{\Delta}$ by classical MDS.  
  \begin{enumerate}
  \item Compute 
    \begin{equation*}
      \mathbf{B} = \tau(\bm{\Delta}) = -\frac{1}{2}(\mathbf{I} -
      \bm{1}\bm{1}^{T}/n) f(\mathbf{P} - \mathbf{Q})\bm{\Pi}^{-1} (\mathbf{I} -
      \bm{1}\bm{1}^{T}/n)
    \end{equation*}
  \item Let $\lambda_1 \geq \lambda_2 \geq \dots \geq \lambda_n$
    denote the eigenvalues of $\mathbf{B}$ and let $\bm{v}_1,
    \bm{v}_2, \dots, \bm{v}_n$ denote the corresponding set of
    orthonormal eigenvectors. Then
    \begin{equation*}
      \mathbf{X} = \Bigl[ \sqrt{\lambda_1} \mathbf{v}_1 |
      \sqrt{\lambda_2} \mathbf{v}_{2} | \cdots |
      \sqrt{\lambda_d} \mathbf{v}_d \Bigr]
    \end{equation*}
    produces a configuration of points in $\mathbb{R}^{d}$.
  \end{enumerate}
\end{frame}
    
\begin{frame}{Embedding $\bm{\Delta} = \kappa(f(\mathbf{P} -
    \mathbf{Q})\bm{\Pi}^{-1})$ in $\mathbb{R}^{d}$: Method 2}

 Embed $\bm{\Delta}$ by the eigenvalues and eigenvectors of $\mathbf{P}$. 
  \begin{enumerate}
  \item Let $\mu_1, \mu_2, \dots, \mu_{n-1}$ be the eigenvalues of
    $\mathbf{P}$, sorted so that $f(\mu_{i}) \geq f(\mu_{i+1})$ and
    $\mu_i \not= 1$ for $1 \leq i \leq n - 1$, and let $\bm{u}_1,
    \bm{u}_2, \dots, \bm{u}_{n-1}$ denote the corresponding set of
    eigenvectors, orthonormal with respect to the inner product
    $\langle
    \bm{u}, \bm{v} \rangle_{\bm{\pi}} = \sum_{k}{u(k) v(k) \pi(k)}$.
  \item Then
    \begin{equation*}
      \mathbf{X} = \Bigl[ \sqrt{f(\mu_1)} \mathbf{u}_1 |
      \sqrt{f(\mu_2)} \mathbf{u}_{2} | \cdots |
      \sqrt{f(\mu_d)} \mathbf{u}_d \Bigr]
    \end{equation*}
    produces a configuration of points in $\mathbb{R}^{d}$.
  \end{enumerate}
\end{frame}

\begin{frame}
 \frametitle{Comparing the Embeddings}
  \begin{columns}[t]
  \begin{column}{0.46\textwidth}
    Method 1: Classical MDS
    \begin{enumerate}
    \item The embedding $\mathbf{X} = \Bigl[ \sqrt{\lambda_1} \mathbf{v}_1 |
        \cdots |
      \sqrt{\lambda_{n-1}} \mathbf{v}_{n-1} \Bigr]$ recovers
      $\bm{\Delta}$ completely.
    \item The embedding dimension of $\bm{\Delta}$
      is $n-1$ with probability $1$.
    \item The best (least squares) $d$-dim representation of
      $\mathbf{X}$ is $\mathbf{X}_d =  \Bigl[ \sqrt{\lambda_1} \mathbf{v}_1 |
       \cdots |
      \sqrt{\lambda_d} \mathbf{v}_{d} \Bigr]$.
    \item $\mathbf{X}_d \mathbf{X}_d^{T}$ is the best
      rank-d approximation of $\mathbf{B}$.
    \end{enumerate}
  \end{column}
  
  \begin{column}{0.54\textwidth}
    Method 2: Eigensystem of $\mathbf{P}$
    \begin{enumerate}
    \item The embedding 
      $\mathbf{X} = \Bigl[ \sqrt{f(\mu_1)} \mathbf{u}_1 |
       \cdots |
      \sqrt{f(\mu_{n-1})} \mathbf{u}_{n-1} \Bigr]$
      recovers 
      $\bm{\Delta}$ completely.
    \item The embedding dimension of $\bm{\Delta}$
     is $n-1$ with probability $1$. 
    \item The best (least squares) $d$-dim representation of
      $\mathbf{X}$ is (usually) \alert{not}
      $\mathbf{X}_d = \Bigl[ \sqrt{f(\mu_1)} \mathbf{u}_1 |
       \cdots |
      \sqrt{f(\mu_d)} \mathbf{u}_d \Bigr]$
   \item Embeddings for \alert{different} $f$ are (non-uniform)
      \alert{scaling} of one another.
    \end{enumerate}
  \end{column}
\end{columns}
\end{frame}

\begin{frame}
  \frametitle{Examples of Embeddings}
  \begin{itemize}
    \item Diffusion maps \cite{coifman06:_diffus_maps} is the
      embedding of diffusion distances using the eigenvalues and
      eigenvectors of $\mathbf{P}$. The $d$ dimensional embedding is
      \begin{equation*}
        \Bigl[ \mu_1^{r} \mathbf{u}_1 | \mu_{2}^{r} \mathbf{u}_2 |
        \cdots | \mu_{d}^{r}  \mathbf{u}_d \Bigr]
      \end{equation*}
    \item The embedding of expected commute time by CMDS turns out
      to be equivalent to embedding using the eigenvalues and
      eigenvectors of the combinatorial
      Laplacian. If $0 = \lambda_1 < \lambda_2 \leq \lambda_3 \leq \cdots
      \leq \lambda_{n}$ are the eigenvalues of the combinatorial
      Laplacian $\mathbf{L}$, and $\bm{v}_1, \bm{v}_2, \cdots,
      \bm{v}_n$ are the corresponding eigenvectors, then
      \begin{equation*}
        C \Bigl[ \tfrac{\bm{v}_{2}}{\lambda_2} |
        \tfrac{\bm{v}_3}{\lambda_3}
        | \cdots | \tfrac{\bm{v}_{d+1}}{\lambda_{d+1}} \Bigr]
      \end{equation*}
      is the $d$ dimensional embedding of
      $\bm{\Delta}_{\mathrm{ect}}$, where $C$ is a constant.  
  \end{itemize}
\end{frame}

\begin{frame}
\frametitle{Embedding of a 3D curve}
  \subfiglabelskip=0pt
  \begin{figure}[htbp]
    \label{fig:logistic}
    \centering
    \subfigure[][]{
      \includegraphics[width=50mm]{out.pdf}
    }
    \hspace{3pt}
    \subfigure[][]{
      \includegraphics[width=50mm]{logistic_ect.pdf}
    }
    \caption{Similarity is computed by $\gamma_{ij} = \exp(-\|x_i -
      x_j\|^{2}/\sigma)$, $\sigma = 0.4$.}
  \end{figure}

\end{frame}

% \begin{frame}
% \frametitle{Embedding Examples (cont'd)}
%   \begin{figure}[htbp]
%     \label{fig:reuter}
%      \includegraphics[height=7cm]{reuters.pdf}
%     \caption{Embedding of 1000 documents from the Reuters 21578 dataset}
%   \end{figure}
% \end{frame}

\begin{frame}
\frametitle{Paths of Even Length  \& Diffusion Distances}
  \subfiglabelskip=0pt
  \begin{figure}[htbp]
    \label{fig:two-step}
    \centering
    \subfigure[][]{
      \includegraphics[width=50mm]{twosteps_data.pdf}
    }
    \hspace{3pt}
    \subfigure[][]{
      \includegraphics[width=50mm]{twosteps_diffusion1.pdf}
    }
%    \caption{Two-step nature of diffusion distances}
  \end{figure}

\end{frame}

\begin{frame}
\bibliography{dissertation}
\end{frame}
\end{document}  
%%% Local Variables: 
%%% mode: latex
%%% TeX-master: t
%%% End: 
