The contribution of this work as follows. We present a novel framework
for the construction of Euclidean distances on undirected graphs. The
framework employs the concept of random walks on graphs. Expected
commute times and diffusion distances are examples of distance
measures that can be obtained under this framework. We also
investigate the question of how to embed the resulting Euclidean
distances. Two competing approaches are presented, namely the
embedding through classical MDS and the embedding through the
eigensystem of the transition matrix. We then show that several
well-known manifold learning algorithms, namely, Laplacian eigenmaps
and diffusion maps, can be interpreted as constructing algorithms that
embed the appropriate distance measures. This intepretation is novel
in the decoupling of the distance measures from the
embeddings. Furthermore, the decoupling of the distance measures from
the embeddings allows one to investigate the properties of these
algorithms from two different perspectives. As an example, we present
an anomaly that might arise in the embeddings constructed by diffusion
maps. This anomaly suggest itself when we look at the notion of
diffusion distances associated with diffusion maps. \\ \\
%
\noindent We then investigate the question of a framework for the construction
of Euclidean distances on directed graphs. Except for some scattered
results in the literature showing that a particular distance measure,
e.g., expected commute times, on directed graphs is a Euclidean
distance measure, there has not been a systematic investigation of
distance measures on directed graphs. We present two results
summarizing this investigation. The first result indicates that such a
framework, if it exists, would be much more restrictive than its
counterpart for undirected graphs. The second result is a framework
for constructing Euclidean distance measures on directed graphs that
is dependent of the graphs. This is in contrast to the framework for
undirected graphs which is not dependent on the graphs. \\ \\
%
\noindent We also considered the problem of embedding dissimilarity measures for
directed graphs. In particular, we show that, in contrast to
embeddings through classical MDS, the embeddings of the distance
measures using the eigensystems of matrices such as the transition
matrices or the Laplacian matrices do not extends to directed
graphs. Therefore, the current presentations in the literature of
manifold learning algorithms such as Laplacian eigenmaps and diffusion
maps do not extend naturally to directed graphs. This reaffirms our
view of decoupling the distance measures from the embeddings. Our last
contribution is a novel approach to embedding directed (assymetric)
proximity data. The approach can be understood as a hybrid three-way
MDS/asymmetric model. Thus, for example, we can embed the mean first
passage times directly, instead of the need to first symmetrize the
mean first passage times to expected commute times.

%%% Local Variables: 
%%% mode: latex
%%% TeX-master: "dissertation"
%%% End: 
