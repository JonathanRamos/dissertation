\documentclass[professionalfonts, hyperref={pdfpagelabels=false,
  colorlinks=true, linkcolor=purple}]{beamer}

\mode<presentation>{
  \usetheme{Boadilla}
  \useinnertheme{rectangles}
  \usecolortheme[cmyk={0,1,0.63,0.29}]{structure}
  \setbeamertemplate{navigation symbols}{}
}
\usepackage[english]{babel}
\usepackage[latin1]{inputenc}
\usepackage{subfigure}
\usepackage{bm}
\usepackage[all]{xy}
\bibliographystyle{plainnat}

%\usepackage{pgfpages}
%\pgfpagesuselayout{4 on 1}[letterpaper, landscape, border shrink=5mm]

\newtheorem{question}[theorem]{Question}
\newtheorem{openquestion}[theorem]{Open Question}
\setbeamercolor{question title}{bg = red}
\setbeamercolor{block body question}{bg=blue!60}
\AtBeginSection[]{
  \begin{frame}<beamer>
    \frametitle{Outline}
    \tableofcontents[currentsection,currentsubsection]
    \end{frame}
}

\begin{document}
\title[Graphs Metrics and Dimensionality Reduction]{Graphs Metrics and
  Dimensionality Reduction}
\author[Tang]{Minh Tang}
\institute[Indiana University]{
  School of Informatics and Computing \\
  Indiana University, Bloomington
}

\date{August 27, 2010}

\begin{frame}
\titlepage
\end{frame}

\begin{frame}
  \frametitle{Isomap} 
  Isomap \cite{tenebaum00:_global_geomet_framew_nonlin_dimen_reduc} is
  one of the best known manifold learning algorithm. Suppose that
  $y_1, y_2, \dots, y_n \in \mathbb{R}^{q}$ lie on a $d$-dimensional
  manifold. To represent $y_1, y_2, \dots, y_n$ as $x_1, x_2,
  \dots, x_n \in \mathbb{R}^{d}$, Isomap replaces Euclidean distance
  in $\mathbb{R}^{q}$ with a clever approximation of geodesic distance
  on the manifold as follows: 
  \vskip10pt
  \begin{enumerate}
  \item Replace Euclidean distance with approximate geodesic
    distance.
    \begin{enumerate}
    \item[(a)] Construct a weighted graph $G = (V,E,\omega)$ with $n$
      vertices. Fix some $\epsilon \geq 0$ and let $v_i \sim v_j$ iff
      $\|y_i - y_j\| \leq \epsilon$. If $v_i \sim v_j$, set
      $\omega_{ij} = \|y_i - y_j\|$.
    \item[(b)] Compute $\bm{\Delta} = (\delta_{ij})$ where
      $\delta_{ij}$ is the shortest path distance between $v_i$ and
      $v_j$ in $G$.
    \end{enumerate}
   \item Embed $\bm{\Delta}$ by CMDS.
  \end{enumerate}

\end{frame}

\begin{frame}{From Similarities to Distances on Graphs}
 The Isomap recipe can be adapted to work with similarities as
  follows.
  \vskip10pt Given a $n \times n$ similarities matrix $\bm{\Gamma} = (\gamma_{ij})$:
  \begin{enumerate}
  \item Transform the similarities to distances. (Isomap
    starts off with dissimilarties).
    \begin{enumerate}
    \item[(a)]Construct a weighted graph $G = (V,E,\omega)$ with $n$
      vertices and edge weights $\omega_{ij} = \gamma_{ij}$.
    \item[(b)] Construct a matrix $\bm{\Delta} = (\delta_{ij})$
      that measures some suitable distance on $G$. 
    \end{enumerate}
  \item Embed $\bm{\Delta}$. 
  \end{enumerate}
  Several popular approaches to transform from similarities to
   distances relies on the concept of a \alert{random walk}.
    
    \vskip10pt Assume that $G$ is connected. Let $\bm{s} =
    \bm{\Gamma}\bm{1}$ and $\mathbf{S} = \mathrm{diag}(\bm{s})$. Then
    the random walk on $G = (V,E,\omega)$ is the Markov chain with
    state space $V$ and transition probabilities $\mathbf{P} =
    \mathbf{S}^{-1}\bm{\Gamma}$. The stationary distribution
    $\bm{\pi}$ of $\mathbf{P}$ exists and is unique, and furthermore,
    $\lim_{k \rightarrow \infty} \mathbf{P}^{k} = \bm{1}\bm{\pi}^{T}
    := \mathbf{Q}$.
\end{frame}

%\section{Distances on Undirected Graphs}

\begin{frame}
  \frametitle{Expected Commute Time}
  Following \cite{kemeny83:_finit_markov_chain}, let
  \begin{equation*}
    \bm{\Pi} = \mathrm{diag}(\bm{\pi}) \quad \text{and} \quad
    \mathbf{Z} = (\mathbf{I} - \mathbf{P} + \mathbf{Q})^{-1}.
  \end{equation*}
  The expected first passage times are given by
  \begin{equation*}
    \mathbf{M} = (\mathbf{1}\mathbf{1}^{T}\mathrm{diag}(\mathbf{Z}) -
    \mathbf{Z})\bm{\Pi}^{-1} 
  \end{equation*}
  and the expected commute times are
  \begin{equation*}
    \bm{\Delta}_{\mathrm{ect}} = \mathbf{M} + \mathbf{M}^{T} =
    \kappa(\mathbf{Z}\bm{\Pi}^{-1})
  \end{equation*}
  It turns out that $\mathbf{Z}\bm{\Pi}^{-1} \succeq
  0$. $\bm{\Delta}_{\mathrm{ect}}$ is thus \alert{EDM-2}.
\end{frame}

\begin{frame}
  \frametitle{Diffusion Distances}
  Let $\bm{e}_i$ and $\bm{e}_j$ denote point masses at vertices $v_i$ and
  $v_j$. After $r$ time steps, under the random walk model with
  transition matrix $\mathbf{P}$, these distributions had diffused to
  $\bm{e}_i^{T} \mathbf{P}^{r}$ and $\bm{e}_j^{T}\mathbf{P}^{r}$. 
  
  \vskip10pt 

  The diffusion distance \cite{coifman06:_diffus_maps} at
  time $r$ between $v_i$ and $v_j$ is
    \begin{equation*}
      \rho_{r}(v_i,v_j) = \| \bm{e}_i^{T} \mathbf{P}^{r} -
      \bm{e}_j^{T}
      \mathbf{P}^{r} \|_{1/\bm{\pi}}
    \end{equation*}
    where the inner product $\langle \cdot, \cdot
    \rangle_{1/\bm{\pi}}$ is defined as
    \begin{equation*}
      \langle \bm{u}, \bm{v} \rangle_{1/\bm{\pi}} = \sum_{k} u(k)
      v(k)/\pi(k)
    \end{equation*}
    
      \vskip10pt
      It turns out that $\Delta_{\rho_{r}^{2}} =
      \kappa(\mathbf{P}^{2r}\bm{\Pi}^{-1})$. 
      Because $\mathbf{P}^{2r}\bm{\Pi}^{-1} \succeq 0$,
      $\Delta_{\rho_{r}^{2}}$ is EDM-2.  
\end{frame}

\begin{frame}{Some Remarks on ECT and Diffusion
  Distances}
  \begin{enumerate}
  \item $\bm{\Delta}_{\mathrm{ect}}$ can be written as
    \begin{equation*}
      \bm{\Delta}_{\mathrm{ect}} = \kappa(\mathbf{Z}\bm{\Pi}^{-1}) =
      \kappa\Bigl( \sum_{k=0}^{\infty}(\mathbf{P} -
      \mathbf{Q})^{k}\bm{\Pi}^{-1}\Bigr).
    \end{equation*}
    The expected commute time between $v_i$ and $v_j$ take into account
    paths of all length between $v_i$ and $v_j$.
  \item Even though $(\mathbf{P} - \mathbf{Q})^{k} =
    \mathbf{P}^{k} - \mathbf{Q}$ for $k \geq 1$,
    $\mathbf{Q}\bm{\Pi}^{-1} = \bm{1}\bm{1}^{T}$ and
    $\kappa(\bm{1}\bm{1}^{T}) = \bm{0}$, one cannot write
    $\bm{\Delta}_{\mathrm{ect}} =
    \kappa\Bigl(\sum_{k=0}^{\infty}\mathbf{P}^{k}\bm{\Pi}^{-1}\Bigr)$
    because $\sum_{k=0}^{\infty}\mathbf{P}^{k}\bm{\Pi}^{-1}$
    doesn't necessarily converge.
  \item $\bm{\Delta}_{\rho_{r}^{2}} =
    \kappa(\mathbf{P}^{2r}\bm{\Pi}^{-1}) = \kappa\bigl((\mathbf{P} -
    \mathbf{Q})^{2r}\bm{\Pi}^{-1}\bigr)$. Diffusion distance between
    $v_i$ and $v_j$ at time $r$ take into account only paths of length
    $2r$.
  \end{enumerate}
\end{frame}

\begin{frame}{General Framework for Euclidean Distances on Graphs}
  We introduce a general family of Euclidean distances constructed
  from random walks on graphs. 
  
  \vskip10pt Let $f$ be a real-valued function with a
    series expansion
    \begin{equation*}
      f(x) = a_0 + a_1 x + a_2 x^2 + \cdots
    \end{equation*}
    and radius of convergence $R \geq 1$. 
  
    \begin{alertblock}{}
      If $f(x) \geq 0$ for $x \in (-1,1)$ (and $\mathbf{P}$
      is irreducible and aperiodic), then
      \begin{equation*}
        \begin{split}
        \bm{\Delta} &= \kappa(f(\mathbf{P} - \mathbf{Q})
        \bm{\Pi}^{-1}) \\ &=
        \kappa\Bigl((a_0
        \mathbf{I} + a_1 (\mathbf{P} - \mathbf{Q}) + a_2 (\mathbf{P} -
        \mathbf{Q})^2 + \cdots)\bm{\Pi}^{-1}\Bigr)
        \end{split}
      \end{equation*}
      is well-defined and EDM-2. 
      \end{alertblock}

      In the above equation, $f$ acts on the
   matrix $\mathbf{P} - \mathbf{Q}$ and not on the entries of
    $\mathbf{P} - \mathbf{Q}$. 
\end{frame}

\begin{frame}{Euclidean Distances on Graphs: Some Examples}
  \begin{alertblock}{}
    \begin{equation*}
      \begin{split}
        \bm{\Delta} &= \kappa(f(\mathbf{P} - \mathbf{Q}) \bm{\Pi}^{-1}) \\ 
        &= \kappa\Bigl((a_0
        \mathbf{I} + a_1 (\mathbf{P} - \mathbf{Q}) + a_2 (\mathbf{P} -
        \mathbf{Q})^2 + \cdots)\bm{\Pi}^{-1}\Bigr)
      \end{split}
    \end{equation*}
  \end{alertblock}

  \vskip 10pt The following functions generate $\bm{\Delta}$ that are
  EDM-2.
  \begin{itemize}
  \item $f(x) = 1/(1-x)$ gives expected commute time.
  \item $f(x) = 1/(1-x)^2$ gives a distance that, in comparison to
    expected commute time, assign longer paths higher weights.
  \item $f(x) = x^{2r}$ gives diffusion distance at time $r$.
  \item $f(x) = - \log{(1-x^2)}$ gives a distance that 
    take into account only paths of even lengths, with longer paths having
    lower weights.
  \item $f(x) = \exp(x)$ gives a distance that take into
    account paths of short length only, i.e. long paths have almost
    no weights.
  \end{itemize}
  \end{frame}

\bibliography{dissertation}

\end{document}


%%% Local Variables: 
%%% mode: latex
%%% TeX-master: t
%%% End: 
