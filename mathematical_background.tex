\chapter{Mathematical Preliminaries}
\section{Graph Laplacians}
\label{sec:graph-laplacians}
We now introduce the concept of the Laplacian matrix of a graph. Our
exposition will be very superficial. For a more comprehensive account
of graph Laplacians, see
\citep{chung05:_laplac_cheeg,cvetkovic80:_spect_graph_theor_applic}.

Let $G = (V,E,\omega)$ be a simple, undirected graph with vertices set
$V$, edges set $E$ and similarity measure $\omega \colon E \mapsto
\mathbb{R}^{\geq 0}$. If $u$ and $v$ are vertices of $G$, we write $u
\sim v$ whenever $\{u,v\} \in E$. The degree of a vertex $v$ is
defined as $\deg(v) = \sum_{u \sim v}{\omega(\{u,v\})}$ and the volume
of $G$ is $\mathrm{Vol}(G) = \sum_{v \in V}{\deg(v)}$.  We denote by
$N$ the number of vertices of $G$. We define $D = (d_{ij})$ as the $N
\times N$ diagonal matrix with diagonal entries $d_{vv} = \deg(v)$.

\begin{definition}
  \label{def:1}
  Let $G = (V,E,\omega)$ be a simple, undirected graph with similarity
  measure $\omega$. The {\em combinatorial} Laplacian of $G$ is the
  matrix $L = L(G)$ with entries
  \begin{equation}
    \label{eq:1}
    L_{uv} = \begin{cases}
      - \omega(\{u,v\}) & \text{if $u \not = v$ and $u \sim v$} \\
      \deg(u) & \text{if $u = v$} \\
      0 & \text{otherwise}
    \end{cases}
  \end{equation}
  The {\em normalized} Laplacian of $G$ is the matrix $\mathcal{L} =
  \mathcal{L}(G)$ with entries
  \begin{equation}
    \label{eq:2}
    \mathcal{L}_{uv} = \begin{cases}
      - \tfrac{\omega(\{u,v\})}{\sqrt{\deg(u)}\sqrt{\deg(v)}} & \text{if $u \not = v$ and $u \sim v$} \\
      1 & \text{if $u = v$} \\
      0 & \text{otherwise}
    \end{cases}
  \end{equation}
\end{definition}
The following proposition lists some simple properties of the
combinatorial and normalized Laplacians. 
\begin{proposition}
  \label{prop:1}
  Let $G = (V,E,\omega)$ be a simple, undirected graph and $L$ and
  $\mathcal{L}$ be its combinatorial and normalized Laplacians,
  respectively. We have
  \begin{itemize}
  \item $L$ and $\mathcal{L}$ are symmetric, positive
    semi-definite matrices.
  \item $\mathcal{L} = D^{-1/2} L D^{-1/2}$
  \item The number of connected components of $G$ is equal to the
    number of zero eigenvalues of either $L$ or $\mathcal{L}$.
  \item The eigenvalues of $\mathcal{L}$ is at most $2$. 
  \end{itemize}
\end{proposition}
\section{Finite Markov Chain}
\label{sec:finite-markov-chain}
\begin{definition}
  \label{def:6}
  Let $\Omega$ be a finite or countably infinite set and
  $\mathbb{Z}^{*}$ be the set of non-negative integers. A sequence
  $\mathbf{X} = (X_n)_{n \in \mathbb{Z}^{*}}$ of random variables with values in
  $\Omega$ is a {\em Markov chain} if
  \begin{equation}
    \label{eq:8}
    \mathbb{P}[X_{n+1} = j \, | \, X_n = i, X_{n-1} = i_{n-1},
    \dots, X_0 = i_0] = \mathbb{P}[X_{n+1} = j \, | \, X_n = i] =
    p_{ij}
  \end{equation}
  for all $n \geq 0$ and all states $i_0, i_1, \dots, i_{n-1}, i,
  j$. The matrix $\mathbf{P}$, possibly infinite, with entries
  $\mathbf{P}(i,j) = p_{ij}$ is then termed the transition matrix of
  $(X_n)_{n \in \mathbb{Z}^*}$.
\end{definition}
Let $\mathbf{X} = (X_n)_{n \in \mathbb{Z}^*}$ be a Markov
chain. Denote by $p_{ij}^{(n)}$ the probability of going from state
$i$ to state $j$ in $n$ steps, i.e.,
\begin{equation}
  \label{eq:11}
  p_{ij}^{(n)} = \mathbb{P}[ X_{n + m } = j \, | \, X_m = i]
\end{equation}
for all $i, j \in \Omega$, and $m,n \in \mathbb{Z}^{*}$. Then
$p_{ij}^{(n)}$ satisfy the {\em Chapman-Kolmogorov equation}
\begin{equation}
  \label{eq:12}
  p_{ij}^{(m+n)} = \sum_{k \in \Omega}{p_{ik}^{(m)}p_{kj}^{(n)}}
\end{equation}
for all $m,n \in \mathbb{Z}^{*}$. Thus if $\mathbf{P}^{(n)}$ is the
matrix with entries $p_{ij}^{(n)}$, then $\mathbf{P}^{(m+n)} =
\mathbf{P}^{(m)}\mathbf{P}^{(n)}$. Since $\mathbf{P}^{(1)} =
\mathbf{P}$, we have
\begin{equation}
  \label{eq:13}
  \mathbf{P}^{(n)} = \mathbf{P}^{n}
\end{equation}
The behaviour of a Markov chain $\mathbf{X} = (X_n)_{n \in
  \mathbb{Z}^{*}}$ is thus completely specified by its transition
matrix $\mathbf{P}$. We can therefore view a Markov chain as being a
sequence of random variables generated by a transition matrix
$\mathbf{P}$. This view will be most helpful in the context of this
dissertation. However, since the transition matrix $\mathbf{P}$ only describes
the conditional probabilities, in order for us to compute the marginal
probabilities $\mathbb{P}[X_n = j]$, we need to specify an initial
distribution for $X_0$.

\begin{definition}
  \label{def:5}
  Let $\mathbf{X}$ be a Markov chain with state space
  $\Omega$. The initial distribution $\mu$ of $\mathbf{X}$ is a probability
  distribution on $\Omega$ such that 
  \begin{equation}
    \label{eq:14}
    \mu(i) = \mathbb{P}[X_0 = i]
  \end{equation}
  for all $i \in \Omega$. 
\end{definition}

\begin{definition}
  \label{def:7}
  Let $\mathbf{X}$ be a Markov chain with state space $\Omega$. Let
  $i$ and $j$ be elements of $\Omega$. $j$ is
  {\em accessible} from $i$, denoted as $i \rightarrow j$, if there
  exists a $n \in \mathbb{Z}^{*}$ such that $p_{ij}^{(n)} > 0$. If $i
  \rightarrow j$ and $j \rightarrow i$, then we say that $i$ and $j$
  {\em communicate}, and we write $i \leftrightarrow j$. A Markov chain is
  {\em irreducible} if $i \leftrightarrow j$ for any $i,j \in \Omega$.
\end{definition}
\begin{definition}
  \label{def:2}
  The stationary distribution $\pi$ of
  $\mathbf{X}$, if it exists, is a probability distribution on
  $\Omega$ such that
  \begin{equation}
    \label{eq:15}
    \pi(j) = \sum_{i \in \Omega}{\pi(i) p_{ij}}
  \end{equation}
  for any $j \in \Omega$. 
\end{definition}

\begin{proposition}
  \label{prop:3}
  If $\mathbf{X}$ is an irreducible Markov chain with state space
  $\Omega$, then there exists a unique stationary distribution $\pi$
  of $\mathbf{X}$, and that $\pi(i) > 0$ for all $i \in \Omega$. 
\end{proposition}

\begin{definition}
  \label{def:3}
  Let $\mathbf{X}$ be a Markov chain
  with transition matrix $\mathbf{P}$. Define 
  \begin{equation}
    \label{eq:5}
    \tau_i = \min\{ t \geq 0 \colon X_t = i \}, \qquad \tau_i^{+} \min
    \{ t \geq 1 \colon X_t = i \}
  \end{equation}
  The expected first passage time from $i$ to $j$, denoted by
  $\mathbb{E}_{i}[\tau_j]$, is defined as
  \begin{equation}
    \label{eq:6}
    \mathbb{E}_{i}[\tau_j] = \sum_{t = 0}^{\infty}{t \, \mathbb{P}(\tau_j =
      t \,|\, X_0 = i)}
  \end{equation}
  The expected first return time from $i$ to $i$, denoted by
  $\mathbb{E}_{i}[\tau_i^{+}]$, is defined as
  \begin{equation}
    \label{eq:7}
    \mathbb{E}_{i}[\tau_i^{+}] = \sum_{t = 1}^{\infty}{t \,
      \mathbb{P}(\tau_v^{+} = t \,|\, X_0 = i)}
  \end{equation}
  $\tau_i$ and $\tau_{i}^{+}$ as declared above are examples of {\em
    stopping times}. 
\end{definition}

\begin{proposition}
  \label{prop:2}
  Let $\mathbf{X}$ be an irreducible
  Markov chain with transition matrix $\mathbf{P}$ and stationary
  distribution $\pi$. We then have that
  \begin{equation}
    \label{eq:9}
    \mathbb{E}_{i}[\tau_i^{+}] = \frac{1}{\pi(i)}
  \end{equation}
\end{proposition}

\begin{definition}
  \label{def:9}
  Let $\mathbf{X}$ be an irreducible Markov chain with transition
  matrix $\mathbf{P}$ and stationary distribution
  $\pi$. $\hat{\mathbf{P}} = (\hat{p}_{ij})$ is said to be the {\em
    time reversal} of $\mathbf{P}$ if, for all pairs $i,j \in \Omega$,
  one has
  \begin{gather}
    \label{eq:16}
    \pi(i) p_{ij} = \pi(j) \hat{p}_{ji} \\
    \shortintertext{or, in other words}
    \label{eq:78}
    \hat{\mathbf{P}} = \bm{\Pi}^{-1} \mathbf{P}^{T} \bm{\Pi} 
  \end{gather}
  $\mathbf{P}$ is said to be {\em time-reversible} if
  $\hat{\mathbf{P}} = \mathbf{P}$.
\end{definition}
Now $\hat{\mathbf{P}}$ also defines a Markov chain
$\hat{\mathbf{X}}$. $\hat{\mathbf{X}}$ will be termed the
time-reversed Markov chain with respect to $\mathbf{X}$. $\pi$ is also
the stationary distribution of $\hat{\mathbf{P}}$ and that
\begin{equation}
  \label{eq:17}
  \mathbb{P}[X_n = j, \dots, X_0 = i] = \mathbb{P}[\hat{X}_0 = i,
  \dots, \hat{X}_n = j] 
\end{equation}
where the initial distribution of $X_0$ and $\hat{X}_0$ are both
identical to the stationary distribution $\pi$.
\section{Random walks on graphs}
\label{sec:random-walks-graphs}
Let $G = (V,E,\omega)$ be a simple, undirected graph. We define the transition
matrix $\mathbf{P}_G = (p_{uv})$ of a Markov chain with state space $V$ as follows
\begin{equation}
  \label{eq:20}
  p_{uv} = \begin{cases}
    \tfrac{\omega(\{u,v\})}{\deg(u)} & \text{if $u \sim v$} \\
    0 & \text{otherwise}
  \end{cases}
\end{equation}
We now note some properties of the Markov chain $\mathbf{X}$ generated
by $\mathbf{P}_G$
\begin{proposition}
  \label{prop:15}
  Let $G$ be an undirected graph and $\mathbf{P}$ be the transition
  matrix on $G$. Let $\mathbf{X}$ be the Markov chain generated by
  $\mathbf{P}$. Then 
\begin{itemize}
\item $\mathbf{X}$ is irreducible if and only if $G$.
\item If $\mathbf{X}$ is irreducible, $\pi(v) =
  \tfrac{\deg(v)}{\mathrm{Vol}(G)}$ for all $v \in V$.
\item $\mathbf{P}$ is time-reversible. Therefore, $\bm{\Pi}\mathbf{P} = \mathbf{P}^{T}\bm{\Pi}$.
\end{itemize}
\end{proposition}
%
We can also define the transition matrix $\mathbf{P}_G$ when $G$ is
directed. $\mathbf{P}_G$ will have entries
\begin{equation}
  \label{eq:18}
  p_{uv} = \begin{cases}
    \tfrac{\omega(e)}{\deg(u)} & \text{if $e = (u,v) \in E$} \\
    0 & \text{otherwise}
  \end{cases}
\end{equation}
If $G$ is directed, then $\mathbf{X}$ is irreducible if and only if
$G$ is strongly connected. However, $\mathbf{P}$ is in general not
time reversible and there's no explicit expression for the
stationary distribution $\pi$ of $\mathbf{P}$. \\ \\
%
%
\noindent Let $G = (V,E)$ be a graph, directed or undirected, and $\mathbf{P}$
be its transition matrix. A function $f \colon V \mapsto \mathbb{R}$
is {\em harmonic} at $v \in V$ if
\begin{equation}
  \label{eq:10}
  f(v) = \sum_{w \in V}{\mathbf{P}(v,w) f(w)}
\end{equation}
$f$ is harmonic on $V$ if it's harmonic for all $v \in V$. If $\mathbf{P}$
is irreducible, we have a simple characterization for harmonic
functions on $V$. Specifically,
\begin{lemma}
  \label{lem:1}
  Suppose that $\mathbf{P}$ is irreducible. A function $f \colon V \mapsto
  \mathbb{R}$ is harmonic on $V$ if and only if $f$ is constant on
  $V$. 
\end{lemma}
\begin{proof}
  It's easy to see that if $f$ is constant on $V$ then it's also
  harmonic on $V$. Thus, let's assume that $f$ is harmonic on $V$.
  Let $v_*$ be a node such that $f(v_*) \geq f(w)$ for all $w \in
  V$. Since $f$ is harmonic, from Eq.~\eqref{eq:10} we have that $f(w)
  = f(v_*)$ for all $w$ such that $\mathbf{P}(v_*,w) > 0$. We thus see
  that every vertex $w$ that's accessible from $v_*$ will satisfy
  $f(w) = f(v_*)$. Since $\mathbf{P}$ is irreducible, $f(v_*) = f(w)$ for
  all $w \in V$. $f$ is thus constant on $V$.
\end{proof}

\begin{proposition}
  \label{prop:6}
  Let $G = (V,E)$ be a graph and $\mathbf{P}$ be its transition
  matrix. Suppose that the Markov chain defined by $\mathbf{P}$ is
  regular. Then there exists a unique stationary distribution $\pi$ of
  $\mathbf{P}$. Furthermore, if $\mathbf{Q} = \mathbf{1} \mathbf{\pi}^{T}$ is the
  matrix with each row being the stationary distribution, then
  \begin{equation}
    \label{eq:22}
    \lim_{k \rightarrow \infty}(\mathbf{P} - \mathbf{Q})^{k} = 0 
  \end{equation}
  Eq.~\eqref{eq:22} is equivalent to $\rho(\mathbf{P}-\mathbf{Q}) < 1$
  where $\rho(\mathbf{P}-\mathbf{Q})$ is the spectral radius of
  $\mathbf{P} - \mathbf{Q}$.
\end{proposition}

\begin{proposition}
  \label{prop:7}
  Let $G = (V,E)$ be a graph and $\mathbf{P}$ be its transition
  matrix. Suppose that $\mathbf{P}$ is regular. Then the matrix $\mathbf{Z} =
  (\mathbf{I} - \mathbf{P} + \mathbf{Q})^{-1}$ exists and is given by
  \begin{equation}
    \label{eq:28}
    \mathbf{Z} = \sum_{k=0}^{\infty}(\mathbf{P} - \mathbf{Q})^{k} = \mathbf{I} +
    \sum_{k=1}^{\infty}(\mathbf{P}^{k} - \mathbf{Q})
  \end{equation}
  
\end{proposition}
\begin{proof}
  Since $\mathbf{P}$ is regular, by Proposition \ref{prop:6}, $\lim_{k
    \rightarrow \infty}(\mathbf{P} - \mathbf{Q})^{k} = 0$. Thus, $\mathbf{Z}$ has
  an expansion in term of a Neumann series
  \begin{equation}
    \label{eq:29}
    \mathbf{Z} = \sum_{k=0}^{\infty}(\mathbf{P} - \mathbf{Q})^{k}
  \end{equation}
  Since $\mathbf{P}\mathbf{Q} = \mathbf{P}1^{T}\mathbf{\pi} = 1^{T}\mathbf{\pi} = 
  1^{T}\mathbf{\pi}\mathbf{P} = \mathbf{Q}\mathbf{P} = \mathbf{Q}$,  
  one has $(\mathbf{P} - \mathbf{Q})^{k} = \mathbf{P}^{k} - \mathbf{Q}$ for $k \geq
  1$. Eq.~\eqref{eq:28} thus follows. 
\end{proof}
The matrix $\mathbf{Z}$ is termed the {\em fundamental matrix}
\citep{kemeny83:_finit_markov_chain}. Some properties of
$\mathbf{Z}$ are given in the following proposition.
\begin{proposition}
  \label{prop:8}
  Let $\mathbf{P}$ be the transition matrix of a regular Markov chain and
  $\mathbf{Z}$ be its fundamental matrix. We have
  \begin{enumerate}[(i)]
  \item $\mathbf{P}\mathbf{Z} = \mathbf{Z} - \mathbf{I} + \mathbf{Q}$. 
  \item $(\mathbf{I} - \mathbf{P})\mathbf{Z} = \mathbf{I} - \mathbf{Q}$.
  \item $\mathbf{Z} \mathbf{J} = \mathbf{J}$. 
  \end{enumerate}
\end{proposition}
\begin{proof}
  $\mathbf{P}\mathbf{Z} = \mathbf{P} - \sum_{k=1}^{\infty}(\mathbf{P}^{k+1} - \mathbf{Q})
  = \mathbf{Z} - \mathbf{I} + \mathbf{Q}$. (i) and (ii) thus follows. For (iii),
  note that $\mathbf{P}^{k}\mathbf{J} = \mathbf{Q}\mathbf{J} = \mathbf{J}$. 
\end{proof}
\section{Distance Geometry}
\label{sec:distance-geometry}
We discuss in this section some notations and results regarding
distance matrices. The notion of an Euclidean distance matrix (EDM) is
of particular importance to our discussion and is defined in
Definition \ref{def:1}. We then introduce two linear transformations
between matrices, the $\kappa$ transform and the $\tau$ transform.
Schoenberg's characterization \citep{schoenberg38:_metric} of Euclidean
distance matrices in terms of positive semidefinite matrices are
stated in Theorem \ref{thm:5}. We also state some simple results
regarding the $\kappa$ transforms that are useful in the context of
this work.

\begin{definition}
  \label{def:10}
  Let $\Delta = (\delta_{ij}) \in M_n(\mathbb{R})$. $\Delta$ is a Type 1
  Euclidean distance matrix (EDM-$1$) if and only if there exists a
  positive integer $p$ and $x_1, x_2, \dots, x_n \in \mathbb{R}^{p}$
  such that $\delta_{ij} = \| x_i - x_j \|$. $\Delta$ is a Type 2
  Euclidean distance matrix (EDM-$2$) if and only if there exists a
  $p$ and $x_1, x_2, \dots, x_n \in \mathbb{R}^{p}$ such that
  $\delta_{ij} = \|x_i - x_j\|^{2}$. The {\em embedding dimension} of
  $\Delta$ is the minimum $p$ such that a configuration of points
  $x_1, x_2, \dots, x_n$ exists with the desired property.
\end{definition}

\begin{definition}
  \label{def:11}
  Let $\mathbf{A} \in M_n(\mathbb{R})$. Define a linear mapping $\tau \colon M_n(\mathbb{R})
  \mapsto M_n(\mathbb{R})$ by
  \begin{equation}
    \label{eq:55}
    \tau(\mathbf{A}) = - \frac{1}{2} \Bigl(\mathbf{I} -
    \frac{\mathbf{J}}{n}\Bigr)\mathbf{A} \Bigl(\mathbf{I} - \frac{\mathbf{J}}{n}\Bigr)
  \end{equation}
  If $a_{ij}$ are the entries of $\mathbf{A}$ then
  \begin{equation}
    \label{eq:56}
    b_{ij} = -\frac{1}{2}\Bigl(a_{ij} - \frac{1}{n}\sum_{j=1}^{n}a_{ij} -
    \frac{1}{n}\sum_{i=1}^{n}{a_{ij}} +
      \frac{1}{n^2}\sum_{i=1}^{n}\sum_{j=1}^{n}a_{ij}\Bigr)
  \end{equation}
  are the entries of $\mathbf{B} = \tau(\mathbf{A})$. $\tau$
  is a continuous mapping from $M_n(\mathbb{R})$ to
  $M_n(\mathbb{R})$. 
\end{definition}
%
The following result provides a necessary and sufficient condition for
$\Delta$ to be an EDM-2 matrix.
\begin{theorem}[\citep{schoenberg38:_metric}]
  \label{thm:5}
  $\Delta$ is an EDM-2 with embedding dimension $p$ if and only
  if $\mathbf{B} = \tau(\Delta)$ is positive semidefinite with rank
  $p$. 
\end{theorem}

\begin{definition}
  \label{def:12}
  Let $\mathbf{A} \in M_n(\mathbb{R})$. Define a linear mapping
  $\kappa \colon M_n(\mathbb{R}) \mapsto M_n(\mathbb{R})$ by
  \begin{equation}
    \label{eq:61}
    \kappa(\mathbf{A}) = \mathbf{J}\mathbf{A}_{\mathrm{dg}} -
    \mathbf{A} - \mathbf{A}^{T} + \mathbf{A}_{\mathrm{dg}}\mathbf{J}
  \end{equation}
  where $\mathbf{A}_{\mathrm{dg}}$ is the diagonal matrix obtained by
  setting the off-diagonal entries of $\mathbf{A}$ to $0$. If $a_{ij}$
  are the entries of $\mathbf{A}$ then
  \begin{equation}
    \label{eq:70}
    b_{ij} = a_{ii} - a_{ij} - a_{ji} + a_{jj}
  \end{equation}
  are the entries of $\mathbf{B} = \kappa(\mathbf{A})$. $\kappa$ is
  also a continuous mapping from $M_n(\mathbb{R})$ to
  $M_n(\mathbb{R})$. 
\end{definition}

\begin{proposition}
  \label{prop:16}
  The $\kappa$ transform has the following properties.
  \begin{itemize}
  \item[(A)] Let $\mathcal{C} = \{ \mathbf{A} \in S_n(\mathbb{R}) \colon
    \mathbf{C}\bm{1}_{n}^{T} = \bm{0} \}$ be the set of symmetric
    matrices with zero row sums and let $\mathcal{D} = \{ \Delta \in
    S_n(\mathbb{R}) \colon \Delta_{\mathrm{dg}} = 0 \}$ be the set of
    symmetric hollow matrices. Then $\kappa$ and $\tau$ are inverse
    mappings between $\mathcal{C}$ and $\mathcal{D}$, i.e.,
  \begin{gather}
    \label{eq:55}
    \mathbf{A} \in \mathcal{C}
    \Longrightarrow \Delta = \kappa(\mathbf{A}) \in \mathcal{D}, \,\,
    \mathbf{A} = \tau(\Delta) \\
    \Delta \in \mathcal{D} \Longrightarrow \mathbf{A} = \tau(\Delta)
    \in \mathcal{C}, \,\, \Delta = \kappa(\mathbf{A})
  \end{gather}
  \item[(B)] $\kappa(\mathbf{J}) = 0$. More generally,
    $\kappa(\bm{a}\bm{1}^{T}) = \kappa(\bm{1}\bm{b}^{T}) = 0$
    for any vector $\bm{a}$, $\bm{b}$.
  \item[(C)] Let $\tilde{\mathbf{X}}$ be the double centering of
    $\mathbf{X}$, i.e.,
  \begin{equation}
    \label{eq:71}
    \tilde{\mathbf{X}} = \Bigl(\mathbf{I} - \frac{\mathbf{J}}{n}\Bigr)\mathbf{X} \Bigl(\mathbf{I} - \frac{\mathbf{J}}{n}\Bigr)
  \end{equation}
  Then $\kappa(\tilde{\mathbf{X}}) = \kappa(\mathbf{X})$.
  \end{itemize}
\end{proposition}
Part (A) of Proposition \ref{prop:16} is from
\citep{critchley88:_certain_linear_mappin}. Part (B) and (C)
follows directly from the definition of the $\kappa$ transform. 
\begin{proposition}
  \label{prop:18}
  Let $\mathbf{A} \in S_n(\mathbb{R})$ be a positive semidefinite
  matrix. Then $\Delta = \kappa(\mathbf{A})$ is EDM-2.
\end{proposition}
\begin{proof}
  The double centering $\tilde{\mathbf{A}}$ of $\mathbf{A}$ is a
  matrix in $\mathcal{C}$. By Proposition
  \ref{prop:16}, $\Delta = \kappa(\mathbf{A}) =
  \kappa(\tilde{\mathbf{A}})$ and that $\tilde{\mathbf{A}} =
  \tau(\Delta)$. Now $\mathbf{A} \succeq 0$ implies
  $\tilde{\mathbf{A}} \succeq 0$. By Schoenberg's criterion, $\Delta =
  \kappa(\tilde{\mathbf{A}})$ is EDM-2. 
\end{proof}
%
\section{Matrix Analysis}
We listed here some results in matrix analysis that are useful within
the scope of this dissertation.
%
Let $\mathbf{A} = (a_{ij})$ be an $n \times n$ matrix with real
entries $a_{ij}$. Denote by $R_i$ the sum $\sum_{j \not = i}{|a_{ij}|}$ of
off-diagonal elements in row $i$. 
%
\begin{theorem}[\citep{gersgorin31:_uber_abgren_eigen_matrix}]
  \label{thm:1}
  Let $\mathbf{A}$ be an $n \times n$ matrix with off-diagonal row
  sums $R_i$. Then the eigenvalue of $\mathbf{A}$ lies in the set
  \begin{equation}
    \label{eq:23}
    \bigcup \{z \in \mathbb{C} \colon |z - a_{ii}| \leq R_i \}
  \end{equation}
\end{theorem}
\begin{definition}
  \label{def:4}
  The matrix $\mathbf{A}$ is said to be diagonally dominant if
  $|a_{ii}| \geq R_i$ for all $i$ and strictly diagonally dominant if
  $|a_{ii}| > R_i$ for all $i$.
\end{definition}
If $\mathbf{A}$ is diagonally dominant, then by Ger\u{s}gorin's
circle theorem, the eigenvalues of $\mathbf{A}$ has nonnegative real
parts. If $\mathbf{A}$ is strictly diagonally dominant, then the
eigenvalues of $\mathbf{A}$ has positive real parts. 
%
\begin{definition}
  \label{def:8}
  Let $Z_n \subset M_{n}(\mathbb{R})$ be the set of matrices with
  non-positive off-diagonal entries, i.e.,
  \begin{equation}
    \label{eq:24}
    Z_n = \{ \mathbf{A} = (a_{ij}) \in M_{n}(\mathbb{R}) \colon a_{ij}
    \leq 0 \,\, \text{if $i \not = j$} \}
  \end{equation}
 A matrix $\mathbf{A} \in Z_n$ is called an $M$-matrix if $A$ is
 positive stable, i.e., if the eigenvalues of $\mathbf{A}$ has
 positive real parts.
\end{definition}
A relationship between $M$-matrices and non-negative matrices is given
by the following result \citep[see][\S 2.5]{horn94:_topic_in_matrix_analy}.
\begin{theorem}
  \label{thm:2}
  $\mathbf{A} \in Z_n$ is a $M$-matrix if and only if $\mathbf{A}$ is
  non-singular and $\mathbf{A}^{-1} \geq 0$.  
\end{theorem}
%%% Local Variables: 
%%% mode: latex
%%% TeX-master: "dissertation"
%%% End: 
