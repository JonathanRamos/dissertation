\section{Metrics on Graphs}
\subsection{Expected commute time}
\label{sec:resistance-distances}
Let $G = (V,E,\omega)$ be a simple, undirected graph with $\omega$
being the similarity measure. The {\em expected commute time}
$\delta(u,v)$ between $u \in V$ and $v \in V$ is defined as
\begin{equation}
  \label{eq:25}
  \delta(u,v) = \frac{\mathbb{E}_{u}[\tau_v] +
    \mathbb{E}_{v}[\tau_u]}{\mathrm{Vol}(G)}
\end{equation}

\begin{proposition}
  \label{prop:4}
  Let $\bm{M}$ be the matrix of first passage time, i.e. $\bm{M}(u,v)
  = \mathbb{E}_{u}[\tau_v]$. $\bm{M}$ is then the unique solution of the
  following matrix equation
  \begin{equation}
    \label{eq:3}
   (\bm{I} - \bm{P})\bm{X} = \bm{J} - \mathrm{Vol}(G) \bm{D}^{-1}
  \end{equation}
  subjected to the condition $\mathrm{diag}(\bm{M}) = \bm{0}$. 
\end{proposition}
\begin{proof}
  If $u = v$, then $\mathbb{E}_{u}[\tau_u] = 0$ and thus $\bm{M}(u,u)
  = 0$. Otherwise, if $u \not = v$, then $\bm{M}(u,v) = \mathbb{E}_{u}[\tau_v]$ can be
  expanded as
  \begin{equation}
    \label{eq:4}
    \mathbb{E}_{u}[\tau_v] = \sum_{w \in V}{\bm{P}(u,w)(1 +
      \mathbb{E}_{w}[\tau_v])} = 1 + \sum_{w \in V}{\bm{P}(u,w)
      \mathbb{E}_{w}[\tau_v]} = 1 + (\bm{PM})(u,v)
  \end{equation}
  Thus, $\bm{F} = \bm{J} + (\bm{P} - \bm{I})\bm{M}$ is a diagonal
  matrix. Futhermore, $\pi^{T} \bm{F} = \pi^{T} \bm{J} +
  \pi^{T}(\bm{P} - \bm{I})\bm{M} = \pi^{T} \bm{J} =
  \bm{1}$. Therefore, $\bm{F}(u,u) = 1/\pi(u) =
  \mathrm{Vol}(G)/\deg(u)$, and thus $\bm{F} = \mathrm{Vol}(G)
  \bm{D}^{-1}$. $\bm{M}$ is thus a solution of the matrix equation
  as given by Eq.~\eqref{eq:3}. 

  We now show that $\bm{M}$ is the unique solution of Eq.~\eqref{eq:3}
  subjected to the condition $\mathrm{diag}(\bm{M}) = \bm{0}$. Let
  $\bm{M}'$ be another solution of Eq.~\eqref{eq:3} subjected to the
  condition $\mathrm{diag}(\bm{M'}) = 0$. Then $\bm{X} =
  \bm{M} - \bm{M}'$ satisfy
  \begin{equation}
    \label{eq:19}
    (\bm{I} - \bm{P})\bm{X} = \bm{0}
  \end{equation}
  By Lemma \ref{lem:1}, each column of $\bm{X}$ is constant. Since
  $\mathrm{diag}(\bm{M}) = \mathrm{diag}(\bm{M'}) = \bm{0}$, each
  column of $\bm{X}$ must be identically $0$. Thus $\bm{M} = \bm{M'}$,
  proving the uniqueness of $\bm{M}$. 
\end{proof}

\begin{proposition}
  \label{prop:5}
  
\end{proposition}


\section{Mathematical Preliminaries}

\subsection{Graph Laplacians}
\label{sec:graph-laplacians}
We now introduce the concept of the Laplacian matrix of a graph. Our
exposition will be very superficial. For a more comprehensive account
of graph Laplacians, please consult \cite{chung05:_laplac_cheeg,cvetkovic80:_spect_graph_theor_applic}.

Let $G = (V,E,\omega)$ be a simple, undirected graph with vertices set
$V$, edges set $E$ and similarity measure $\omega \colon E \mapsto
\mathbb{R}^{\geq 0}$. If $u$ and $v$ are vertices of $G$, we write $u \sim v$
whenever $\{u,v\} \in E$. The degree of a vertex $v$ is defined as
$\deg(v) = \sum_{u \sim v}{\omega(\{u,v\})}$ and the volume of $G$ is
$\mathrm{Vol}(G) = \sum_{v \in V}{\deg(v)}$.  We denote by $N$ the
number of vertices of $G$. We define $D = (d_{ij})$ as the $N \times
N$ diagonal matrix with diagonal entries $d_{vv} = \deg(v)$.

\begin{definition}
  \label{def:1}
  Let $G = (V,E,\omega)$ be a simple, undirected graph with similarity
  measure $\omega$. The {\em combinatorial} Laplacian of $G$ is the
  matrix $L = L(G)$ with entries
  \begin{equation}
    \label{eq:1}
    L_{uv} = \begin{cases}
      - \omega(\{u,v\}) & \text{if $u \not = v$ and $u \sim v$} \\
      \deg(u) & \text{if $u = v$} \\
      0 & \text{otherwise}
    \end{cases}
  \end{equation}
  The {\em normalized} Laplacian of $G$ is the matrix $\mathcal{L} =
  \mathcal{L}(G)$ with entries
  \begin{equation}
    \label{eq:2}
    \mathcal{L}_{uv} = \begin{cases}
      - \tfrac{\omega(\{u,v\})}{\sqrt{\deg(u)}\sqrt{\deg(v)}} & \text{if $u \not = v$ and $u \sim v$} \\
      1 & \text{if $u = v$} \\
      0 & \text{otherwise}
    \end{cases}
  \end{equation}
\end{definition}
The following proposition lists some simple properties of the
combinatorial and normalized Laplacians. 
\begin{proposition}
  \label{prop:1}
  Let $G = (V,E,\omega)$ be a simple, undirected graph and $L$ and
  $\mathcal{L}$ be its combinatorial and normalized Laplacians,
  respectively. We have
  \begin{itemize}
  \item $L$ and $\mathcal{L}$ are symmetric, positive
    semi-definite matrices.
  \item $\mathcal{L} = D^{-1/2} L D^{-1/2}$
  \item The number of connected components of $G$ is equal to the
    number of zero eigenvalues of either $L$ or $\mathcal{L}$.
  \item The eigenvalues of $\mathcal{L}$ is at most $2$. 
  \end{itemize}
\end{proposition}
\subsection{Finite Markov Chain}
\begin{definition}
  \label{def:6}
  Let $\Omega$ be a finite or countably infinite set and
  $\mathbb{Z}^{*}$ be the set of non-negative integers. A sequence
  $\mathbf{X} = (X_n)_{n \in \mathbb{Z}^{*}}$ of random variables with values in
  $\Omega$ is a {\em Markov chain} if
  \begin{equation}
    \label{eq:8}
    \mathbb{P}[X_{n+1} = j \, | \, X_n = i, X_{n-1} = i_{n-1},
    \dots, X_0 = i_0] = \mathbb{P}[X_{n+1} = j \, | \, X_n = i] =
    p_{ij}
  \end{equation}
  for all $n \geq 0$ and all states $i_0, i_1, \dots, i_{n-1}, i,
  j$. The matrix $\mathbf{P}$, possibly infinite, with entries
  $\mathbf{P}(i,j) = p_{ij}$ is then termed the transition matrix of
  $(X_n)_{n \in \mathbb{Z}^*}$.
\end{definition}
Let $\mathbf{X} = (X_n)_{n \in \mathbb{Z}^*}$ be a Markov
chain. Denote by $p_{ij}^{(n)}$ the probability of going from state
$i$ to state $j$ in $n$ steps, i.e.,
\begin{equation}
  \label{eq:11}
  p_{ij}^{(n)} = \mathbb{P}[ X_{n + m } = j \, | \, X_m = i]
\end{equation}
for all $i, j \in \Omega$, and $m,n \in \mathbb{Z}^{*}$. Then
$p_{ij}^{(n)}$ satisfy the {\em Chapman-Kolmogorov equation}
\begin{equation}
  \label{eq:12}
  p_{ij}^{(m+n)} = \sum_{k \in \Omega}{p_{ik}^{(m)}p_{kj}^{(n)}}
\end{equation}
for all $m,n \in \mathbb{Z}^{*}$. Thus if $\mathbf{P}^{(n)}$ is the
matrix with entries $p_{ij}^{(n)}$, then $\mathbf{P}^{(m+n)} =
\mathbf{P}^{(m)}\mathbf{P}^{(n)}$. Since $\mathbf{P}^{(1)} =
\mathbf{P}$, we have
\begin{equation}
  \label{eq:13}
  \mathbf{P}^{(n)} = \mathbf{P}^{n}
\end{equation}
The behaviour of a Markov chain $\mathbf{X} = (X_n)_{n \in
  \mathbb{Z}^{*}}$ is thus completely specified by its transition
matrix $\mathbf{P}$. We can therefore view a Markov chain as being a
sequence of random variables generated by a transition matrix
$\mathbf{P}$. This view will be most helpful in the context of this
dissertation. However, since the transition matrix $\mathbf{P}$ only describes
the conditional probabilities, in order for us to compute the marginal
probabilities $\mathbb{P}[X_n = j]$, we need to specify an initial
distribution for $X_0$.

\begin{definition}
  \label{def:5}
  Let $\mathbf{X}$ be a Markov chain with state space
  $\Omega$. The initial distribution $\mu$ of $\mathbf{X}$ is a probability
  distribution on $\Omega$ such that 
  \begin{equation}
    \label{eq:14}
    \mu(i) = \mathbb{P}[X_0 = i]
  \end{equation}
  for all $i \in \Omega$. 
\end{definition}

\begin{definition}
  \label{def:7}
  Let $\mathbf{X}$ be a Markov chain with state space $\Omega$. Let
  $i$ and $j$ be elements of $\Omega$. $j$ is
  {\em accessible} from $i$, denoted as $i \rightarrow j$, if there
  exists a $n \in \mathbb{Z}^{*}$ such that $p_{ij}^{(n)} > 0$. If $i
  \rightarrow j$ and $j \rightarrow i$, then we say that $i$ and $j$
  {\em communicate}, and we write $i \leftrightarrow j$. A Markov chain is
  {\em irreducible} if $i \leftrightarrow j$ for any $i,j \in \Omega$.
\end{definition}
\begin{definition}
  \label{def:2}
  The stationary distribution $\pi$ of
  $\mathbf{X}$, if it exists, is a probability distribution on
  $\Omega$ such that
  \begin{equation}
    \label{eq:15}
    \pi(j) = \sum_{i \in \Omega}{\pi(i) p_{ij}}
  \end{equation}
  for any $j \in \Omega$. 
\end{definition}

\begin{proposition}
  \label{prop:3}
  If $\mathbf{X}$ is an irreducible Markov chain with state space
  $\Omega$, then there exists a unique stationary distribution $\pi$
  of $\mathbf{X}$, and that $\pi(i) > 0$ for all $i \in \Omega$. 
\end{proposition}

\begin{definition}
  \label{def:3}
  Let $\mathbf{X}$ be a Markov chain
  with transition matrix $\mathbf{P}$. Define 
  \begin{equation}
    \label{eq:5}
    \tau_i = \min\{ t \geq 0 \colon X_t = i \}, \qquad \tau_i^{+} \min
    \{ t \geq 1 \colon X_t = i \}
  \end{equation}
  The expected first passage time from $i$ to $j$, denoted by
  $\mathbb{E}_{i}[\tau_j]$, is defined as
  \begin{equation}
    \label{eq:6}
    \mathbb{E}_{i}[\tau_j] = \sum_{t = 0}^{\infty}{t \, \mathbb{P}(\tau_j =
      t \,|\, X_0 = i)}
  \end{equation}
  The expected first return time from $i$ to $i$, denoted by
  $\mathbb{E}_{i}[\tau_i^{+}]$, is defined as
  \begin{equation}
    \label{eq:7}
    \mathbb{E}_{i}[\tau_i^{+}] = \sum_{t = 1}^{\infty}{t \,
      \mathbb{P}(\tau_v^{+} = t \,|\, X_0 = i)}
  \end{equation}
  $\tau_i$ and $\tau_{i}^{+}$ as declared above are examples of {\em
    stopping times}. 
\end{definition}

\begin{proposition}
  \label{prop:2}
  Let $\mathbf{X}$ be an irreducible
  Markov chain with transiton matrix $\mathbf{P}$ and stationary
  distribution $\pi$. We then have that
  \begin{equation}
    \label{eq:9}
    \mathbb{E}_{i}[\tau_i^{+}] = \frac{1}{\pi(i)}
  \end{equation}
\end{proposition}

\begin{definition}
  \label{def:9}
  Let $\mathbf{X}$ be an irreducible Markov chain with transition
  matrix $\mathbf{P}$ and stationary distribution
  $\pi$. $\hat{\mathbf{P}} = (\hat{p}_{ij})$ is said to be the {\em
    time reversal} of $\mathbf{P}$ if, for all pairs $i,j \in \Omega$,
  one has
  \begin{equation}
    \label{eq:16}
    \pi(i) p_{ij} = \pi(j) \hat{p}_{ji}
  \end{equation}
  $\mathbf{P}$ is said to be {\em time-reversible} if
  $\hat{\mathbf{P}} = \mathbf{P}$.
\end{definition}
Now $\hat{\mathbf{P}}$ also defines a Markov chain
$\hat{\mathbf{X}}$. $\hat{\mathbf{X}}$ will be termed the
time-reversed Markov chain with respect to $\mathbf{X}$. $\pi$ is also
the stationary distribution of $\hat{\mathbf{P}}$ and that
\begin{equation}
  \label{eq:17}
  \mathbb{P}[X_n = j, \dots, X_0 = i] = \mathbb{P}[\hat{X}_0 = i,
  \dots, \hat{X}_n = j] 
\end{equation}
where the initial distribution of $X_0$ and $\widehat{X}_0$ are both
identical to the stationary distribution $\pi$.
\subsection{Random walks on graphs}
\label{sec:random-walks-graphs}
Let $G = (V,E,\omega)$ be a simple, undirected graph. We define the transition
matrix $\mathbf{P}_G = (p_{uv})$ of a Markov chain with state space $V$ as follows
\begin{equation}
  \label{eq:20}
  p_{uv} = \begin{cases}
    \tfrac{\omega(\{u,v\})}{\deg(u)} & \text{if $u \sim v$} \\
    0 & \text{otherwise}
  \end{cases}
\end{equation}
We now note some properties of the Markov chain $\mathbf{X}$ generated
by $\mathbf{P}_G$
\begin{itemize}
\item $\mathbf{X}$ is irreducible if and only if $G$ is connected.
\item If $\mathbf{X}$ is irreducible, $\pi(v) =
  \tfrac{\deg(v)}{\mathrm{Vol}(G)}$ for all $v \in V$.
\item $\mathbf{P}$ is time-reversible.
\end{itemize}
We can also define the transition matrix $\mathbf{P}_G$ when $G$ is
directed. $\mathbf{P}_G$ will have entries
\begin{equation}
  \label{eq:18}
  p_{uv} = \begin{cases}
    \tfrac{\omega(e)}{\deg(u)} & \text{if $e = (u,v) \in E$} \\
    0 & \text{otherwise}
  \end{cases}
\end{equation}
If $G$ is directed, then $\mathbf{X}$ is irreducible if and only if
$G$ is strongly connected. However, $\mathbf{P}$ is in general not
time reversible and there's no explicit expression for the
stationary distribution $\pi$ of $\mathbf{P}$.

Let $G = (V,E)$ be a graph, directed or undirected, and $\bm{P}$
be its transition matrix. A function $f \colon V \mapsto \mathbb{R}$
is {\em harmonic} at $v \in V$ if
\begin{equation}
  \label{eq:10}
  f(v) = \sum_{w \in V}{\bm{P}(v,w) f(w)}
\end{equation}
$f$ is harmonic on $V$ if it's harmonic for all $v \in V$. If $\bm{P}$
is irreducible, we have a simple characterization for harmonic
functions on $V$. Specifically,
\begin{lemma}
  \label{lem:1}
  Suppose that $\bm{P}$ is irreducible. A function $f \colon V \mapsto
  \mathbb{R}$ is harmonic on $V$ if and only if $f$ is constant on
  $V$. 
\end{lemma}
\begin{proof}
  It's easy to see that if $f$ is constant on $V$ then it's also
  harmonic on $V$. Thus, let's assume that $f$ is harmonic on $V$.
  Let $v_*$ be a node such that $f(v_*) \geq f(w)$ for all $w \in
  V$. Since $f$ is harmonic, from Eq.~\eqref{eq:10} we have that $f(w)
  = f(v_*)$ for all $w$ such that $\bm{P}(v_*,w) > 0$. We thus see
  that every vertex $w$ that's accessible from $v_*$ will satisfy
  $f(w) = f(v_*)$. Since $\bm{P}$ is irreducible, $f(v_*) = f(w)$ for
  all $w \in V$. $f$ is thus constant on $V$.
\end{proof}

%%% Local Variables: 
%%% mode: latex
%%% TeX-master: "dissertation"
%%% End: 
