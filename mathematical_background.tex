\section{Metrics on Graphs}
\subsection{Expected commute time}
\label{sec:resistance-distances}
Let $G = (V,E,\omega)$ be a graph, directed or undirected, with $\omega$
being the similarity measure. The {\em expected commute time}
$\delta(u,v)$ between $u \in V$ and $v \in V$ is defined as
\begin{equation}
  \label{eq:25}
  \delta(u,v) = \mathbb{E}_{u}[\tau_v] + \mathbb{E}_{v}[\tau_u]
\end{equation}
We assume in this dissertation that the Markov chain defined by the
transition matrix $\mathbf{P}$ of $G$ is regular, i.e., there exists a
$n_0 \in \mathbb{N}$ such that the entries of $\mathbf{P}^{n}$ is positive
for all $n \geq n_0$. 

\begin{proposition}
  \label{prop:4}
  Let $\mathbf{M}$ be the matrix of first passage time, i.e. $\mathbf{M}(u,v)
  = \mathbb{E}_{u}[\tau_v]$. $\mathbf{M}$ is then the unique solution of the
  following matrix equation
  \begin{equation}
    \label{eq:3}
   (\mathbf{I} - \mathbf{P})\mathbf{X} = \mathbf{J} - \bm{\Pi}^{-1}
  \end{equation}
  subjected to the condition 
  \begin{equation}
    \label{eq:32}
 \mathbf{M}_{\mathrm{dg}} = \mathbf{0}, \qquad \mathbf{M}(u,v) \geq 0   
  \end{equation}
  Thus, $\mathbf{M} = \mathbf{X} - \mathbf{J}\mathbf{X}_{\mathrm{dg}}$ where $\mathbf{X}$ satisfy Eq. \eqref{eq:3}.
\end{proposition}
\begin{proof}
  If $u = v$, then $\mathbb{E}_{u}[\tau_u] = 0$ and thus $\mathbf{M}(u,u)
  = 0$. Otherwise, if $u \not = v$, then $\mathbf{M}(u,v) = \mathbb{E}_{u}[\tau_v]$ can be
  expanded as
  \begin{equation}
    \label{eq:4}
    \mathbb{E}_{u}[\tau_v] = \sum_{w \in V}{\mathbf{P}(u,w)(1 +
      \mathbb{E}_{w}[\tau_v])} = 1 + \sum_{w \in V}{\mathbf{P}(u,w)
      \mathbb{E}_{w}[\tau_v]} = 1 + (\mathbf{PM})(u,v)
  \end{equation}
  Thus, $\mathbf{F} = \mathbf{J} + (\mathbf{P} - \mathbf{I})\mathbf{M}$ is a diagonal
  matrix. Futhermore, $\pi \mathbf{F} = \pi \mathbf{J} + \pi(\mathbf{P} -
  \mathbf{I})\mathbf{M} = \mathbf{1}$. Therefore, $\mathbf{F}(u,u) = 1/\pi(u)$ and
  thus $\mathbf{F} = \bm{\Pi}^{-1}$. $\mathbf{M}$ is thus a solution of the
  matrix equation as given by Eq.~\eqref{eq:3}.

  We now show that $\mathbf{M}$ is the unique solution of Eq.~\eqref{eq:3}
  subjected to the condition in Eq.~\eqref{eq:32}. Let $\mathbf{M}'$ be another solution of Eq.~\eqref{eq:3} subjected to the
  condition in Eq.~\eqref{eq:32}. Then $\mathbf{Y} = \mathbf{M} -
  \mathbf{M}'$ satisfy
  \begin{equation}
    \label{eq:19}
    (\mathbf{I} - \mathbf{P})\mathbf{Y} = \mathbf{0}
  \end{equation}
  By Lemma \ref{lem:1}, each column of $\mathbf{Y}$ is constant. Since
  $\mathbf{M}_{\mathrm{dg}} = \mathbf{M'}_{\mathrm{dg}} = \mathbf{0}$, each
  column of $\mathbf{Y}$ must be identically $0$. Thus $\mathbf{M} = \mathbf{M'}$,
  proving the uniqueness of $\mathbf{M}$. If $\mathbf{X}$ satisfy
  Eq. \eqref{eq:3}, then $\mathbf{X} - \mathrm{J}\mathbf{X}_{\mathrm{dg}}$
  satisfy the condition in Eq.~\eqref{eq:32}.
\end{proof}

\begin{proposition}
  \label{prop:5}
  Let $\mathbf{Q} = \mathbf{1}^{T}\mathbf{\pi}$ be the matrix with each row being
  the stationary distribution $\pi$. The matrix $\mathbf{M}$ of expected
  first passage time is given by
  \begin{equation}
    \label{eq:21}
    \mathbf{M} = \mathbf{J}\mathbf{Z}_{\mathrm{dg}} \bm{\Pi}^{-1} - \mathbf{Z}
    \bm{\Pi}^{-1}
  \end{equation}
  where $\mathbf{Z} = (\mathbf{I} - \mathbf{P} + \mathbf{Q})^{-1}$. 
\end{proposition}
\begin{proof}
  We first show that $\mathbf{X} = (\mathbf{I} - \mathbf{P} + \mathbf{Q})^{-1}(\mathbf{J}
  - \bm{\Pi}^{-1})$ satisfy Eq. \eqref{eq:3}. We have from Proposition
  \ref{prop:8} that $(\mathbf{I} - \mathbf{P})\mathbf{X} = (\mathbf{I} -
  \mathbf{P})\mathbf{Z}(\mathbf{J} - \bm{\Pi}^{-1}) = (\mathbf{I} - \mathbf{Q})(\mathbf{J} -
  \bm{\Pi})^{-1}$. Since
  $\mathbf{Q}\mathbf{J} = \mathbf{J} = \mathbf{Q}\bm{\Pi}^{-1}$, one has
  \begin{equation}
    \label{eq:27}
    (\mathbf{I} - \mathbf{P})\mathbf{X} = \mathbf{J} - \bm{\Pi}^{-1}
  \end{equation}
  and thus $\mathbf{X}$ satisfy Eq. \eqref{eq:3}. Also, from Proposition
  \ref{prop:8}, we have
  \begin{equation}
    \label{eq:31}
    \mathbf{X} = \mathbf{Z}(\mathbf{J} - \bm{\Pi}^{-1}) = \mathbf{J} -
    \mathbf{Z}\bm{\Pi}^{-1}
  \end{equation}
  and thus $\mathbf{X} - \mathbf{J}\mathbf{X}_{\mathrm{dg}} =
  \mathbf{J}\mathbf{Z}_{\mathrm{dg}} \bm{\Pi}^{-1} - \mathbf{Z}\bm{\Pi}^{-1}$. 
\end{proof}

\begin{proposition}
  \label{prop:10}
  The matrix $\Delta_{\delta}$ of expected commute time is given by 
  \begin{equation}
    \label{eq:33}
 \Delta_{\delta} = \mathbf{M} + \mathbf{M}^{T}
  = \mathbf{J}\mathbf{Z}_{\mathrm{dg}}\bm{\Pi}^{-1} - \mathbf{Z}\bm{\Pi}^{-1} -
  \bm{\Pi}^{-1}\mathbf{Z}^{T} - \bm{\Pi}^{-1}\mathbf{Z}_{\mathrm{dg}} \mathbf{J} =
  \tfrac{1}{2} \kappa(\mathbf{Z}\bm{\Pi}^{-1} +
  \bm{\Pi}^{-1}\mathbf{Z}^{T}).   
  \end{equation}
  The matrix $\mathbf{Z}\bm{\Pi}^{-1} + \bm{\Pi}^{-1}\mathbf{Z}^{T}$ is
  positive definite, and thus $\Delta_{\delta}$ is an EDM-2
  matrix.
\end{proposition}
\begin{proof}
  We use the following characterization of positive definite
  matrix. Let $\mathbf{A}$ be invertible. Then $\mathbf{A} + \mathbf{A}^{T}$ is positive definite if and only if
  $\mathbf{A}^{-1} + (\mathbf{A}^{-1})^{T}$ is positive definite
  \cite{horn94:_topic_in_matrix_analy}. Since $\mathbf{Z}\bm{\Pi}^{-1}$ is
  invertible,
  \begin{equation}
    \label{eq:34}
    \begin{split}
    \mathbf{Z}\bm{\Pi}^{-1} + \bm{\Pi}^{-1}\mathbf{Z}^{T} \succ 0
    & \leftrightarrow \bm{\Pi}(\mathbf{I} - \mathbf{P} + \mathbf{Q}) + (\mathbf{I} -
    \mathbf{P}^{T} + \mathbf{Q}^{T})\bm{\Pi} \succ 0 \\
    & \leftrightarrow \bm{\Pi}(\mathbf{I} - \mathbf{P}) + (\mathbf{I} -
    \mathbf{P}^{T})\bm{\Pi} + 2 \pi \pi^{T} \succ 0 \\
    & \leftrightarrow \bm{\Pi}(\mathbf{I} - \frac{\mathbf{P} + \mathbf{P}_{*}}{2})
    + 2 \pi \pi^{T} \succ 0
  \end{split}      
  \end{equation}
  where $\mathbf{P}_{*}$ is the time-reverseral of $\mathbf{P}$. Now,
  $\bm{\Pi}(\mathbf{I} - \frac{\mathbf{P} + \mathbf{P}_{*}}{2})$ is
  symmetric. Furthermore, $\frac{\mathbf{P} + \mathbf{P}_{*}}{2}$ is a
  stochastic matrix and thus $\bm{\Pi}(\mathbf{I} - \frac{\mathbf{P} +
    \mathbf{P}_{*}}{2})$ is diagonally dominant. By Ger\u{s}gorin's
  circle theorem \cite{gersgorin31:_uber_abgren_eigen_matrix},
  $\bm{\Pi}(\mathbf{I} - \frac{\mathbf{P} + \mathbf{P}_{*}}{2})$ is
  positive semidefinite. Thus, $\mathbf{Z}\bm{\Pi}^{-1} +
  \bm{\Pi}^{-1}\mathbf{Z}^{T} \succ 0$, and $\Delta_{\delta}$ is an
  EDM-2 matrix.
\end{proof}

There exists in the literatures a notion of distances known as
resistance distance
\cite{bapat99:_resis_distan_in_graph,klein93:_resis_distan}. Let $G =
(V,E,\omega)$ be an undirected graph with similarity measure
$\omega$. Let $\mathbf{L}$ be the combinatorial Laplacian of $G$. The
resistance distance $r(u,v)$ between $u, v \in V$ is defined as
\begin{equation}
  \label{eq:35}
  r(u,v) = \tfrac{1}{2}(\mathbf{L}^{\dagger}(u,u) - \mathbf{L}^{\dagger}(u,v) -
  \mathbf{L}^{\dagger}(v,u) + \mathbf{L}^{\dagger}(v,v))
\end{equation}
where $\mathbf{L}^{\dagger}$ is the {\em Moore-Penrose} pseudo-inverse of
$\mathbf{L}$. It's widely known that for undirected graphs, resistance
distance is proportional to expected commute
$\delta(u,v)$. Specifically,
\begin{equation}
  \label{eq:36}
  r(u,v) = \frac{2 \delta(u,v)}{\mathrm{Vol}(G)}
\end{equation}
Eq.~\eqref{eq:36} is an easy corollary of the following result.
\begin{proposition}
  \label{prop:11}
  Let $G = (V,E,\omega)$ be an undirected graph with $|V| = n$. The
  Moore-Penrose pseudo-inverse $\mathbf{L}^{\dagger}$ of $\mathbf{L}$ is given
  by
  \begin{equation}
    \label{eq:37}
    \mathbf{L}^{\dagger} = c \Bigl(\mathbf{I} - \frac{\mathbf{J}}{n}\Bigr) \mathbf{Z}
    \bm{\Pi}^{-1} \Bigl(\mathbf{I} - \frac{\mathbf{J}}{n}\Bigr)
  \end{equation}
  where $c = 1/\mathrm{Vol}(G)$ is a constant. 
\end{proposition}
\begin{proof}
  We will show that $\mathbf{L}^{\dagger}$ as defined by Eq.~\eqref{eq:37}
  satisfies the following conditions for a Moore-Penrose pseudo-inverse
  \begin{gather*}
    \mathbf{L}\mathbf{L}^{\dagger} = \mathbf{L}^{\dagger}\mathbf{L} \tag{(i)} \\
    \mathbf{L}\mathbf{L}^{\dagger}\mathbf{L} = \mathbf{L} \tag{(ii)} \\
    \mathbf{L}^{\dagger}\mathbf{L} \mathbf{L}^{\dagger} = \mathbf{L}^{\dagger}
    \tag{(iii)}
  \end{gather*}
  If $G = (V,E,\omega)$ is an undirected graph, then $\pi(u) =
  \deg(u)/\mathrm{Vol}(G)$ and thus $\mathbf{D} = \mathrm{Vol}(G)
  \bm{\Pi}$. Therefore $\mathbf{L} = \mathbf{D}(\mathbf{I} - \mathbf{P}) =
  \mathrm{Vol}(G) \bm{\Pi}(\mathbf{I} - \mathbf{P})$. We also have
  \begin{equation}
    \label{eq:38}
    \Bigl(\mathbf{I} - \frac{\mathbf{J}}{n}\Bigr)\mathbf{L} = \mathbf{L}
  \end{equation}
  and thus
  \begin{equation}
    \label{eq:39}
    \begin{split}
      \mathbf{L}^{\dagger}\mathbf{L} &= \Bigl(\mathbf{I} - \frac{\mathbf{J}}{n}\Bigr) \mathbf{Z}
       \bm{\Pi}^{-1} \bm{\Pi}(\mathbf{I} - \mathbf{P}) \\
       &= \Bigl(\mathbf{I} - \frac{\mathbf{J}}{n}\Bigr) (\mathbf{I} - \mathbf{P} + \mathbf{Q})^{-1}
       (\mathbf{I} - \mathbf{P}) \\
       &= \Bigl(\mathbf{I} - \frac{\mathbf{J}}{n}\Bigr)(\mathbf{I} - \mathbf{Q}) \\
       &= \Bigl(\mathbf{I} - \frac{\mathbf{J}}{n}\Bigr)
   \end{split}
  \end{equation}
  Similarly,
  \begin{equation}
    \label{eq:40}
    \begin{split}
      \mathbf{L}\mathbf{L}^{\dagger} &= \bm{\Pi}(\mathbf{I} - \mathbf{P}) \mathbf{Z}
      \bm{\Pi}^{-1} \Bigl(\mathbf{I} - \frac{\mathbf{J}}{n}\Bigr) \\
      &= \bm{\Pi}(\mathbf{I} - \mathbf{P}) (\mathbf{I} - \mathbf{P} + \mathbf{Q})^{-1}
      \bm{\Pi}^{-1} \Bigl(\mathbf{I} - \frac{\mathbf{J}}{n}\Bigr) \\
      &= (\mathbf{I} - \mathbf{Q}^{T})\Bigl(\mathbf{I} - \frac{\mathbf{J}}{n}\Bigr) \\
      &= \Bigl(\mathbf{I} - \frac{\mathbf{J}}{n}\Bigr)
   \end{split}
  \end{equation}
  Thus, condition (i) is satisfied. Furthermore, we also have, from
  Eq.~\eqref{eq:39} and Eq.~\eqref{eq:40}, that
  \begin{gather*}
   \mathbf{L}\mathbf{L}^{\dagger}\mathbf{L} = \Bigl(\mathbf{I} -
   \frac{\mathbf{J}}{n}\Bigr) \mathbf{L}
   = \mathbf{L} \\
    \mathbf{L}^{\dagger}\mathbf{L}\mathbf{L}^{\dagger} = c \Bigl(\mathbf{I} -
    \frac{\mathbf{J}}{n}\Bigr) \mathbf{Z}
     \bm{\Pi}^{-1} \Bigl(\mathbf{I} - \frac{\mathbf{J}}{n}\Bigr) \Bigl(\mathbf{I} -
     \frac{\mathbf{J}}{n}\Bigr) = c \Bigl(\mathbf{I} - \frac{\mathbf{J}}{n}\Bigr) \mathbf{Z}
     \bm{\Pi}^{-1} \Bigl(\mathbf{I} - \frac{\mathbf{J}}{n}\Bigr) = \mathbf{L}^{\dagger}
  \end{gather*}
  Thus, $\mathbf{L}^{\dagger}$ as defined by Eq.~\eqref{eq:37} is the
  Moore-Penrose inverse of $\mathbf{L}$. 
\end{proof}
If $G$ is an undirected graph, then $\mathbf{Z}\bm{\Pi}^{-1} =
\bm{\Pi}^{-1}\mathbf{Z}^{T}$ and so 
\begin{equation}
  \label{eq:42}
 \Delta_{\delta} =
\kappa(\mathbf{Z}\bm{\Pi}^{-1}) = \mathrm{Vol}(G)
\kappa(\mathbf{L}^{\dagger}) 
\end{equation}
Eq.~\eqref{eq:36} thus follows from Proposition \ref{prop:11}, as
claimed. 

\subsection{Diffusion distances}
\label{sec:diffusion-distances}
Let $G = (V,E,\omega)$ be a graph, directed or undirected, with
$\omega$ being a similarity measure between vertices of $V$. Denote by
$\mathbf{P}$ the probability transition matrix of $G$. The diffusion
distances at time $t$, $\rho_{t}(u,v)$, between $u,v \in V$ is defined as
\cite{coifman06:_diffus_maps}
\begin{equation}
  \label{eq:43}
  \rho^{2}_{t}(u,v) = \sum_{w \in V}{\Bigl(\mathbf{P}^{t}(u,w) -
      \mathbf{P}^{t}(v,w)\Bigr)^2 \frac{1}{\pi(w)}}
\end{equation}
\begin{proposition}
  \label{prop:12}
  Diffusion distances as defined by Eq.~\eqref{eq:43} can also be
  written as
  \begin{equation}
    \label{eq:44}
    \begin{split}
      \rho_{t}^{2}(u,v) &= \frac{(\mathbf{P}^{t}\mathbf{P}_{*}^{t})(u,u) -
        (\mathbf{P}^{t}\mathbf{P}_{*}^{t})(v,u)}{\pi(u)} +
      \frac{(\mathbf{P}^{t}\mathbf{P}_{*}^{t})(v,v) -
        (\mathbf{P}^{t}\mathbf{P}_{*}^{t})(u,v)}{\pi(v)}  \\
      &= (\mathbf{P}^{t}\mathbf{P}_{*}^{t}\bm{\Pi}^{-1})(u,u) -
      (\mathbf{P}^{t}\mathbf{P}_{*}^{t}\bm{\Pi}^{-1})(v,u) \\
      &+ (\mathbf{P}^{t}\mathbf{P}_{*}^{t}\bm{\Pi}^{-1})(v,v) -
      (\mathbf{P}^{t}\mathbf{P}_{*}^{t}\bm{\Pi}^{-1})(u,v)
    \end{split}
  \end{equation}
  where $\mathbf{P}_{*}$ is the time-reversal of $\mathbf{P}$. 
\end{proposition}
\begin{proof}
  We assumed throughout this dissertation that $\mathbf{P}$ is
  irreducible. Thus, there exists a unique stationary distribution
  $\pi$ for $\mathbf{P}$. Furthermore, the time-reversal $\mathbf{P}_{*}$ of
  $\mathbf{P}$ exists and is defined by
  \begin{equation}
    \label{eq:45}
    \pi(u) \mathbf{P}(u,v) = \pi(v) \mathbf{P}_{*}(v,u) 
  \end{equation}
  i.e, $\mathbf{P}_{*} = \bm{\Pi}^{-1} \mathbf{P}^{T} \bm{\Pi}$. Thus, by
  expanding the square of $(\mathbf{P}^{t}(u,w) - \mathbf{P}^{t}(v,w))^{2}$ in
  Eq.~\eqref{eq:43} and using Eq.~\eqref{eq:45}, one has
  \begin{equation}
    \label{eq:46}
    \begin{split}
      \rho_{t}^{2}(u,v) &= \sum_{w \in V}{\Bigl(\mathbf{P}^{t}(u,w) -
        \mathbf{P}^{t}(v,w)\Bigr)^2 \frac{1}{\pi(w)}} \\
      &= \sum_{w \in V}{\frac{\mathbf{P}^{t}(u,w)\mathbf{P}^{t}(u,w) -
          \mathbf{P}^{t}(u,w)\mathbf{P}^{t}(v,w)}{\pi(w)}} \\
      &+\sum_{w \in V}{\frac{\mathbf{P}^{t}(v,w)\mathbf{P}^{t}(v,w) -
          \mathbf{P}^{t}(v,w)\mathbf{P}^{t}(u,w)}{\pi(w)}} \\
      &= \sum_{w \in
        V}{\frac{\mathbf{P}^{t}(u,w)\mathbf{P}_{*}^{t}(w,u) -
          \mathbf{P}^{t}(v,w)\mathbf{P}_{*}^{t}(w,u)}{\pi(u)}} \\ &+
      \sum_{w \in V}{\frac{\mathbf{P}^{t}(v,w)\mathbf{P}_{*}^{t}(w,v)
          -
          \mathbf{P}^{t}(u,w)\mathbf{P}_{*}^{t}(w,v)}{\pi(v)}} \\
      &= \frac{(\mathbf{P}^{t}\mathbf{P}_{*}^{t})(u,u) -
        (\mathbf{P}^{t}\mathbf{P}_{*}^{t})(v,u)}{\pi(u)} +
      \frac{(\mathbf{P}^{t}\mathbf{P}_{*}^{t})(v,v) -
        (\mathbf{P}^{t}\mathbf{P}_{*}^{t})(u,v)}{\pi(v)} 
    \end{split} 
  \end{equation}
  which is exactly Eq.~\eqref{eq:44}. 
\end{proof} 
From Eq.~\eqref{eq:44}, the matrix $\Delta_{\rho_{t}^{2}}$
of squared diffusion distances can be written
as 
\begin{equation} 
  \label{eq:47} 
  \Delta_{\rho_{t}^{2}} =
  \kappa(\mathbf{P}^{t}\mathbf{P}_{*}^{t}\mathbf{\Pi^{-1}}) 
\end{equation}
Since $\mathbf{P}_{*}^{t} = \bm{\Pi}^{-1}\mathbf{P}^{T}\bm{\Pi}$, one
has 
\begin{equation} 
  \label{eq:48}
  \mathbf{P}^{t}\mathbf{P}_{*}^{t}\mathbf{\Pi^{-1}} =
  \mathbf{P}^{t}\bm{\Pi}^{-1}(\mathbf{P}^{t})^{T} \succeq
  0 \end{equation} Thus $\Delta_{\rho_{t}^{2}}$ is an EDM-2 matrix,
and so $\Delta_{\rho_{t}}$ is an EDM-1 matrix for all $t$. We state
the above observation as the following
proposition 
\begin{proposition} 
\label{prop:14} 
Let $G = (V,E,\omega)$ be a graph and $\mathbf{P}$ be its transition
matrix. The matrix $\Delta_{\rho_{t}}$ of diffusion distances is then
an EDM-1 matrix for any $t$.
\end{proposition}
%
We now note the connection between expected commute time and diffusion
distances for when $G$ is an undirected graph. If $G$ is an undirected
graph, then $\mathbf{P} = \mathbf{P}_*$ and thus
$\Delta_{\rho_{t}^{2}} = \kappa(\mathbf{P}^{t}\mathbf{P}_{*}^{t}
\bm{\Pi^{-1}}) = \kappa(\mathbf{P}^{2t}\bm{\Pi^{-1}})$. Furthermore,
\begin{equation}
  \label{eq:49}
\kappa(\mathbf{P}^{2t}\bm{\Pi^{-1}}) =
\kappa(\mathbf{P}^{2t}\bm{\Pi^{-1}} - \mathbf{J}) =
\kappa((\mathbf{P}^{2t} - \mathbf{Q}) \bm{\Pi^{-1}})
\end{equation}
Let $\mathbf{T}_{m} = \Bigl(\mathbf{I} + \sum_{k =
  1}^{m}{(\mathbf{P}^{k} - \mathbf{Q})}\Bigr)\bm{\Pi}^{-1}$ for $m
\geq 0$. Then $\| \mathbf{T}_m - \mathbf{Z}\bm{\Pi}^{-1} \| \rightarrow 0$ as
$m \rightarrow \infty$ for any matrix norm $\| \cdot \|$. Moreover,
for any $n$, $\kappa$ is a bounded linear operator from the vector
space of $n \times n$ square matrices to the space of $n \times n$
square matrices. Thus, we have
\begin{equation}
  \label{eq:50}
  \lim_{m \rightarrow \infty}\kappa(\mathbf{T}_m) = \lim_{m \rightarrow \infty}
  \sum_{k=0}^{m}{\kappa((\mathbf{P}^{m} - \mathbf{Q})\bm{\Pi}^{-1})} =
    \kappa(\mathbf{Z}\bm{\Pi}^{-1})
\end{equation}
Thus, if we let $\mathbf{P}_{2} = \mathbf{P}^{2}$ be the transition matrix
of the two-step random walk on $G$, then $\mathbf{P}^{2t} =
\mathbf{P}_{2}^{t}$ and also that $\mathbf{Q}_{2} = \mathbf{Q}$ and thus
\begin{equation}
  \label{eq:51}
  \sum_{t = 0}^{\infty} \Delta_{\rho_{t}^{2}} = \sum_{t = 0}^{\infty}
  \kappa((\mathbf{P}_{2}^{t} - \mathbf{Q})\bm{\Pi}^{-1}) =
  \kappa(\mathbf{Z}_{2} \bm{\Pi}^{-1})
\end{equation}
where $\mathbf{Z}_{2}$ is the fundamental matrix for
$\mathbf{P}_{2}$. Thus, the expected commute time with respect to $\mathbf{P}_2$ is the
sum of the diffusion distances with respect to $\mathbf{P}$ at all
time-scale $t$. 
\subsection{Forest metrics}
Let $G = (V,E,\omega)$ be an undirected graph with $\omega$ being the
similarity measure between vertices of $G$. Denote by $\mathbf{L}$ the
combinatorial Laplacian of $G$. Let $\alpha \geq 0$ be a fixed
constant and defined the matrix $\mathbf{Q}_{\alpha}$ by
\begin{equation}
  \label{eq:30}
  \mathbf{Q}_{\alpha} = (\mathbf{I} + \alpha \mathbf{L})^{-1}
\end{equation}
Chebotarev and Shamis
\cite{chebotarev02:_fores_metric_for_graph_vertic} defined a notion of
distance $\eta(u,v)$ between vertices of $G$ by
\begin{equation}
  \label{eq:41}
  \eta_\alpha(u,v) = \mathbf{Q}_\alpha(u,u) - \mathbf{Q}_\alpha(u,v) -
  \mathbf{Q}_\alpha(v,u) + \mathbf{Q}_\alpha(v,v)
\end{equation}
The $\eta_{\alpha}$ is called a family of forest metrics on $G$ by
Chebotarev and Shamis
\cite{chebotarev02:_fores_metric_for_graph_vertic}. Some properties of
the family of forest metrics are given below, c.f. \cite{chebotarev02:_fores_metric_for_graph_vertic}.
\begin{theorem}
  \label{thm:4}
  For any $\alpha \geq 0$, the matrix $\mathbf{Q}_{\alpha}$ is
  positive definite. Furthermore, $\mathbf{Q}_{\alpha}$ is a doubly
  stochastic matrix. The matrix $\Delta_{\eta_{\alpha}}$ of
  forest metrics between vertices of $G$ is thus an EDM-2 matrix for
  all $\alpha \geq 0$.  
\end{theorem}

%An interpretation for the entries in $\mathbf{Q}_{\alpha}$ is also
%available in \cite{chebotarev02:_fores_metric_for_graph_vertic}. Specifically, let 
We now extends the notion of forest metrics as defined by Chebotarev
and Shamis to directed graphs. Let $G = (V,E,\omega)$ be a graph with similarity measure
$\omega$. Let $\mathbf{P}$ be the transition matrix of $G$. We assumed
that $\mathbf{P}$ is irreducible. Consider the matrix
$\mathbf{X}_{\beta} = \mathbf{I} +
\alpha \bm{\Pi}(\mathbf{I} - \mathbf{P})$ with $\beta \geq 0$. Note  
that for an undirected graph $G$, $\mathbf{L} = \mathrm{Vol}(G)
\bm{\Pi}^{-1}(\mathbf{I} - \mathbf{P})$. Thus $\mathbf{X}_{\beta}$ subsumes
the role of $(\mathbf{I} + \alpha \mathbf{L})$ for general graphs.  
%
%
\noindent Now, $\mathbf{X}_{\beta}$ is strictly diagonally dominant. Furthermore, the
off-diagonal entries of $\mathbf{X}_{\beta}$ is non-positive. Thus,
$\mathbf{X}_{\beta}$ is a $M$-matrix as defined by Definition
\ref{def:8} for all $\beta \geq 0$. By
Theorem \ref{thm:2}, $\mathbf{X}_{\beta}^{-1}$ exists and is a non-negative
matrix. Now, the vector $\bm{1}$ of all ones is an eigenvector of
$\mathbf{X}_{\beta}$ with eigenvalue $1$ since $(\mathbf{I} -
\mathbf{P})\bm{1} = 0$. Thus $\bm{1}$ is also an eigenvector of
$\mathbf{X}_{\beta}^{-1}$ with eigenvalue $1$. Since
$\mathbf{X}_{\beta}^{-1}$ is a
non-negative matrix, we thus have that $\mathbf{X}_{\beta}^{-1}$ is a
stochastic matrix. We summarized the above observations in the
following proposition.  
\begin{proposition}
  \label{prop:9}
  Let $\mathbf{X}_{\beta} = \mathbf{I} + \beta \bm{\Pi}(\mathbf{I} -
  \mathbf{P})$ for some fixed $\beta \geq 0$. Then
  $\mathbf{X}_{\beta}^{-1}$ exists and is a stochastic
  matrix. 
\end{proposition}
From the fact that $\mathbf{X}_{\beta}$ is strictly diagonally dominant, we
see that $\mathbf{X}_{\beta} + \mathbf{X}_{\beta}^{T}$ is symmetric and strictly
diagonally dominant and hence is a positive definite matrix. Thus
$\mathbf{X}_{\beta}^{-1} + (\mathbf{X}_{\beta}^{T})^{-1}$ is also positive definite. 
We thus have the following result 
\begin{proposition}
  \label{prop:13}
  The matrix $\Delta_{\eta_{\beta}} =
  \kappa\Bigl(\tfrac{1}{2}(\mathbf{X}_{\beta}^{-1} +
  (\mathbf{X}_{\beta}^{-1})^{T})\Bigr)$ is an EDM-2 matrix. 
\end{proposition}
Now, if $G$ is an undirected graph, $\mathbf{X}_{\beta} =
\mathbf{X}_{\beta}^{T}$, and $\mathbf{X}_{\beta}^{-1} =
\mathbf{X}_{\beta}^{-1} = \mathbf{Q}_{\alpha}$ for $\alpha =
\beta/\mathrm{Vol}(G)$. Proposition \ref{prop:9} and Proposition
\ref{prop:13} combined to give a generalization of Theorem \ref{thm:4}
to general graphs. 

\section{Mathematical Preliminaries}

\subsection{Graph Laplacians}
\label{sec:graph-laplacians}
We now introduce the concept of the Laplacian matrix of a graph. Our
exposition will be very superficial. For a more comprehensive account
of graph Laplacians, please consult \cite{chung05:_laplac_cheeg,cvetkovic80:_spect_graph_theor_applic}.

Let $G = (V,E,\omega)$ be a simple, undirected graph with vertices set
$V$, edges set $E$ and similarity measure $\omega \colon E \mapsto
\mathbb{R}^{\geq 0}$. If $u$ and $v$ are vertices of $G$, we write $u \sim v$
whenever $\{u,v\} \in E$. The degree of a vertex $v$ is defined as
$\deg(v) = \sum_{u \sim v}{\omega(\{u,v\})}$ and the volume of $G$ is
$\mathrm{Vol}(G) = \sum_{v \in V}{\deg(v)}$.  We denote by $N$ the
number of vertices of $G$. We define $D = (d_{ij})$ as the $N \times
N$ diagonal matrix with diagonal entries $d_{vv} = \deg(v)$.

\begin{definition}
  \label{def:1}
  Let $G = (V,E,\omega)$ be a simple, undirected graph with similarity
  measure $\omega$. The {\em combinatorial} Laplacian of $G$ is the
  matrix $L = L(G)$ with entries
  \begin{equation}
    \label{eq:1}
    L_{uv} = \begin{cases}
      - \omega(\{u,v\}) & \text{if $u \not = v$ and $u \sim v$} \\
      \deg(u) & \text{if $u = v$} \\
      0 & \text{otherwise}
    \end{cases}
  \end{equation}
  The {\em normalized} Laplacian of $G$ is the matrix $\mathcal{L} =
  \mathcal{L}(G)$ with entries
  \begin{equation}
    \label{eq:2}
    \mathcal{L}_{uv} = \begin{cases}
      - \tfrac{\omega(\{u,v\})}{\sqrt{\deg(u)}\sqrt{\deg(v)}} & \text{if $u \not = v$ and $u \sim v$} \\
      1 & \text{if $u = v$} \\
      0 & \text{otherwise}
    \end{cases}
  \end{equation}
\end{definition}
The following proposition lists some simple properties of the
combinatorial and normalized Laplacians. 
\begin{proposition}
  \label{prop:1}
  Let $G = (V,E,\omega)$ be a simple, undirected graph and $L$ and
  $\mathcal{L}$ be its combinatorial and normalized Laplacians,
  respectively. We have
  \begin{itemize}
  \item $L$ and $\mathcal{L}$ are symmetric, positive
    semi-definite matrices.
  \item $\mathcal{L} = D^{-1/2} L D^{-1/2}$
  \item The number of connected components of $G$ is equal to the
    number of zero eigenvalues of either $L$ or $\mathcal{L}$.
  \item The eigenvalues of $\mathcal{L}$ is at most $2$. 
  \end{itemize}
\end{proposition}
\subsection{Finite Markov Chain}
\begin{definition}
  \label{def:6}
  Let $\Omega$ be a finite or countably infinite set and
  $\mathbb{Z}^{*}$ be the set of non-negative integers. A sequence
  $\mathbf{X} = (X_n)_{n \in \mathbb{Z}^{*}}$ of random variables with values in
  $\Omega$ is a {\em Markov chain} if
  \begin{equation}
    \label{eq:8}
    \mathbb{P}[X_{n+1} = j \, | \, X_n = i, X_{n-1} = i_{n-1},
    \dots, X_0 = i_0] = \mathbb{P}[X_{n+1} = j \, | \, X_n = i] =
    p_{ij}
  \end{equation}
  for all $n \geq 0$ and all states $i_0, i_1, \dots, i_{n-1}, i,
  j$. The matrix $\mathbf{P}$, possibly infinite, with entries
  $\mathbf{P}(i,j) = p_{ij}$ is then termed the transition matrix of
  $(X_n)_{n \in \mathbb{Z}^*}$.
\end{definition}
Let $\mathbf{X} = (X_n)_{n \in \mathbb{Z}^*}$ be a Markov
chain. Denote by $p_{ij}^{(n)}$ the probability of going from state
$i$ to state $j$ in $n$ steps, i.e.,
\begin{equation}
  \label{eq:11}
  p_{ij}^{(n)} = \mathbb{P}[ X_{n + m } = j \, | \, X_m = i]
\end{equation}
for all $i, j \in \Omega$, and $m,n \in \mathbb{Z}^{*}$. Then
$p_{ij}^{(n)}$ satisfy the {\em Chapman-Kolmogorov equation}
\begin{equation}
  \label{eq:12}
  p_{ij}^{(m+n)} = \sum_{k \in \Omega}{p_{ik}^{(m)}p_{kj}^{(n)}}
\end{equation}
for all $m,n \in \mathbb{Z}^{*}$. Thus if $\mathbf{P}^{(n)}$ is the
matrix with entries $p_{ij}^{(n)}$, then $\mathbf{P}^{(m+n)} =
\mathbf{P}^{(m)}\mathbf{P}^{(n)}$. Since $\mathbf{P}^{(1)} =
\mathbf{P}$, we have
\begin{equation}
  \label{eq:13}
  \mathbf{P}^{(n)} = \mathbf{P}^{n}
\end{equation}
The behaviour of a Markov chain $\mathbf{X} = (X_n)_{n \in
  \mathbb{Z}^{*}}$ is thus completely specified by its transition
matrix $\mathbf{P}$. We can therefore view a Markov chain as being a
sequence of random variables generated by a transition matrix
$\mathbf{P}$. This view will be most helpful in the context of this
dissertation. However, since the transition matrix $\mathbf{P}$ only describes
the conditional probabilities, in order for us to compute the marginal
probabilities $\mathbb{P}[X_n = j]$, we need to specify an initial
distribution for $X_0$.

\begin{definition}
  \label{def:5}
  Let $\mathbf{X}$ be a Markov chain with state space
  $\Omega$. The initial distribution $\mu$ of $\mathbf{X}$ is a probability
  distribution on $\Omega$ such that 
  \begin{equation}
    \label{eq:14}
    \mu(i) = \mathbb{P}[X_0 = i]
  \end{equation}
  for all $i \in \Omega$. 
\end{definition}

\begin{definition}
  \label{def:7}
  Let $\mathbf{X}$ be a Markov chain with state space $\Omega$. Let
  $i$ and $j$ be elements of $\Omega$. $j$ is
  {\em accessible} from $i$, denoted as $i \rightarrow j$, if there
  exists a $n \in \mathbb{Z}^{*}$ such that $p_{ij}^{(n)} > 0$. If $i
  \rightarrow j$ and $j \rightarrow i$, then we say that $i$ and $j$
  {\em communicate}, and we write $i \leftrightarrow j$. A Markov chain is
  {\em irreducible} if $i \leftrightarrow j$ for any $i,j \in \Omega$.
\end{definition}
\begin{definition}
  \label{def:2}
  The stationary distribution $\pi$ of
  $\mathbf{X}$, if it exists, is a probability distribution on
  $\Omega$ such that
  \begin{equation}
    \label{eq:15}
    \pi(j) = \sum_{i \in \Omega}{\pi(i) p_{ij}}
  \end{equation}
  for any $j \in \Omega$. 
\end{definition}

\begin{proposition}
  \label{prop:3}
  If $\mathbf{X}$ is an irreducible Markov chain with state space
  $\Omega$, then there exists a unique stationary distribution $\pi$
  of $\mathbf{X}$, and that $\pi(i) > 0$ for all $i \in \Omega$. 
\end{proposition}

\begin{definition}
  \label{def:3}
  Let $\mathbf{X}$ be a Markov chain
  with transition matrix $\mathbf{P}$. Define 
  \begin{equation}
    \label{eq:5}
    \tau_i = \min\{ t \geq 0 \colon X_t = i \}, \qquad \tau_i^{+} \min
    \{ t \geq 1 \colon X_t = i \}
  \end{equation}
  The expected first passage time from $i$ to $j$, denoted by
  $\mathbb{E}_{i}[\tau_j]$, is defined as
  \begin{equation}
    \label{eq:6}
    \mathbb{E}_{i}[\tau_j] = \sum_{t = 0}^{\infty}{t \, \mathbb{P}(\tau_j =
      t \,|\, X_0 = i)}
  \end{equation}
  The expected first return time from $i$ to $i$, denoted by
  $\mathbb{E}_{i}[\tau_i^{+}]$, is defined as
  \begin{equation}
    \label{eq:7}
    \mathbb{E}_{i}[\tau_i^{+}] = \sum_{t = 1}^{\infty}{t \,
      \mathbb{P}(\tau_v^{+} = t \,|\, X_0 = i)}
  \end{equation}
  $\tau_i$ and $\tau_{i}^{+}$ as declared above are examples of {\em
    stopping times}. 
\end{definition}

\begin{proposition}
  \label{prop:2}
  Let $\mathbf{X}$ be an irreducible
  Markov chain with transition matrix $\mathbf{P}$ and stationary
  distribution $\pi$. We then have that
  \begin{equation}
    \label{eq:9}
    \mathbb{E}_{i}[\tau_i^{+}] = \frac{1}{\pi(i)}
  \end{equation}
\end{proposition}

\begin{definition}
  \label{def:9}
  Let $\mathbf{X}$ be an irreducible Markov chain with transition
  matrix $\mathbf{P}$ and stationary distribution
  $\pi$. $\hat{\mathbf{P}} = (\hat{p}_{ij})$ is said to be the {\em
    time reversal} of $\mathbf{P}$ if, for all pairs $i,j \in \Omega$,
  one has
  \begin{equation}
    \label{eq:16}
    \pi(i) p_{ij} = \pi(j) \hat{p}_{ji}
  \end{equation}
  $\mathbf{P}$ is said to be {\em time-reversible} if
  $\hat{\mathbf{P}} = \mathbf{P}$.
\end{definition}
Now $\hat{\mathbf{P}}$ also defines a Markov chain
$\hat{\mathbf{X}}$. $\hat{\mathbf{X}}$ will be termed the
time-reversed Markov chain with respect to $\mathbf{X}$. $\pi$ is also
the stationary distribution of $\hat{\mathbf{P}}$ and that
\begin{equation}
  \label{eq:17}
  \mathbb{P}[X_n = j, \dots, X_0 = i] = \mathbb{P}[\hat{X}_0 = i,
  \dots, \hat{X}_n = j] 
\end{equation}
where the initial distribution of $X_0$ and $\widehat{X}_0$ are both
identical to the stationary distribution $\pi$.
\subsection{Random walks on graphs}
\label{sec:random-walks-graphs}
Let $G = (V,E,\omega)$ be a simple, undirected graph. We define the transition
matrix $\mathbf{P}_G = (p_{uv})$ of a Markov chain with state space $V$ as follows
\begin{equation}
  \label{eq:20}
  p_{uv} = \begin{cases}
    \tfrac{\omega(\{u,v\})}{\deg(u)} & \text{if $u \sim v$} \\
    0 & \text{otherwise}
  \end{cases}
\end{equation}
We now note some properties of the Markov chain $\mathbf{X}$ generated
by $\mathbf{P}_G$
\begin{itemize}
\item $\mathbf{X}$ is irreducible if and only if $G$ is connected.
\item If $\mathbf{X}$ is irreducible, $\pi(v) =
  \tfrac{\deg(v)}{\mathrm{Vol}(G)}$ for all $v \in V$.
\item $\mathbf{P}$ is time-reversible.
\end{itemize}
We can also define the transition matrix $\mathbf{P}_G$ when $G$ is
directed. $\mathbf{P}_G$ will have entries
\begin{equation}
  \label{eq:18}
  p_{uv} = \begin{cases}
    \tfrac{\omega(e)}{\deg(u)} & \text{if $e = (u,v) \in E$} \\
    0 & \text{otherwise}
  \end{cases}
\end{equation}
If $G$ is directed, then $\mathbf{X}$ is irreducible if and only if
$G$ is strongly connected. However, $\mathbf{P}$ is in general not
time reversible and there's no explicit expression for the
stationary distribution $\pi$ of $\mathbf{P}$.

Let $G = (V,E)$ be a graph, directed or undirected, and $\mathbf{P}$
be its transition matrix. A function $f \colon V \mapsto \mathbb{R}$
is {\em harmonic} at $v \in V$ if
\begin{equation}
  \label{eq:10}
  f(v) = \sum_{w \in V}{\mathbf{P}(v,w) f(w)}
\end{equation}
$f$ is harmonic on $V$ if it's harmonic for all $v \in V$. If $\mathbf{P}$
is irreducible, we have a simple characterization for harmonic
functions on $V$. Specifically,
\begin{lemma}
  \label{lem:1}
  Suppose that $\mathbf{P}$ is irreducible. A function $f \colon V \mapsto
  \mathbb{R}$ is harmonic on $V$ if and only if $f$ is constant on
  $V$. 
\end{lemma}
\begin{proof}
  It's easy to see that if $f$ is constant on $V$ then it's also
  harmonic on $V$. Thus, let's assume that $f$ is harmonic on $V$.
  Let $v_*$ be a node such that $f(v_*) \geq f(w)$ for all $w \in
  V$. Since $f$ is harmonic, from Eq.~\eqref{eq:10} we have that $f(w)
  = f(v_*)$ for all $w$ such that $\mathbf{P}(v_*,w) > 0$. We thus see
  that every vertex $w$ that's accessible from $v_*$ will satisfy
  $f(w) = f(v_*)$. Since $\mathbf{P}$ is irreducible, $f(v_*) = f(w)$ for
  all $w \in V$. $f$ is thus constant on $V$.
\end{proof}

\begin{proposition}
  \label{prop:6}
  Let $G = (V,E)$ be a graph and $\mathbf{P}$ be its transition
  matrix. Suppose that the Markov chain defined by $\mathbf{P}$ is
  regular. Then there exists a unique stationary distribution $\pi$ of
  $\mathbf{P}$. Furthermore, if $\mathbf{Q} = \mathbf{1} \mathbf{\pi}^{T}$ is the
  matrix with each row being the stationary distribution, then
  \begin{equation}
    \label{eq:22}
    \lim_{k \rightarrow \infty}(\mathbf{P} - \mathbf{Q})^{k} = 0 
  \end{equation}
\end{proposition}

\begin{proposition}
  \label{prop:7}
  Let $G = (V,E)$ be a graph and $\mathbf{P}$ be its transition
  matrix. Suppose that $\mathbf{P}$ is regular. Then the matrix $\mathbf{Z} =
  (\mathbf{I} - \mathbf{P} + \mathbf{Q})^{-1}$ exists and is given by
  \begin{equation}
    \label{eq:28}
    \mathbf{Z} = \sum_{k=0}^{\infty}(\mathbf{P} - \mathbf{Q})^{k} = \mathbf{I} +
    \sum_{k=1}^{\infty}(\mathbf{P}^{k} - \mathbf{Q})
  \end{equation}
  
\end{proposition}
\begin{proof}
  Since $\mathbf{P}$ is regular, by Proposition \ref{prop:6}, $\lim_{k
    \rightarrow \infty}(\mathbf{P} - \mathbf{Q})^{k} = 0$. Thus, $\mathbf{Z}$ has
  an expansion in term of a Neumann series
  \begin{equation}
    \label{eq:29}
    \mathbf{Z} = \sum_{k=0}^{\infty}(\mathbf{P} - \mathbf{Q})^{k}
  \end{equation}
  Since $\mathbf{P}\mathbf{Q} = \mathbf{P}1^{T}\mathbf{\pi} = 1^{T}\mathbf{\pi} = 
  1^{T}\mathbf{\pi}\mathbf{P} = \mathbf{Q}\mathbf{P} = \mathbf{Q}$,  
  one has $(\mathbf{P} - \mathbf{Q})^{k} = \mathbf{P}^{k} - \mathbf{Q}$ for $k \geq
  1$. Eq.~\eqref{eq:28} thus follows. 
\end{proof}
The matrix $\mathbf{Z}$ is termed the {\emph fundamental matrix}
\cite{kemeny83:_finit_markov_chain}. Some properties of
$\mathbf{Z}$ are given in the following proposition.
\begin{proposition}
  \label{prop:8}
  Let $\mathbf{P}$ be the transition matrix of a regular Markov chain and
  $\mathbf{Z}$ be its fundamental matrix. We have
  \begin{enumerate}[(i)]
  \item $\mathbf{P}\mathbf{Z} = \mathbf{Z} - \mathbf{I} + \mathbf{Q}$. 
  \item $(\mathbf{I} - \mathbf{P})\mathbf{Z} = \mathbf{I} - \mathbf{Q}$.
  \item $\mathbf{Z} \mathbf{J} = \mathbf{J}$. 
  \end{enumerate}
\end{proposition}
\begin{proof}
  $\mathbf{P}\mathbf{Z} = \mathbf{P} - \sum_{k=1}^{\infty}(\mathbf{P}^{k+1} - \mathbf{Q})
  = \mathbf{Z} - \mathbf{I} + \mathbf{Q}$. (i) and (ii) thus follows. For (iii),
  note that $\mathbf{P}^{k}\mathbf{J} = \mathbf{Q}\mathbf{J} = \mathbf{J}$. 
\end{proof}
\subsection{Matrix Analysis}
We listed here some results in matrix analysis that are useful within
the scope of this dissertation.
%
Let $\mathbf{A} = (a_{ij})$ be an $n \times n$ matrix with real
entries $a_{ij}$. Denote by $R_i$ the sum $\sum_{j \not = i}{|a_{ij}|}$ of
off-diagonal elements in row $i$. 
%
\begin{theorem}[Ger\u{s}gorin's circle theorem \cite{gersgorin31:_uber_abgren_eigen_matrix}]
  \label{thm:1}
  Let $\mathbf{A}$ be an $n \times n$ matrix with off-diagonal row
  sums $R_i$. Then the eigenvalue of $\mathbf{A}$ lies in the set
  \begin{equation}
    \label{eq:23}
    \bigcup \{z \in \mathbb{C} \colon |z - a_{ii}| \leq R_i \}
  \end{equation}
\end{theorem}
\begin{definition}
  \label{def:4}
  The matrix $\mathbf{A}$ is said to be diagonally dominant if
  $|a_{ii}| \geq R_i$ for all $i$ and strictly diagonally dominant if
  $|a_{ii}| > R_i$ for all $i$.
\end{definition}
If $\mathbf{A}$ is diagonally dominant, then by Ger\u{s}gorin's
circle theorem, the eigenvalues of $\mathbf{A}$ has nonnegative real
parts. If $\mathbf{A}$ is strictly diagonally dominant, then the
eigenvalues of $\mathbf{A}$ has positive real parts. 
%
\begin{definition}
  \label{def:8}
  Let $Z_n \subset M_{n}(\mathbb{R})$ be the set of matrices with
  non-positive off-diagonal entries, i.e.,
  \begin{equation}
    \label{eq:24}
    Z_n = \{ \mathbf{A} = (a_{ij}) \in M_{n}(\mathbb{R}) \colon a_{ij}
    \leq 0 \,\, \text{if $i \not = j$} \}
  \end{equation}
 A matrix $\mathbf{A} \in Z_n$ is called an $M$-matrix if $A$ is
 positive stable, i.e., if the eigenvalues of $\mathbf{A}$ has
 positive real parts.
\end{definition}
A relationship between $M$-matrices and non-negative matrices is given
by the following result \citet[\S
2.5]{horn94:_topic_in_matrix_analy}.
\begin{theorem}
  \label{thm:2}
  $\mathbf{A}$ is a $M$-matrix if and only if $\mathbf{A}$ is
  non-singular and $\mathbf{A}^{-1} \geq 0$.  
\end{theorem}
%%% Local Variables: 
%%% mode: latex
%%% TeX-master: "dissertation"
%%% End: 
