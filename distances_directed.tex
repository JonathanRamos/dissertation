We use the following characterization of positive definite matrix. Let
$\mathbf{A}$ be invertible. Then $\mathbf{A} + \mathbf{A}^{T}$ is
positive definite if and only if $\mathbf{A}^{-1} +
(\mathbf{A}^{-1})^{T}$ is positive definite
\cite{horn94:_topic_in_matrix_analy}. Since $\mathbf{Z}\bm{\Pi}^{-1}$
is invertible,
  \begin{equation}
    \label{eq:34}
    \begin{split}
    \mathbf{Z}\bm{\Pi}^{-1} + \bm{\Pi}^{-1}\mathbf{Z}^{T} \succ 0
    & \leftrightarrow \bm{\Pi}(\mathbf{I} - \mathbf{P} + \mathbf{Q}) + (\mathbf{I} -
    \mathbf{P}^{T} + \mathbf{Q}^{T})\bm{\Pi} \succ 0 \\
    & \leftrightarrow \bm{\Pi}(\mathbf{I} - \mathbf{P}) + (\mathbf{I} -
    \mathbf{P}^{T})\bm{\Pi} + 2 \pi \pi^{T} \succ 0 \\
    & \leftrightarrow \bm{\Pi}(\mathbf{I} - \frac{\mathbf{P} + \mathbf{P}_{*}}{2})
    + 2 \pi \pi^{T} \succ 0
  \end{split}      
  \end{equation}
  where $\mathbf{P}_{*}$ is the time-reverseral of $\mathbf{P}$. Now,
  $\bm{\Pi}(\mathbf{I} - \frac{\mathbf{P} + \mathbf{P}_{*}}{2})$ is
  symmetric. Furthermore, $\frac{\mathbf{P} + \mathbf{P}_{*}}{2}$ is a
  stochastic matrix and thus $\bm{\Pi}(\mathbf{I} - \frac{\mathbf{P} +
    \mathbf{P}_{*}}{2})$ is diagonally dominant. By Ger\u{s}gorin's
  circle theorem \cite{gersgorin31:_uber_abgren_eigen_matrix},
  $\bm{\Pi}(\mathbf{I} - \frac{\mathbf{P} + \mathbf{P}_{*}}{2})$ is
  positive semidefinite. Thus, $\mathbf{Z}\bm{\Pi}^{-1} +
  \bm{\Pi}^{-1}\mathbf{Z}^{T} \succ 0$, and $\Delta_{\delta}$ is an
  EDM-2 matrix.
  % Forest metrics
%An interpretation for the entries in $\mathbf{Q}_{\alpha}$ is also
%available in \cite{chebotarev02:_fores_metric_for_graph_vertic}. Specifically, let 
We now extends the notion of forest metrics as defined by Chebotarev
and Shamis to directed graphs. Let $G = (V,E,\omega)$ be a graph with
similarity measure $\omega$. Let $\mathbf{P}$ be the transition matrix
of $G$. We assumed that $\mathbf{P}$ is irreducible. Consider the
matrix $\mathbf{X}_{\beta} = \mathbf{I} + \alpha \bm{\Pi}(\mathbf{I} -
\mathbf{P})$ with $\beta \geq 0$. Note that for an undirected graph
$G$, $\mathbf{L} = \mathrm{Vol}(G) \bm{\Pi}^{-1}(\mathbf{I} -
\mathbf{P})$. Thus $\mathbf{X}_{\beta}$ subsumes the role of
$(\mathbf{I} + \alpha \mathbf{L})$ for general graphs.
%
%
\noindent Now, $\mathbf{X}_{\beta}$ is strictly diagonally dominant. Furthermore, the
off-diagonal entries of $\mathbf{X}_{\beta}$ is non-positive. Thus,
$\mathbf{X}_{\beta}$ is a $M$-matrix as defined by Definition
\ref{def:8} for all $\beta \geq 0$. By
Theorem \ref{thm:2}, $\mathbf{X}_{\beta}^{-1}$ exists and is a non-negative
matrix. Now, the vector $\bm{1}$ of all ones is an eigenvector of
$\mathbf{X}_{\beta}$ with eigenvalue $1$ since $(\mathbf{I} -
\mathbf{P})\bm{1} = 0$. Thus $\bm{1}$ is also an eigenvector of
$\mathbf{X}_{\beta}^{-1}$ with eigenvalue $1$. Since
$\mathbf{X}_{\beta}^{-1}$ is a
non-negative matrix, we thus have that $\mathbf{X}_{\beta}^{-1}$ is a
stochastic matrix. We summarized the above observations in the
following proposition.  
\begin{proposition}
  \label{prop:9}
  Let $\mathbf{X}_{\beta} = \mathbf{I} + \beta \bm{\Pi}(\mathbf{I} -
  \mathbf{P})$ for some fixed $\beta \geq 0$. Then
  $\mathbf{X}_{\beta}^{-1}$ exists and is a stochastic
  matrix. 
\end{proposition}
From the fact that $\mathbf{X}_{\beta}$ is strictly diagonally dominant, we
see that $\mathbf{X}_{\beta} + \mathbf{X}_{\beta}^{T}$ is symmetric and strictly
diagonally dominant and hence is a positive definite matrix. Thus
$\mathbf{X}_{\beta}^{-1} + (\mathbf{X}_{\beta}^{T})^{-1}$ is also positive definite. 
We thus have the following result 
\begin{proposition}
  \label{prop:13}
  The matrix $\Delta_{\eta_{\beta}} =
  \kappa\Bigl(\tfrac{1}{2}(\mathbf{X}_{\beta}^{-1} +
  (\mathbf{X}_{\beta}^{-1})^{T})\Bigr)$ is an EDM-2 matrix. 
\end{proposition}
Now, if $G$ is an undirected graph, $\mathbf{X}_{\beta} =
\mathbf{X}_{\beta}^{T}$, and $\mathbf{X}_{\beta}^{-1} =
\mathbf{X}_{\beta}^{-1} = \mathbf{Q}_{\alpha}$ for $\alpha =
\beta/\mathrm{Vol}(G)$. Proposition \ref{prop:9} and Proposition
\ref{prop:13} combined to give a generalization of Theorem \ref{thm:4}
to general graphs. 
%%% Local Variables: 
%%% mode: latex
%%% TeX-master: "dissertation"
%%% End: 
