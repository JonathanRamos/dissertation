\documentclass[12pt,reqno,final,ugsabstract,ugsabstractsigs]{iuthesis}
\pagestyle{chapter}
%\usepackage{geometry}
\usepackage{graphicx}
\usepackage{subfigure}
\usepackage{amsmath}
\usepackage{amssymb}
\usepackage{amsthm}
\usepackage{mathrsfs}
\usepackage{enumerate}
\usepackage{bm}
\usepackage{tkz-graph}
\usetikzlibrary{arrows}
\usepackage{subfigure}
\usepackage[ruled]{algorithm2e}
\newtheorem{theorem}{Theorem}
\newtheorem{lemma}[theorem]{Lemma}
\newtheorem{proposition}[theorem]{Proposition}
\theoremstyle{definition}
\newtheorem{definition}[theorem]{Definition}
\usepackage[colorlinks=true,pagebackref,linkcolor=magenta]{hyperref}
\usepackage[colon,sort&compress]{natbib}
\numberwithin{equation}{chapter}
\numberwithin{section}{chapter}
\renewcommand\arraystretch{1.2}
\let\underbrace\LaTeXunderbrace
\let\overbrace\LaTeXoverbrace
\newcommand{\argmax}{\operatornamewithlimits{argmax}}
\newcommand{\argmin}{\operatornamewithlimits{argmin}}
\newcommand{\name}{Minh Tang}
\newcommand{\addr}{150 S. Woodlawn Ave, Bloomington, IN 47401}
\newcommand{\phone}{(812) 391-4892}


%%%%%%%%%%%%%%%%%%%%%%%%%%%%%%%%%%%%%%%%%%%%%%%%%%%%%%%%%
% New commands and environments
                
% This defines how the name looks
\newcommand{\bigname}[1]{
        \begin{center}\fontfamily{phv}\selectfont\Large\scshape#1\end{center}
}

% A ressection is a main section (<H1>Section</H1>)
\newenvironment{ressection}[1]{
	\vspace{4pt}
	{\fontfamily{phv}\selectfont\Large#1}
	\begin{itemize}
	\vspace{3pt}
}{
	\end{itemize}
}

% A resitem is a simple list element in a ressection (first level)
\newcommand{\resitem}[1]{
	\vspace{-4pt}
	\item \begin{flushleft} #1 \end{flushleft}
}

% A ressubitem is a simple list element in anything but a ressection (second level)
\newcommand{\ressubitem}[1]{
	\vspace{-1pt}
	\item \begin{flushleft} #1 \end{flushleft}
}

% A resbigitem is a complex list element for stuff like jobs and education:
%  Arg 1: Name of company or university
%  Arg 2: Location
%  Arg 3: Title and/or date range
\newcommand{\resbigitem}[3]{
	\vspace{-5pt}
	\item
	\textbf{#1}---#2 \\
	\textit{#3}
}

% This is a list that comes with a resbigitem
\newenvironment{ressubsec}[3]{
	\resbigitem{#1}{#2}{#3}
	\vspace{-2pt}
	\begin{itemize}
}{
	\end{itemize}
}

% This is a simple sublist
\newenvironment{reslist}[1]{
	\resitem{\textbf{#1}}
	\vspace{-5pt}
	\begin{itemize}
}{
	\end{itemize}
}
% \usepackage{fancyhdr}
% \pagestyle{fancy}
% \fancyhead{}
% \fancyfoot{}
% \fancyhead[RO]{\slshape \leftmark}
% \fancyfoot[C]{\thepage}
%\renewcommand{\baselinestretch}{2}
\setlength{\topmargin}{0in}
\setlength{\oddsidemargin}{0.5in}
\begin{document}
\title{Graph metrics and dimension reduction}
\author{Minh Tang}
\advisor{Michael W. Trosset}
\secondreader{Dirk van Gucht}
\thirdreader{David Leake}
\fourthreader{Paul Purdom}
\department{Informatics and Computing}
\departmentname{School}
\submitdate{October 2010}
\copyrightyear{2010}
\pagenumbering{roman}
\setcounter{page}{2}
%\input{acceptance.tex}
\begin{acknowledgements}
  It has been a long and sometimes difficult road towards the
  completion of this dissertation. I would like to express my thanks
  to all the people who has made this possible. \\ \\
%
  \noindent I owe my deepest gratitude to my advisor, Prof.~Michael
  W.  Trosset. I met Michael during an important junction of my
  graduate studies when I was lost in my search for a research
  direction.  Michael imparted on me the penchant for clarity of
  thoughts and expression. His kindness and patience had made the
  process of completing this dissertation a much more enjoyable and
  enlightening experience. With high probability, I am indebted to him
  more than I realized. \\ \\
%
  \noindent I would like to express my gratitude to my co-advisor,
  Prof.~Dirk van Gucht. I am grateful to Dirk for the many hours of
  discussion on various topics, from data mining to
  uncertainty. Dirk's guidance during the early stages of my graduate
  studies was crucial in inspiring me to do research. \\ \\
%
  \noindent I had received a lot of help and support from Prof.~Paul
  W. Purdom and Prof.~David Leake. Paul's book was one of my first
  source in learning about analysis of algorithms. Both Paul and David
  had supervised my independent studies and I thank them for their
  guidance and feedback to my semidemihemi-baked ideas. \\ \\
%
  \noindent I would like to thank Dana Fielding for her words of
  encouragement as well as helping me with the paperworks. In
  particular, Dana extricated me from the mess caused by my failure to
  signed up for the required number of credits. I imagine it must have
  been as difficult as pulling a hippopotamus from a marsh. I would
  also like to thank Debbie Norris for her help in that incident. \\ \\
%
  \noindent I am indebted to Prof.~Carey E. Priebe for his influences
  in various aspects of this work. It was the discussions between Carey
  and Michael that motivated a large part of this dissertation. My
  work was funded by grants where Carey and Michael are the principal investigators. \\
  \\
%
  \noindent I am grateful to Prof.~Amol Mali and Prof.~Christine Cheng
  from the University of Wisconsin, Milwaukee. Amol guided me through
  my masters program. Christine introduced me to combinatorics and
  mathematical rigour through her lectures on graphs algorithms. \\ \\
%
  \noindent I have had the pleasure of knowing a lot of nice people
  during my years at IU\@. Special thanks to Brent Castle and Michel
  Salim for the interesting and fun discussions on various topics. I
  thank Aaron Jones, Bill Butske, Bledar Doraci, Carlos Castro, Darius
  Strapoc, Grigor Khatchatryan, Huy Vo, Kerry Krutilla, Mikhail
  Gorshteyn, Paul Kanczuzewski, Snea Thinsan, Van Ly, and Wael Abu
  Shammala for the laughers and fun during the time spent chasing the
  soccer ball together. \\ \\
%
  \noindent Finally, I would like to thank my parents for their love
  and constant support. Without them, all my successes would be for
  naught. \\ \\
%
  \noindent The research described herein was supported by a grant
  from the Office of Naval Research and by a subcontract to Carey
  E. Priebe's National Security Science \& Engineering Faculty
  Fellowship.
\end{acknowledgements}
\begin{abstract}
  Since the introduction of Isomap and
  Locally Linear Embedding in 2000, there has
  been an explosion of interest in techniques for nonlinear dimension
  reduction.  We present a framework that unifies several prominent
  techniques, notably diffusion maps and Laplacian
  eigenmaps.  Our framework relies on the construction of various
  Euclidean distances on undirected graphs and the subsequent
  embedding of these distances in various Euclidean spaces.  We also
  consider how to construct and embed Euclidean distances on directed
  graphs.
\end{abstract}
\frontmatter
\maketitle
\signaturepage
%\copyrightpage
\makeack
\makeabstract
\tableofcontents
\mainmatter
%The contribution of this work as follows. We present a novel framework
for the construction of Euclidean distances on undirected graphs. The
framework employs the concept of random walks on graphs. Expected
commute times and diffusion distances are examples of distance
measures that can be obtained under this framework. We also
investigate the question of how to embed the resulting Euclidean
distances. Two competing approaches are presented, namely the
embedding through classical MDS and the embedding through the
eigensystem of the transition matrix. We then show that several
well-known manifold learning algorithms, namely, Laplacian eigenmaps
and diffusion maps, can be interpreted as constructing algorithms that
embed the appropriate distance measures. This intepretation is novel
in the decoupling of the distance measures from the
embeddings. Furthermore, the decoupling of the distance measures from
the embeddings allows one to investigate the properties of these
algorithms from two different perspectives. As an example, we present
an anomaly that might arise in the embeddings constructed by diffusion
maps. This anomaly suggest itself when we look at the notion of
diffusion distances associated with diffusion maps. \\ \\
%
\noindent We then investigate the question of a framework for the construction
of Euclidean distances on directed graphs. Except for some scattered
results in the literature showing that a particular distance measure,
e.g., expected commute times, on directed graphs is a Euclidean
distance measure, there has not been a systematic investigation of
distance measures on directed graphs. We present two results
summarizing this investigation. The first result indicates that such a
framework, if it exists, would be much more restrictive than its
counterpart for undirected graphs. The second result is a framework
for constructing Euclidean distance measures on directed graphs that
is dependent of the graphs. This is in contrast to the framework for
undirected graphs which is not dependent on the graphs. \\ \\
%
\noindent We also considered the problem of embedding dissimilarity measures for
directed graphs. In particular, we show that, in contrast to
embeddings through classical MDS, the embeddings of the distance
measures using the eigensystems of matrices such as the transition
matrices or the Laplacian matrices do not extends to directed
graphs. Therefore, the current presentations in the literature of
manifold learning algorithms such as Laplacian eigenmaps and diffusion
maps do not extend naturally to directed graphs. This reaffirms our
view of decoupling the distance measures from the embeddings. Our last
contribution is a novel approach to embedding directed (assymetric)
proximity data. The approach can be understood as a hybrid three-way
MDS/asymmetric model. Thus, for example, we can embed the mean first
passage times directly, instead of the need to first symmetrize the
mean first passage times to expected commute times.

%%% Local Variables: 
%%% mode: latex
%%% TeX-master: "dissertation"
%%% End: 

\chapter{Introduction}
\label{cha:introduction}

\section{Hessian eigenmaps}
\label{sec:hessian-eigenmaps}
Hessian eigenmaps \citet{donoho03:_hesian} was described as a locally
linear embedding technique for high-dimensionality data. Hessian
eigenmaps assume that the data points lies on a  
Riemannian manifold $\mathcal{M} \subset \mathbb{R}^{n}$ such
that $\mathcal{M}$ is locally isometric to an open connected subset
$\Omega \subset \mathbb{R}^{d}$. For $f \colon \mathcal{M} \mapsto
\mathbb{R}$, define a quadratic form $\mathcal{H}(f)$ associated with
$f$ by $\mathcal{H}(f) = \int_{M}{ \| H_f(m) \|_{F}^{2} dm }$ where
$H_f(m)$ is the Hessian of $f$ at $m$ and $\| \cdot \|_F$ is the
Frobenius norm. $\mathcal{H}(f)$ averages the Frobenius norm of the
Hessian of $f$ over $\mathcal{M}$. The embedding coordinates
corresponds to the basis for the null space of $\mathcal{H}(f)$. The
procedure can be described as follows. 

Let $\{m_i\}_{i=1}^{N}$ be a collection of $N$ points in
$\mathbb{R}^{n}$. Suppose that $K$ is a parameter chosen to be the
size of the nearest neighborhoods and $d$ is a parameter chosen to be
the dimension of the embedding coordinates. The Hessian eigenmaps
procedure proceed as follows \citet{donoho03:_hesian}
\begin{enumerate}
\item Identify neighbors: For each $m_i$, set $\mathcal{N}_i$ to be
  the set of the $K$ nearest neighbours of $m_i$. For each
  neighbourhood $\mathcal{N}_i$, set $\mathbf{M}_i$ to be the $K \times n$
  matrix whose rows are the points $m_j \in \mathcal{N}_i$. Define
  $\tilde{\mathbf{M}_i}$ to be $(\mathbf{I} -
  \mathbf{J}/K)\mathbf{M}_i$, i.e. $\tilde{\mathbf{M}_i}$ is
  the row centering of $\mathbf{M}_i$.
\item Do principal component analysis (PCA) on
  $\tilde{\mathbf{M}_i}$. This is equivalent to finding the singular
  value decomposition (SVD) of $\tilde{\mathbf{M}_i} = \mathbf{U}_i
  \bm{\Lambda} \mathbf{V}_i$ and using the first $d$ columns of
  $\mathbf{U}_i$ as the tangent coordinates of points in
  $\mathcal{N}_i$.
\item Estimate the entries of the Hessian coordinates by fitting a
  polynomial of degree two 
\end{enumerate}


%%% Local Variables: 
%%% mode: latex
%%% TeX-master: "dissertation"
%%% End: 

\chapter{Distances on undirected graphs}
\label{cha:dist-undir-graphs}
The main theme of this chapter is distances on undirected graphs. As
we have seen in the discussion of \S \ref{cha:introduction}, the
notions of distances on graphs played an important role in several
manifold learning algorithms. This chapter focus on several related
notion of distances on graphs such as expected commute time, diffusion
distances and forest metrics. The main thread connecting these graph
metrics is the notion of random walks on graphs. This leads us to
the consideration of graph metrics that can be expressed as series
expansion of the probability transion matrices in \S
\ref{sec:graph-metr-funct}. Other notion of distances on graph had
been proposed. For example, \citet{yen08:_famil_of_dissim_measur_between}
and \citet{chebotarev08:_new_famil_of_graph_distan} proposed families
of graph distances that approaches shortest path distances and
expected commute time in the limits.
\section{Expected commute time}
\label{sec:expect-comm-time}
Let $G = (V,E,\omega)$ be an undirected graph, with $\omega$ being the
similarity measure. We assume that the transition matrix
$\mathbf{P}$ on $G$ is irreducible and aperiodic, see
\S ~\ref{sec:finite-markov-chain}. The {\em expected commute time}
$\delta(u,v)$ between $u \in V$ and $v \in V$ is defined as
\begin{equation}
  \label{eq:25}
  \delta(u,v) = \mathbb{E}_{u}[\tau_v] + \mathbb{E}_{v}[\tau_u]
\end{equation}
%
\noindent Let $\mathbf{M}$ be the matrix of mean first passage time,
i.e., $\mathbf{M}(u,v) = \mathbb{E}_{u}[\tau_v]$. The following
proposition shows that $\mathbf{M}$ is the unique solution of a given
matrix equation.
\begin{proposition}
  \label{prop:4}
 $\mathbf{M}$ is the unique solution of the following matrix equation
  \begin{equation}
    \label{eq:3}
   (\mathbf{I} - \mathbf{P})\mathbf{X} = \mathbf{J} - \bm{\Pi}^{-1}
  \end{equation}
  subjected to the condition 
  \begin{equation}
    \label{eq:32}
 \mathbf{M}_{\mathrm{dg}} = \mathbf{0}, \qquad \mathbf{M}(u,v) \geq 0   
  \end{equation}
  where $\mathbf{M}_{\mathrm{dg}}$ is the diagonal matrix formed
    by setting the off-diagonal entries of $\mathbf{M}$ to zero.
    Thus, $\mathbf{M} = \mathbf{X} -
    \mathbf{J}\mathbf{X}_{\mathrm{dg}}$ where $\mathbf{X}$ satisfy
    Eq. \eqref{eq:3}.v
\end{proposition}
\begin{proof}
  If $u = v$, then $\mathbb{E}_{u}[\tau_u] = 0$ and thus
  $\mathbf{M}(u,u) = 0$. Otherwise, if $u \not = v$, then
  $\mathbf{M}(u,v) = \mathbb{E}_{u}[\tau_v]$ can be expanded as
  \begin{equation}
    \label{eq:4}
    \mathbb{E}_{u}[\tau_v] = 1 + \sum_{w \in V}{\mathbf{P}(u,w)
      \mathbb{E}_{w}[\tau_v]} = 1 + (\mathbf{PM})(u,v)
  \end{equation}
  Thus, $\mathbf{F} = \mathbf{J} + (\mathbf{P} -
  \mathbf{I})\mathbf{M}$ is a diagonal matrix. Futhermore, if $\pi$ is
  the vector of stationary distribution, then
  \begin{equation}
    \label{eq:26}
    \pi^{T} \mathbf{F} = \pi^{T} \mathbf{J} + \pi^{T} (\mathbf{P} -
    \mathbf{I})\mathbf{M} = \mathbf{1}   
  \end{equation}
  Therefore, $\mathbf{F}(u,u) = 1/\pi(u)$ and thus $\mathbf{F} =
  \bm{\Pi}^{-1}$. We thus have $\bm{\Pi}^{-1} = \mathbf{J} +
  (\mathbf{P} - \mathbf{I})\mathbf{M}$. $\mathbf{M}$ is then a 
  solution of the matrix equation as given by Eq.~\eqref{eq:3}.

  We now show that $\mathbf{M}$ is the unique solution of
  Eq.~\eqref{eq:3} subjected to the condition in
  Eq.~\eqref{eq:32}. Let $\mathbf{M}'$ be another solution of
  Eq.~\eqref{eq:3} subjected to the condition in
  Eq.~\eqref{eq:32}. Then $\mathbf{Y} = \mathbf{M} - \mathbf{M}'$
  satisfy
  \begin{equation}
    \label{eq:19}
    (\mathbf{I} - \mathbf{P})\mathbf{Y} = \mathbf{0}
  \end{equation}
  By Lemma ~\ref{lem:1}, each column of $\mathbf{Y}$ is constant. Since
  $\mathbf{M}_{\mathrm{dg}} = \mathbf{M'}_{\mathrm{dg}} = \mathbf{0}$, each
  column of $\mathbf{Y}$ must be identically $0$. Thus $\mathbf{M} = \mathbf{M'}$,
  proving the uniqueness of $\mathbf{M}$. If $\mathbf{X}$ satisfy
  Eq. \eqref{eq:3}, then $\mathbf{X} - \mathbf{J}\mathbf{X}_{\mathrm{dg}}$
  satisfy the condition in Eq.~\eqref{eq:32}.
\end{proof}
%
\begin{proposition}
  \label{prop:5}
  Let $\mathbf{Q} = \mathbf{1}^{T}\mathbf{\pi}$ be the matrix with
  each row being the stationary distribution $\pi$. The matrix
  $\mathbf{M}$ of mean first passage time is given by
  \begin{equation}
    \label{eq:21}
    \mathbf{M} = \mathbf{J}(\mathbf{Z} \bm{\Pi}^{-1})_{\mathrm{dg}} - \mathbf{Z}
    \bm{\Pi}^{-1}
  \end{equation}
  where $\mathbf{Z} = (\mathbf{I} - \mathbf{P} + \mathbf{Q})^{-1}$. 
\end{proposition}
\begin{proof}
  From Proposition ~\ref{prop:7}, we know that $\mathbf{Z} =
  (\mathbf{I} - \mathbf{P} + \mathbf{Q})^{-1}$ is well defined.  We
  first show that $\mathbf{X} = (\mathbf{I} - \mathbf{P} +
  \mathbf{Q})^{-1}(\mathbf{J} - \bm{\Pi}^{-1})$ satisfy
  Eq. \eqref{eq:3}. We have from Proposition ~\ref{prop:8} that
  $(\mathbf{I} - \mathbf{P})\mathbf{X} = (\mathbf{I} -
  \mathbf{P})\mathbf{Z}(\mathbf{J} - \bm{\Pi}^{-1}) = (\mathbf{I} -
  \mathbf{Q})(\mathbf{J} - \bm{\Pi}^{-1})$. Since
  $\mathbf{Q}\mathbf{J} = \mathbf{J} = \mathbf{Q}\bm{\Pi}^{-1}$, one
  has
  \begin{equation}
    \label{eq:27}
    (\mathbf{I} - \mathbf{P})\mathbf{X} =(\mathbf{I} - \mathbf{Q})(\mathbf{J} -
  \bm{\Pi}^{-1}) = \mathbf{J} - \bm{\Pi}^{-1}
  \end{equation}
  and thus $\mathbf{X}$ satisfy Eq. \eqref{eq:3}. Also, from Proposition
  ~\ref{prop:8}, we have
  \begin{equation}
    \label{eq:31}
    \mathbf{X} = \mathbf{Z}(\mathbf{J} - \bm{\Pi}^{-1}) = \mathbf{J} -
    \mathbf{Z}\bm{\Pi}^{-1}
  \end{equation}
  and thus $\mathbf{X} - \mathbf{J}\mathbf{X}_{\mathrm{dg}} =
  \mathbf{J}(\mathbf{Z}\bm{\Pi}^{-1})_{\mathrm{dg}} - \mathbf{Z}\bm{\Pi}^{-1}$. 
\end{proof}
The expected commute time $\delta(u,v)$ is then just
$\delta(u,v) = \mathbf{M}(u,v)
+ \mathbf{M}^{T}(u,v)$. Thus, if we let $\Delta_{\delta}$ be the
matrix of expected commute time between the vertices, then
\begin{equation}
  \label{eq:52}
  \Delta_\delta = \mathbf{M} + \mathbf{M}^{T} = 
  \mathbf{J}(\mathbf{Z}\bm{\Pi}^{-1})_{\mathrm{dg}} - \mathbf{Z}\bm{\Pi}^{-1} -
  \bm{\Pi}^{-1}\mathbf{Z}^{T} + (\bm{\Pi}^{-1}\mathbf{Z})_{\mathrm{dg}}\mathbf{J}
\end{equation}
We now show that $\Delta_\delta$ is an EDM-2 matrix whenever $G$ is an
undirected graph. 
\begin{proposition}
  \label{prop:10}
  If $G$ is an undirected graph, then $\mathbf{Z}\bm{\Pi}^{-1} =
  \bm{\Pi}^{-1}\mathbf{Z}^{T}$. The matrix $\Delta_{\delta}$ of
  expected commute time is then given by 
  \begin{equation}
    \label{eq:33}
 \Delta_{\delta}  
  = \mathbf{J}(\mathbf{Z}\bm{\Pi}^{-1})_{\mathrm{dg}} - \mathbf{Z}\bm{\Pi}^{-1} -
  \bm{\Pi}^{-1}\mathbf{Z}^{T} + (\bm{\Pi}^{-1}\mathbf{Z})_{\mathrm{dg}} \mathbf{J} =
   \kappa(\mathbf{Z}\bm{\Pi}^{-1})
  \end{equation}
  The matrix $\mathbf{Z}\bm{\Pi}^{-1}$
  is positive definite, and $\Delta_{\delta}$ is an EDM-2 matrix.
\end{proposition}
\begin{proof}
  From Proposition ~\ref{prop:15}, we have $\mathbf{P}\bm{\Pi}^{-1} =
  \bm{\Pi}^{-1}\mathbf{P}^{T}$. Thus, $\mathbf{P}^{k}\bm{\Pi}^{-1} =
  \bm{\Pi}^{-1}(\mathbf{P}^{T})^{k}$. Furthermore, $\mathbf{Q}\bm{\Pi}^{-1} =
  \mathbf{J}$ and so
  \begin{equation}
    \label{eq:53}
    \mathbf{Z}\bm{\Pi}^{-1} = \biggl[\mathbf{I} +
    \sum_{k=1}^{\infty}(\mathbf{P}^{k} -
    \mathbf{Q})\biggr]\bm{\Pi}^{-1} =
    \bm{\Pi}^{-1}\biggl[\mathbf{I} +
    \sum_{k=1}^{\infty}((\mathbf{P}^{T})^{k} - \mathbf{Q}^{T})\biggr]
    = \bm{\Pi}^{-1}\mathbf{Z}^{T}
  \end{equation}
  Now, $\mathbf{Z}\bm{\Pi}^{-1}$ is positive definite if and only if
  $\bm{\Pi}\mathbf{Z}^{-1} = \bm{\Pi}(\mathbf{I} - \mathbf{P} +
  \mathbf{Q}) \succ 0$. Since $\bm{\Pi}(\mathbf{I} - \mathbf{P} +
  \mathbf{Q}) = \bm{\Pi}(\mathbf{I} - \mathbf{P}) + \pi\pi^{T}$, we
  see that $\bm{\Pi}\mathbf{Z}^{-1} \succ 0$ if $\bm{\Pi}(\mathbf{I} -
  \mathbf{P}) \succeq 0$. We know that $\bm{\Pi}(\mathbf{I} -
  \mathbf{P})$ is symmetric and diagonally dominant (see Definition
  ~\ref{def:4}). By Ger\u{s}gorin circle theorem, the eigenvalues of
  $\bm{\Pi}(\mathbf{I} - \mathbf{P})$ are non-negative. Thus,
  $\bm{\Pi}(\mathbf{I} - \mathbf{P}) \succeq 0$ and
  the claim that $\mathbf{Z}\bm{\Pi}^{-1}$ is positive definite
  follows. $\Delta_{\delta} = \kappa(\mathbf{Z}\bm{\Pi}^{-1})$ is then
  an EDM-2 matrix.
\end{proof}
%
There exists in the literatures a notion of distances known as
resistance distance
\citep{bapat99:_resis_distan_in_graph,klein93:_resis_distan}. Let $G =
(V,E,\omega)$ be an undirected graph with similarity measure
$\omega$. Let $\mathbf{L}$ be the combinatorial Laplacian of $G$. The
resistance distance $r(u,v)$ between $u, v \in V$ is defined as
\begin{equation}
  \label{eq:35}
  r(u,v) = \tfrac{1}{2}(\mathbf{L}^{\dagger}(u,u) - \mathbf{L}^{\dagger}(u,v) -
  \mathbf{L}^{\dagger}(v,u) + \mathbf{L}^{\dagger}(v,v))
\end{equation}
where $\mathbf{L}^{\dagger}$ is the {\em Moore-Penrose} \/pseudo-inverse of
$\mathbf{L}$. It's widely known that for undirected graphs, resistance
distance is proportional to expected commute
$\delta(u,v)$. Specifically,
\begin{equation}
  \label{eq:36}
  r(u,v) = \frac{2 \delta(u,v)}{\mathrm{Vol}(G)}
\end{equation}
Eq.~\eqref{eq:36} is an easy corollary of the following result.
\begin{proposition}
  \label{prop:11}
  Let $G = (V,E,\omega)$ be an undirected graph with $|V| = n$. The
  Moore-Penrose pseudo-inverse $\mathbf{L}^{\dagger}$ of $\mathbf{L}$ is given
  by
  \begin{equation}
    \label{eq:37}
    \mathbf{L}^{\dagger} = c \Bigl(\mathbf{I} - \frac{\mathbf{J}}{n}\Bigr) \mathbf{Z}
    \bm{\Pi}^{-1} \Bigl(\mathbf{I} - \frac{\mathbf{J}}{n}\Bigr)
  \end{equation}
  where $c = 1/\mathrm{Vol}(G)$ is a constant. 
\end{proposition}
\begin{proof}
  We will show that $\mathbf{L}^{\dagger}$ as defined by Eq.~\eqref{eq:37}
  satisfies the following conditions for a Moore-Penrose pseudo-inverse
  \begin{gather*}
    \mathbf{L}\mathbf{L}^{\dagger} = \mathbf{L}^{\dagger}\mathbf{L} \tag{(i)} \\
    \mathbf{L}\mathbf{L}^{\dagger}\mathbf{L} = \mathbf{L} \tag{(ii)} \\
    \mathbf{L}^{\dagger}\mathbf{L} \mathbf{L}^{\dagger} = \mathbf{L}^{\dagger}
    \tag{(iii)}
  \end{gather*}
  If $G = (V,E,\omega)$ is an undirected graph, then $\pi(u) =
  \deg(u)/\mathrm{Vol}(G)$ and thus $\mathbf{D} = \mathrm{Vol}(G)
  \bm{\Pi}$. Therefore $\mathbf{L} = \mathbf{D}(\mathbf{I} - \mathbf{P}) =
  \mathrm{Vol}(G) \bm{\Pi}(\mathbf{I} - \mathbf{P})$. We also have
  \begin{equation}
    \label{eq:38}
    \Bigl(\mathbf{I} - \frac{\mathbf{J}}{n}\Bigr)\mathbf{L} = \mathbf{L}
  \end{equation}
  and thus
  \begin{equation}
    \label{eq:39}
    \begin{split}
      \mathbf{L}^{\dagger}\mathbf{L} &= \Bigl(\mathbf{I} -
      \frac{\mathbf{J}}{n}\Bigr) \mathbf{Z}
      \bm{\Pi}^{-1} \bm{\Pi}(\mathbf{I} - \mathbf{P}) \\
      &= \Bigl(\mathbf{I} - \frac{\mathbf{J}}{n}\Bigr) (\mathbf{I} -
      \mathbf{P} + \mathbf{Q})^{-1}
      (\mathbf{I} - \mathbf{P}) \\
      &= \Bigl(\mathbf{I} - \frac{\mathbf{J}}{n}\Bigr)(\mathbf{I} - \mathbf{Q}) \\
      &= \Bigl(\mathbf{I} - \frac{\mathbf{J}}{n}\Bigr)
   \end{split}
  \end{equation}
  Similarly,
  \begin{equation}
    \label{eq:40}
    \begin{split}
      \mathbf{L}\mathbf{L}^{\dagger} &= \bm{\Pi}(\mathbf{I} -
      \mathbf{P}) \mathbf{Z}
      \bm{\Pi}^{-1} \Bigl(\mathbf{I} - \frac{\mathbf{J}}{n}\Bigr) \\
      &= \bm{\Pi}(\mathbf{I} - \mathbf{P}) (\mathbf{I} - \mathbf{P} +
      \mathbf{Q})^{-1}
      \bm{\Pi}^{-1} \Bigl(\mathbf{I} - \frac{\mathbf{J}}{n}\Bigr) \\
      &= (\mathbf{I} - \mathbf{Q}^{T})\Bigl(\mathbf{I} - \frac{\mathbf{J}}{n}\Bigr) \\
      &= \Bigl(\mathbf{I} - \frac{\mathbf{J}}{n}\Bigr)
   \end{split}
  \end{equation}
  Thus, condition (i) is satisfied. Furthermore, we also have, from
  Eq.~\eqref{eq:39} and Eq.~\eqref{eq:40}, that
  \begin{gather*}
    \mathbf{L}\mathbf{L}^{\dagger}\mathbf{L} = \Bigl(\mathbf{I} -
    \frac{\mathbf{J}}{n}\Bigr) \mathbf{L}
    = \mathbf{L} \\
    \mathbf{L}^{\dagger}\mathbf{L}\mathbf{L}^{\dagger} = c
    \Bigl(\mathbf{I} - \frac{\mathbf{J}}{n}\Bigr) \mathbf{Z}
    \bm{\Pi}^{-1} \Bigl(\mathbf{I} - \frac{\mathbf{J}}{n}\Bigr)
    \Bigl(\mathbf{I} - \frac{\mathbf{J}}{n}\Bigr) = c \Bigl(\mathbf{I}
    - \frac{\mathbf{J}}{n}\Bigr) \mathbf{Z} \bm{\Pi}^{-1}
    \Bigl(\mathbf{I} - \frac{\mathbf{J}}{n}\Bigr) =
    \mathbf{L}^{\dagger}
  \end{gather*}
  Thus, $\mathbf{L}^{\dagger}$ as defined by Eq.~\eqref{eq:37} is the
  Moore-Penrose inverse of $\mathbf{L}$. 
\end{proof}
Proposition ~\ref{prop:11} is equivalent to saying that
$\mathbf{L}^{\dagger}$ is a constant times the double centering of
$\mathbf{Z}\bm{\Pi}^{-1}$. By Proposition ~\ref{prop:16}, we have 
\begin{equation}
  \label{eq:42}
 \Delta_{\delta} = \kappa(\mathbf{Z}\bm{\Pi}^{-1}) = \mathrm{Vol}(G)
\kappa(\mathbf{L}^{\dagger}) 
\end{equation}
Eq.~\eqref{eq:36} thus follows from Proposition ~\ref{prop:11}, as
claimed. \\
\\
\noindent
From Eq.~\eqref{eq:36} we can make an observation about expected
commute time and resistance distances. Expected commute time is scale
invariant with respect to the similarity measure $\omega$, i.e., if we
replace $G = (V,E,\omega)$ with $G' = (V,E,\omega')$ where $\omega' =
\alpha \omega$ and $\alpha > 0$ is a constant, then $\delta_{G}(u,v) =
\delta_{G'}(u,v)$. Resistance distance is however not scale
invariant. In fact, we have $r_{G'}(u,v) = r_{G}(u,v)/\alpha$. This
also leads to different results between expected commute time and
resistance distances when we consider the union of graphs. Let $G_1 =
(V_1, E_1, \omega_1)$ and $G_2 = (V_2,E_2, \omega_2)$ be two graphs
and form their union by joining a $u \in V_1$ to a $v \in V_2$ with
edge weight $1$. If $G_1$ and $G_2$ are sufficiently large, then the
change in expected commute time between the vertices is on a smaller
scale than the change in volume, and resistance distance betwen the
vertices start to get closer to each other. For large graphs, this
might mean that resistance distance between the vertices will not be a
useful measure of distance. \citet{radl09} showed that for some models
of random graphs, resistance distances between vertices converges to
the sum of the degree of the vertices. These observations means that
even though these two notion of distances
are closely related, they are not exactly equivalent.
\section{Diffusion distances}
\label{sec:diffusion-distances}
Let $G = (V,E,\omega)$ be an undirected graph with
$\omega$ being a similarity measure between vertices of $V$. Denote by
$\mathbf{P}$ the probability transition matrix of $G$. The diffusion
distances at time $t$, $\rho_{t}(u,v)$, between $u,v \in V$ is defined as
\citep{coifman06:_diffus_maps}
\begin{equation}
  \label{eq:43}
  \rho^{2}_{t}(u,v) = \sum_{w \in V}{\Bigl(\mathbf{P}^{t}(u,w) -
      \mathbf{P}^{t}(v,w)\Bigr)^2 \frac{1}{\pi(w)}}
\end{equation}
\begin{proposition}
  \label{prop:12}
  Diffusion distances as defined by Eq.~\eqref{eq:43} can also be
  written as
  \begin{equation}
    \label{eq:44}
    \begin{split}
      \rho_{t}^{2}(u,v) &= \frac{\mathbf{P}^{2t}(u,u) -
        \mathbf{P}^{2t}(v,u)}{\pi(u)} +
      \frac{\mathbf{P}^{2t}(v,v) -
        \mathbf{P}^{2t}(u,v)}{\pi(v)}  \\
      &= (\mathbf{P}^{2t}\bm{\Pi}^{-1})(u,u) -
      (\mathbf{P}^{2t}\bm{\Pi}^{-1})(v,u) \\
      &+ (\mathbf{P}^{2t}\bm{\Pi}^{-1})(v,v) -
      (\mathbf{P}^{2t}\bm{\Pi}^{-1})(u,v)
    \end{split}
  \end{equation}
The matrix of squared diffusion distances $\Delta_{\rho_t^2}$ is then
$\Delta_{\rho_t^2} = \kappa(\mathbf{P}^{2t}\bm{\Pi}^{-1})$. 
\end{proposition}
\begin{proof}
  Since $G$ is undirected, $\mathbf{P}$ is time-reversible and hence
  \begin{equation}
    \label{eq:45}
    \pi(u) \mathbf{P}(u,v) = \pi(v) \mathbf{P}(v,u) 
  \end{equation}
  By
  expanding the square of $(\mathbf{P}^{t}(u,w) - \mathbf{P}^{t}(v,w))^{2}$ in
  Eq.~\eqref{eq:43} and using Eq.~\eqref{eq:45}, one has
  \begin{equation}
    \label{eq:46}
    \begin{split}
      \rho_{t}^{2}(u,v) &= \sum_{w \in V}{\Bigl(\mathbf{P}^{t}(u,w) -
        \mathbf{P}^{t}(v,w)\Bigr)^2 \frac{1}{\pi(w)}} \\
      &= \sum_{w \in V}{\frac{\mathbf{P}^{t}(u,w)\mathbf{P}^{t}(u,w) -
          \mathbf{P}^{t}(u,w)\mathbf{P}^{t}(v,w)}{\pi(w)}} \\
      &+\sum_{w \in V}{\frac{\mathbf{P}^{t}(v,w)\mathbf{P}^{t}(v,w) -
          \mathbf{P}^{t}(v,w)\mathbf{P}^{t}(u,w)}{\pi(w)}} \\
      &= \sum_{w \in
        V}{\frac{\mathbf{P}^{t}(u,w)\mathbf{P}^{t}(w,u) -
          \mathbf{P}^{t}(v,w)\mathbf{P}^{t}(w,u)}{\pi(u)}} \\ &+
      \sum_{w \in V}{\frac{\mathbf{P}^{t}(v,w)\mathbf{P}^{t}(w,v)
          -
          \mathbf{P}^{t}(u,w)\mathbf{P}^{t}(w,v)}{\pi(v)}} \\
      &= \frac{(\mathbf{P}^{2t})(u,u) -
        (\mathbf{P}^{2t})(v,u)}{\pi(u)} +
      \frac{(\mathbf{P}^{2t})(v,v) -
        (\mathbf{P}^{2t})(u,v)}{\pi(v)} 
    \end{split} 
  \end{equation}
  which is exactly Eq.~\eqref{eq:44}. 
\end{proof} 
Since $\mathbf{P}^{2t}\bm{\Pi}^{-1} =
\mathbf{P}^{t}\mathbf{P}^{t}\bm{\Pi}^{-1} =
\mathbf{P}^{t}\bm{\Pi}^{-1}\mathbf{P}^{T} \succeq 0$,
$\Delta_{\rho_{t}^2} = \kappa(\mathbf{P}^{2t}\bm{\Pi}^{-1})$ is an
EDM-2 matrix. We state the above observation as the following
proposition
\begin{proposition} 
\label{prop:14} 
Let $G$ be an undirected graph and $\mathbf{P}$ be its transition
matrix. The matrix $\Delta_{\rho_{t}^2}$ of squared diffusion distances is 
an EDM-2 matrix for any $t$. $\Delta_{\rho_t}$ is then an EDM-1 matrix.
\end{proposition}
%
We would now like to make an observation regarding diffusion distances
on undirected graphs. From Eq.~\eqref{eq:46}, we observed that
$\rho_{t}^{2}(u,v)$ only depends on the probability between nodes
connected by paths of length $2t$. Thus, diffusion distances between
any two nodes $u$ and $v$ of $G$ for any time scale $t$ only keeps
tracks of paths of even length in $G$. Diffusion distances might
be unintuitive in some scenarios. For a contrived example,
consider the case where $G$ is a cycle. Then, there might be pairs of nodes that are
adjacent to each other and that have diffusion distances larger than
the nodes that are on two different segments of the cycle. We will
take a closer look at this phenomenon in a later chapter of the
thesis. \\ \\
%
\noindent
We now note the connection between expected commute time and diffusion
distances for when $G$ is an undirected graph. From Proposition
~\ref{prop:16}, we have
\begin{equation}
  \label{eq:49}
\kappa(\mathbf{P}^{2t}\bm{\Pi^{-1}}) =
\kappa(\mathbf{P}^{2t}\bm{\Pi^{-1}} - \mathbf{J}) =
\kappa((\mathbf{P}^{2t} - \mathbf{Q}) \bm{\Pi^{-1}})
\end{equation}
Let $\mathbf{T}_{m} = \Bigl(\mathbf{I} + \sum_{k =
  1}^{m}{(\mathbf{P}^{k} - \mathbf{Q})}\Bigr)\bm{\Pi}^{-1}$ for $m
\geq 0$. Then $\| \mathbf{T}_m - \mathbf{Z}\bm{\Pi}^{-1} \| \rightarrow 0$ as
$m \rightarrow \infty$. This can be seen as follow. 
\begin{equation}
  \label{eq:47}
  \begin{split}
  \| \mathbf{T}_m - \mathbf{Z}\bm{\Pi}^{-1} \| &=
  \|\sum_{k=m+1}^{\infty}(\mathbf{P} - \mathbf{Q})^{k}\bm{\Pi}^{-1}
    \| \\
   &\leq \| \sum_{k=m+1}^{\infty}(\mathbf{P} - \mathbf{Q})^{k} \|
   \|\bm{\Pi}^{-1} \| \\
   &\leq \sum_{k=m+1}^{\infty} \|(\mathbf{P} - \mathbf{Q})^{k} \|
   \|\bm{\Pi}^{-1} \| \\
   &\leq \sum_{k=m+1}^{\infty} (p+\epsilon)^{k} \| \bm{\Pi}^{-1} \| \\
   &\leq C (p+\epsilon)^{m+1}
  \end{split}
\end{equation}
where $C < \infty$, $p < 1$ is the spectral radius of $\mathbf{P} -
\mathbf{Q}$ and $p + \epsilon < 1$. The last term in Eq.~\eqref{eq:47}
tends to $0$ as $m \rightarrow \infty$, and so $\| \mathbf{T}_m -
\mathbf{Z}\bm{\Pi}^{-1} \| \rightarrow 0$. Now, for any $n$, $\kappa$
is a bounded linear operator from the vector space of $n \times n$
square matrices to the space of $n \times n$ square matrices. Thus, we
have
\begin{equation}
  \label{eq:50}
  \lim_{m \rightarrow \infty}\kappa(\mathbf{T}_m) = \lim_{m \rightarrow \infty}
  \sum_{k=0}^{m}{\kappa((\mathbf{P}^{m} - \mathbf{Q})\bm{\Pi}^{-1})} =
    \kappa(\mathbf{Z}\bm{\Pi}^{-1})
\end{equation}
Thus, if we let $\breve{\mathbf{P}} = \mathbf{P}^{2}$ be the transition matrix
of the two-step random walk on $G$, then $\mathbf{P}^{2t} =
\breve{\mathbf{P}}^{t}$ and also that $\breve{\mathbf{Q}} = \mathbf{Q}$ and thus
\begin{equation}
  \label{eq:51}
  \sum_{t = 0}^{\infty} \Delta_{\rho_{t}^{2}} = \sum_{t = 0}^{\infty}
  \kappa((\breve{\mathbf{P}}^{t} - \mathbf{Q})\bm{\Pi}^{-1}) =
  \kappa(\breve{\mathbf{Z}} \bm{\Pi}^{-1})
\end{equation}
where $\breve{\mathbf{Z}}$ is the fundamental matrix for
$\breve{\mathbf{P}}$. Thus, the expected commute time with respect to
$\breve{\mathbf{P}}$ is the
sum of the diffusion distances with respect to $\mathbf{P}$ at all
time-scale $t$. We note this fact in the following proposition.
\begin{proposition}
  \label{prop:17}
  Let $G = (V,E,\omega)$ be an undirected graph and $\mathbf{P}$ be
  the transition matrix of the random walk on $G$. $\mathbf{P}^{2}$ is
  then the transition matrix of the two-step random walk on $G$. We
  denote by $\rho_{t}^{2}$ the squared diffusion distances between
  vertices of $G$ with respect to the transition matrix
  $\mathbf{P}$. We also denote by $\delta_P^{2}$ the expected commute
  time between vertices of $G$ with respect to the two-step random
  walk as given by $\mathbf{P}^{2}$. We then have
  \begin{equation}
    \label{eq:34}
    \delta_{P^{2}}(u,v) = \sum_{t = 0}^{\infty}{\rho_{t}^{2}(u,v)}
  \end{equation}
  The sum in Eq.~\eqref{eq:34} is convergent by Eq.~\eqref{eq:51}.
\end{proposition}
The above proposition was stated incorrectly in 
\citep{qui07:_clust}. The reasoning in
\citep{qui07:_clust} leads to the replacement of the term
$\delta_{P^2}(u,v)$ by the term $\delta_{P}(u,v)$ on the left hand
side of Eq.~\eqref{eq:34}. 
%
%
\section{Forest metrics}
\label{sec:forest-metrics}
Let $G = (V,E,\omega)$ be an undirected graph with $\omega$ being the
similarity measure between vertices of $G$. Denote by $\mathbf{L}$ the
combinatorial Laplacian of $G$. Let $\alpha \geq 0$ be a fixed
constant and defined the matrix $\mathbf{Q}_{\alpha}$ by
\begin{equation}
  \label{eq:30}
  \mathbf{Q}_{\alpha} = (\mathbf{I} + \alpha \mathbf{L})^{-1}
\end{equation}
%
\citep{chebotarev02:_fores_metric_for_graph_vertic} defined a notion of
distance $\eta_\alpha(u,v)$ between vertices of $G$ by
\begin{equation}
  \label{eq:41}
  \eta_\alpha(u,v) = \mathbf{Q}_\alpha(u,u) - \mathbf{Q}_\alpha(u,v) -
  \mathbf{Q}_\alpha(v,u) + \mathbf{Q}_\alpha(v,v)
\end{equation}
The $\eta_{\alpha}$ is called a family of {\em forest metrics} \/ on $G$.
Some properties of forest metrics are given below, see
\citep{chebotarev02:_fores_metric_for_graph_vertic}.
\begin{theorem}
  \label{thm:4}
  For any $\alpha \geq 0$, $\mathbf{Q}_{\alpha}$ is positive
  definite. Furthermore, $\mathbf{Q}_{\alpha}$ is a doubly stochastic
  matrix. The matrix $\Delta_{\eta_{\alpha}}$ of forest metrics
  between vertices of $G$ is thus an EDM-2 matrix for all $\alpha \geq
  0$.
\end{theorem}
The following interpretation for the entries of $\mathbf{Q}_{\alpha}$
was given in \citep{chebotarev02:_fores_metric_for_graph_vertic}. A
forest on $G = (V,E,\omega)$ is an {\em acyclic} \/ subgraph $G' =
(V,E',\omega')$ of $G$ where $E' \subset E$, and $\omega'$ is $\omega$
restricted to $E'$. A tree is thus a forest with a single connected
component. Define the weight $\varepsilon(F)$ of a forest $F =
(V,E',\omega')$ to be
\begin{equation}
  \varepsilon(F) = \prod_{e \in E'}{\omega'(e)} 
\end{equation}
If $\mathscr{F} = \{F_1, F_2, \dots, F_m\}$ is a collection of forests
on $G$, we define the weight $\varepsilon(\mathscr{F})$ of
$\mathscr{F}$ by $\varepsilon(\mathscr{F}) = \sum_{F_k \in
  \mathscr{F}}{\varepsilon(F_k)}$. Now, let $G = (V,E,\omega)$ be a
graph. Define $G_\alpha$ to be the graph formed by multiplying the
similarity measure $\omega$ on $G$ by $\alpha$, i.e., $G_\alpha =
(V,E,\alpha \omega)$. Denote by $\mathscr{F}_{G_\alpha}$ the
collection of all forests on $G_\alpha$. Furthermore, denote by
$\mathscr{F}_{G_\alpha}^{uv} \subset \mathscr{F}_{G_\alpha}$ the
collection of forests on $G_\alpha$ where $u$ and $v$ belong to the
same tree rooted at $u$. We then have
\begin{equation}
  \label{eq:48}
  \mathbf{Q}_{\alpha}(u,v) = \frac{\varepsilon(\mathscr{F}_{G_{\alpha}}^{uv})}{\varepsilon(\mathscr{F}_{G_{\alpha}})}
\end{equation}
A proof of Eq.~\eqref{eq:48} using the all minors matrix tree theorem
\citep{chaiken82,moon94:_some_deter_expan_and_matrix_tree_theor} is in
\citep{chebotarev02:_fores_laplac}. \\
\\
%
\noindent
Chebotarev and Shamis also note a relationship between forest metrics
and resistance distances in
\citep{chebotarev02:_fores_metric_for_graph_vertic}. Let
$G_{\alpha}^{*} = (V(G_{\alpha}^{*}),E(G_{\alpha}^{*}),\omega')$ be the
graph formed by augmenting to $G_\alpha$ a vertex $v_*$ as follows.
\begin{itemize}
\item $V(G_{\alpha}^{*}) = V(G_{\alpha}) \cup \{v_*\}$, $v_* \not \in
  V(G_{\alpha})$.
\item $E(G_{\alpha}^{*}) = E(G_{\alpha}) \cup \{ \{v, v_*\} \colon
  v \in V(G_{\alpha}) \}$
\item $\omega'(e) = \omega(e)$ if $e \in E(G_{\alpha})$ and $\omega'(e) = 1$ if $e \in \{ \{v,v_*\}
  \colon v \in V(G_{\alpha}) \}$. 
\end{itemize}
The relationship between forest metrics and resistance distances is
then given by the following proposition.
\begin{proposition}
  \label{prop:9}
  Let $G = (V,E,\omega)$ and $G_{\alpha}^{*}$ be as defined
  above. Then
  \begin{equation}
    \label{eq:54}
    r_{G_{\alpha}^{*}}(u,v) = \eta_{\alpha}(u,v)
  \end{equation}
  where $u$ and $v$ are vertices of $G$. 
\end{proposition}
\begin{proof}\citep{chebotarev02:_fores_metric_for_graph_vertic}
  Let $n = |V(G)|$ be the number of vertices in $G$. Let
  $\mathbf{Q}_{\alpha}^{0}$ be the matrix obtained from
  $\mathbf{Q}_{\alpha}$ by the addition of the zero row and zero
  column corresponding to the vertex $v_*$.
  \begin{equation}
    \label{eq:58}
    \mathbf{Q}_{\alpha}^{0}  =  \left[ \begin{array}{c|c}
        0 & \bm{0} \\ \hline
        \bm{0}^{T} & (\mathbf{I} + \alpha \mathbf{L})^{-1}
        \end{array} \right]
  \end{equation}
  Let $\mathbf{L}_{G_{\alpha}^{*}}$ be the Laplacian matrix of $G_{\alpha}^{*}$, i.e., 
  \begin{equation}
    \label{eq:57}
    \mathbf{L}_{G_{\alpha}^{*}} = \left[ \begin{array}{c|c}
        n & \bm{-1} \\ \hline
        \bm{-1}^{T} & (\mathbf{I} + \alpha \mathbf{L})
        \end{array} \right]
  \end{equation}
  Then $\mathbf{L}_{G_{\alpha}^{*}}\mathbf{Q}_{\alpha}^{0}\mathbf{L}_{G_{\alpha}^{*}} =
  \mathbf{L}_{G_{\alpha}^{*}}$. Thus $\mathbf{Q}_{\alpha}^{0}$ is a generalized
  inverse of $\mathbf{L}_{G_{\alpha}^{*}}$. $\mathbf{Q}_{\alpha}^{0}$ can then be
  written as 
  \begin{equation}
    \label{eq:59}
    \mathbf{Q}_{\alpha}^{0} = \mathbf{L}_{G_{\alpha}^{*}}^{\dagger} +
    \bm{a}\bm{1}^{T} + \bm{1}^{T}\bm{b}
  \end{equation}
  for some vectors $\bm{a}$,$\bm{b}$ and
  $\mathbf{L}_{G_{\alpha}^{*}}^{\dagger}$, the Moore-Penrose pseudo
  inverse of $\mathbf{L}_{G_{\alpha}^{*}}$. Since
  $\kappa(\bm{a}\bm{1}^{T}) = \mathbf{0} = \kappa(\bm{1}^{T}\bm{b})$
  (see Proposition \ref{prop:16}), we have
  \begin{equation}
    \label{eq:60}
    \kappa(\mathbf{Q}_{\alpha}^{0}) = \kappa(\mathbf{L}_{G_{\alpha}^{*}}^{\dagger})
  \end{equation}
  Now, $\kappa(\mathbf{Q}_{\alpha}^{0})$ coincides with
  $\kappa(\mathbf{Q}_{\alpha})$ for the set of vertices corresponding
  to $V(G)$, and Eq.~\eqref{eq:54} is thus established.
\end{proof}

\section{$f(\mathbf{P} - \mathbf{Q})$ and graph metrics}
\label{sec:graph-metr-funct}
We recalled from \S ~\ref{sec:expect-comm-time} that the matrix
$\Delta_{\delta}$ of expected commute time is given by
$\Delta_{\delta} = \kappa((\mathbf{I} - \mathbf{P} -
\mathbf{Q})^{-1}\bm{\Pi}^{-1})$. We also recalled from \S
~\ref{sec:diffusion-distances} that the matrix $\Delta_{\rho_{t}^{2}}$
of squared diffusion distance is given by $\Delta_{\rho_{t}^{2}} =
\kappa(\mathbf{P}^{2t}\bm{\Pi}^{-1})$. Proposition ~\ref{prop:17}
stated that expected commute time with respect to the two-step random
walk is the sum of squared diffusion distances at time scale
$t=0,1,\dots$. \\
\\
\noindent
We know from \S \ref{sec:expect-comm-time} that
\begin{equation}
\label{eq:63}
  (\mathbf{I} - \mathbf{P} + \mathbf{Q})^{-1}\bm{\Pi}^{-1} =
\biggl[\mathbf{I} +
  \sum_{k=1}^{\infty}{(\mathbf{P}^{k} -
    \mathbf{Q})}\biggr]\bm{\Pi}^{-1} = \bm{\Pi}^{-1} +
  \sum_{k=1}^{\infty}{(\mathbf{P}^{k}\bm{\Pi}^{-1} - \mathbf{J})}
\end{equation}
From Proposition \ref{prop:16} we also know that $\kappa(\mathbf{X} -
\mathbf{J}) = \kappa(\mathbf{X})$. Thus
\begin{equation}
  \label{eq:62}
  \kappa((\mathbf{I} - \mathbf{P} + \mathbf{Q})^{-1}\bm{\Pi}^{-1}) = \kappa(\bm{\Pi}^{-1} +
  \sum_{k=1}^{\infty}{(\mathbf{P}^{k}\bm{\Pi}^{-1} - \mathbf{J})}) =
  \kappa(\bm{\Pi}^{-1} + \sum_{k=1}^{\infty}{(\mathbf{P}^{k}\bm{\Pi}^{-1})})
\end{equation}
should hold, except that the sum in the rightmost term in
Eq.~\eqref{eq:62} is not necessarilty convergent since
$\rho(\mathbf{P}) = 1$. Before we worry about this problem, let's
consider the sum in the rightmost term in Eq.~\eqref{eq:62} as is. This sum
say that the expected commute time $\delta(u,v)$ is the
$\kappa$ transform of terms are formed by taking into account the
probability of all the paths between the $u$ and $v$. This
interpretation is easy to understand and confirm that
expected commute time is a sensible notion of distances on graphs. The
interpretation of the sum in Eq. \eqref{eq:63} is harder.  However,
the sum in Eq.~\eqref{eq:63} is always convergent since
$\rho(\mathbf{P} - \mathbf{Q}) < 1$. We have thus arrived at a
situation where a matrix power series in $\mathbf{P} - \mathbf{Q}$ is
convergent and has a simple intepretation. Our aim in this section is
to extend the above observations into a more general result that will
allow us to have a general notion of distances on
graphs that is both well defined and also easily interpretable. \\ \\
\noindent
We first show that any matrix power series of the form
\begin{equation}
  \label{eq:64}
  \sum_{k=0}^{\infty}{c_k (\mathbf{P} - \mathbf{Q})^{k}} = 
c_0\mathbf{I} + c_1(\mathbf{P} - \mathbf{Q}) + c_2(\mathbf{P} -
  \mathbf{Q})^{2} + \cdots
\end{equation}
is convergent, as long as $\limsup_{n \rightarrow \infty} |c_k|^{1/k}
\leq 1$, This is a consequence of the following result \citep[\S
6.2]{horn94:_topic_in_matrix_analy}
\begin{theorem}
  \label{thm:3}
  Let $f(t)$ be a scalar-valued analytic function with a power series
  representation $f(t) = c_0 + c_1t + c_2 t^2 + \cdots$ that has radius
  of convergence $R > 0$. If $\mathbf{A} \in M_n$ is a $n \times n$
  square matrix and $\rho(A) < R$, then the matrix power series
  $f(\mathbf{A}) = c_0 \mathbf{I} + c_1 \mathbf{A} + c_2 \mathbf{A}^2
  + \cdots$ converges with respect to every norm on $M_n$. Furthermore,
  the sum is equal to the primary matrix function $f(\mathbf{A})$
  associated with the stem $f(t)$.
\end{theorem}
The radius of convergence of a power series is given by the {\em
  Cauchy-Hadamard} \/ formula \citep[\S V.3]{gamelin01:_compl_analy}
\begin{equation}
  \label{eq:65}
  R = \frac{1}{\limsup_{k \rightarrow \infty}{|c_k|^{1/k}}}
\end{equation}
The sum in Eq.~\eqref{eq:64} thus converges if $R \geq 1$, i.e.,
if $\limsup_{n \rightarrow \infty} |c_k|^{1/k} \leq 1$. 
\begin{proposition}
  \label{prop:13}
  Let $G = (V,E,\omega)$ be an undirected graph with similarity
  measure $\omega$. Let $\mathbf{P}$ be the transition matrix of
  $G$. Suppose that $\mathbf{P}$ is irreducible and aperiodic.  Let
  $f$ be a scalar-valued analytic function with radius of convergence
  $R \geq 1$. Assume also that $f$ is non-negative on the interval
  $(-1,1)$. Then $f((\mathbf{P} - \mathbf{Q}))\bm{\Pi}^{-1}$ is a well
  defined, positive semidefinite matrix. $\kappa(f(\mathbf{P} -
  \mathbf{Q})\bm{\Pi}^{-1})$ is then an EDM-2 matrix.
\end{proposition}
\begin{proof}
  Since $G$ is a directed graph, $\mathbf{P}$ is time-reversible. By
  Proposition ~\ref{prop:15}, $\bm{\Pi}\mathbf{P} =
  \mathbf{P}^{T}\bm{\Pi}$. Thus, the matrix
  $\bm{\Pi}^{1/2}\mathbf{P}\bm{\Pi}^{-1/2}$ is
  symmetric. Similarly, the matrix $\mathbf{N} =
  \bm{\Pi}^{1/2}(\mathbf{P} - \mathbf{Q})\bm{\Pi}^{-1/2}$ is
  symmetric. Now, $f(\mathbf{N})$ is well defined and furthermore
  \begin{equation}
    \label{eq:66}
    f(\mathbf{N}) = \bm{\Pi}^{1/2}f(\mathbf{P} - \mathbf{Q})\bm{\Pi}^{-1/2}
  \end{equation}
  Since $\mathbf{N}$ is symmetric, the spectrum 
  $\sigma(f(\mathbf{N}))$ of $f(\mathbf{N})$ is
  \begin{equation}
    \label{eq:67}
    \sigma(f(\mathbf{N})) = \{ f(\lambda) \colon \lambda \in
    \sigma(\mathbf{N}) \}
  \end{equation}
  Now, $f(\mathbf{P} - \mathbf{Q})$ is similar to $f(\mathbf{N})$ and
  thus $\sigma(f(\mathbf{P} - \mathbf{Q})) =
  \sigma(f(\mathbf{N}))$. However, $(\mathbf{P} - \mathbf{Q})$ is also
  similar to $\mathbf{N}$ and so $\sigma(\mathbf{N}) =
  \sigma(\mathbf{P} - \mathbf{Q}) \subset
  (-1,1)$. Therefore 
  \begin{equation}
    \label{eq:68}
    \sigma(f(\mathbf{P} - \mathbf{Q})) = \{ f(\lambda) \colon \lambda \in
    \sigma(\mathbf{P} - \mathbf{Q})\} \subset \{ f(\lambda) \colon
    \lambda \in (-1,1) \}
  \end{equation}
  Since $f$ is non-negative on $(-1,1)$, $\sigma(f(\mathbf{P} -
  \mathbf{Q})) \subset \mathbb{R}^{\geq 0}$. Thus $f(\mathbf{N})$ is
  positive semidefinite. Now, $f(\mathbf{P} -
  \mathbf{Q})\bm{\Pi}^{-1}$ can be written as
  \begin{equation}
    \label{eq:69}
f(\mathbf{P} - \mathbf{Q})\bm{\Pi}^{-1} = \bm{\Pi}^{-1/2}
f(\mathbf{N}) \bm{\Pi}^{-1/2}
  \end{equation}
and so $f(\mathbf{P} - \mathbf{Q})\bm{\Pi}^{-1} \succeq
0$, as desired. 
\end{proof}
As consequences of Proposition ~\ref{prop:13}, the following notion of
distances on graphs are all well-defined. Furthermore, the resulting
distance matrix are all EDM-2 matrix.
\begin{enumerate}
\item $\Delta_1 = \kappa(\mathbf{X}_1\bm{\Pi}^{-1})$ where
  $\mathbf{X}_1 = \sum_{k=1}^{\infty}{\tfrac{1}{k}(\mathbf{P}^{2k} -
    \mathbf{Q})}$. $\mathbf{X}_1$ is then the primary matrix function
  $\log{(\mathbf{I} - \mathbf{P^2} + \mathbf{Q})^{-1}}$. The terms of
  $\mathbf{X}_1$ are formed by taking into account all even paths
  between nodes, where each of the paths is scaled by its length, with
  longer paths being discouraged.
\item $\Delta_{2} = \kappa(\mathbf{X}_2\bm{\Pi}^{-1})$ where
  $\mathbf{X}_2 = \sum_{k=0}^{\infty}{(k+1)(\mathbf{P} -
    \mathbf{Q})^{k}}$. $\mathbf{X}_2$ is then the primary matrix function
  $(\mathbf{I} - \mathbf{P} + \mathbf{Q})^{-2}$. The terms of
  $\mathbf{X}_2$ are formed by taking into account all paths
  between nodes, where each of the paths is again scaled by its
  length, with longer paths being encouraged.
\item $\Delta_{3} = \kappa(\exp(\mathbf{P})\bm{\Pi}^{-1})$. This is
  similar to the diffusion kernels of \citet{kondor02:_diffus} where
  they construct a family of kernels of the form $\exp(\beta
  \mathbf{L})$ with $\mathbf{L}$ being the combinatorial Laplacian.
\end{enumerate}
%
%
If $f$ satisfy the conditions in Proposition \ref{prop:13}, then
$f(\mathbf{P} - \mathbf{Q})\bm{\Pi}^{-1}$ defines a kernel
matrix. This is somewhat similar to the construction of graph kernels
by spectral transforms in
\citet{zhu05:_semi,chapelle03:_clust_kernel_semi_super_learn,smola03:_kernel}.
Apart from the fact that the two approaches gave different set of
kernel matrices, there is a key difference between the two
approaches. A kernel matrix constructed by Proposition \ref{prop:13}
has entries that can be interpreted while only a small subset of the
kernel matrix that can be constructed through spectral transform will
have entries that are easily interpretable. \\ \\
\noindent
Let $G = (V,E,\omega)$ be a graph and $\mathbf{P}$ be a transition
matrix on $G$. We could consider a random walk on $G$ with transition
probabilities $\mathbf{P}_{\alpha} = (1 - \alpha)\mathbf{I} +
\alpha \mathbf{P}$ for some $\alpha \in (0,1]$. For example, lazy
random walks on a graph $G$ is obtained by setting $\alpha = 1/2$. If
$\mathbf{P}$ is irreducible and aperiodic, then
$\mathbf{P}_{\alpha}$ is also irreducible and aperiodic for
all $\alpha \in (0,1]$. Furthermore, the stationary distribution
$\pi_{\alpha}$ of $\mathbf{P}_{\alpha}$ and $\pi$ of $\mathbf{P}$ also
coincides. Proposition \ref{prop:13} can be slightly 
generalized to random walks with transition probabilities
$\mathbf{P}_{\alpha}$ as follows
\begin{proposition}
  \label{prop:24}
  Let $G = (V,E,\omega)$ be an undirected graph with similarity
  measure $\omega$. Let $\mathbf{P}$ be the transition matrix of
  $G$. Suppose that $\mathbf{P}$ is irreducible and aperiodic.  Let
  $f$ be a scalar-valued analytic function with radius of convergence
  $R \geq 1$. Let $\alpha \in (0,1]$ be fixed. Suppose that $f$ is
  non-negative on the interval $(1 - 2\alpha,1)$. Then $f((\mathbf{P}_{\alpha}
  - \mathbf{Q}))\bm{\Pi}^{-1}$ is a well defined, positive
  semidefinite matrix. $\kappa(f(\mathbf{P}_{\alpha} -
  \mathbf{Q})\bm{\Pi}^{-1})$ is then an EDM-2 matrix.
\end{proposition}
\begin{proof}
  The proof is almost identical to the proof of Proposition \ref{prop:13}. We
  only need to verify that the eigenvalues of $\mathbf{P}_{\alpha} - \mathbf{Q}$
  lie in the interval $(1 - 2\alpha, 1)$ and this is trivial since
  the eigenvalues of $\mathbf{P}_{\alpha} \in 1 - \alpha +
  \alpha(-1,1] = (1 - 2\alpha, 1]$ and thus the eigenvalues of
  $\mathbf{P}_{\alpha} - \mathbf{Q}$ lie in $(1 - 2\alpha, 1)$, as desired.
\end{proof}

%%% Local Variables: 
%%% mode: latex
%%% TeX-master: "dissertation"
%%% End: 

  We use the following characterization of positive definite
  matrix. Let $\mathbf{A}$ be invertible. Then $\mathbf{A} + \mathbf{A}^{T}$ is positive definite if and only if
  $\mathbf{A}^{-1} + (\mathbf{A}^{-1})^{T}$ is positive definite
  \cite{horn94:_topic_in_matrix_analy}. Since $\mathbf{Z}\bm{\Pi}^{-1}$ is
  invertible,
  \begin{equation}
    \label{eq:34}
    \begin{split}
    \mathbf{Z}\bm{\Pi}^{-1} + \bm{\Pi}^{-1}\mathbf{Z}^{T} \succ 0
    & \leftrightarrow \bm{\Pi}(\mathbf{I} - \mathbf{P} + \mathbf{Q}) + (\mathbf{I} -
    \mathbf{P}^{T} + \mathbf{Q}^{T})\bm{\Pi} \succ 0 \\
    & \leftrightarrow \bm{\Pi}(\mathbf{I} - \mathbf{P}) + (\mathbf{I} -
    \mathbf{P}^{T})\bm{\Pi} + 2 \pi \pi^{T} \succ 0 \\
    & \leftrightarrow \bm{\Pi}(\mathbf{I} - \frac{\mathbf{P} + \mathbf{P}_{*}}{2})
    + 2 \pi \pi^{T} \succ 0
  \end{split}      
  \end{equation}
  where $\mathbf{P}_{*}$ is the time-reverseral of $\mathbf{P}$. Now,
  $\bm{\Pi}(\mathbf{I} - \frac{\mathbf{P} + \mathbf{P}_{*}}{2})$ is
  symmetric. Furthermore, $\frac{\mathbf{P} + \mathbf{P}_{*}}{2}$ is a
  stochastic matrix and thus $\bm{\Pi}(\mathbf{I} - \frac{\mathbf{P} +
    \mathbf{P}_{*}}{2})$ is diagonally dominant. By Ger\u{s}gorin's
  circle theorem \cite{gersgorin31:_uber_abgren_eigen_matrix},
  $\bm{\Pi}(\mathbf{I} - \frac{\mathbf{P} + \mathbf{P}_{*}}{2})$ is
  positive semidefinite. Thus, $\mathbf{Z}\bm{\Pi}^{-1} +
  \bm{\Pi}^{-1}\mathbf{Z}^{T} \succ 0$, and $\Delta_{\delta}$ is an
  EDM-2 matrix.
%%% Local Variables: 
%%% mode: latex
%%% TeX-master: "dissertation"
%%% End: 

\chapter{Graph metrics and dimension reduction}
\label{cha:graph-metr-dimens}
We have seen in \S~\ref{cha:dist-undir-graphs} and
\S~\ref{cha:dist-direct-graphs} several notion of graph metrics. As we
have mentioned previously, several manifold learning algorithms can be
viewed as embedding a graph using some proximities measure on the
graph. The aim of this chapter is to expound on this point of
view. For the case where the graphs are undirected, we will see that
there might exists different plausible embeddings of the same graph
metrics. For example, one can embed expected commute time on
undirected graphs either by classical MDS, or by using the system of
eigenvalues and eigenvectors of the probability transition matrix.
The situation is slightly different for the case of directed graphs,
where classical MDS seems to be the most natural approach.  This lead
us to propose the view that the main difference between Isomap,
Laplacian eigenmaps, and diffusion maps is in the choice of
proximities measure between the vertices of the underlying graphs.
\section{Embedding by classical MDS}
\label{sec:embedd-class-mds}
Let $\Delta = \kappa(f(\mathbf{P} - \mathbf{Q})\bm{\Pi}^{-1})$ be a
distance matrix. We assume that $f$ satisfies the conditions in
Proposition~\ref{prop:13} so that $\Delta$ is EDM-2. The most
straightforward embedding of $\Delta$ is by classical MDS (see
\S~\ref{sec:classical-mds}). Specifically, let $\mathbf{B} =
\tau(\Delta)$ be the doubly centered inner product matrix formed from
$\Delta$. Classical MDS computes the eigendecomposition of
$\mathbf{B}$ and embeds $\Delta$ into $\mathbb{R}^{d}$ by using the
$d$ largest eigenvalues and eigenvectors of $\mathbf{B}$. By a result
in \cite{eckart36:_approx}, the resulting embedding is the
best rank-$d$ approximation to the correct configuration. \\ \\
\noindent
As an example, consider the problem of embedding a graph $G$ using
expected commute time and CMDS\@. Let $\Delta_\delta$ be the
matrix of expected commute time between the vertices of $G$. From
\S~\ref{sec:expect-comm-time}, $\Delta_{\delta}$ can be written as
\begin{equation*}
  \Delta_{\delta} = \kappa(\mathbf{Z}\bm{\Pi}^{-1}) = \mathrm{Vol}(G)
  \kappa(\mathbf{L}^{\dagger})
\end{equation*}
Because $\mathbf{L}$ is doubly centered, $\mathbf{L}^{\dagger}$ is
also doubly centered and so $\tau(\Delta_{\delta}) =
\tau(\kappa(\mathbf{L}^{\dagger})) = \mathbf{L}^{\dagger}$ by
Proposition~\ref{prop:16}. If $\lambda_1 \geq \lambda_2 \geq \dots \geq
\lambda_N$ are the eigenvalues of $\mathbf{L}^{\dagger}$ and
$\bm{\nu}_1, \bm{\nu}_2, \dots, \bm{\nu}_N$ are the corresponding
eigenvectors, then the embedding of vertex $v_i \in V$ into
$\mathbb{R}^{d}$ using expected commute time and classical MDS is
\begin{equation}
  \label{eq:33}
  \sqrt{\mathrm{Vol}(G)} 
\Bigl(\sqrt{\lambda}_1 \bm{\nu}_1(i), \dots, \sqrt{\lambda}_d \bm{\nu}_d(i) \Bigr)
\end{equation}
Because the eigenvectors of $\mathbf{L}^{\dagger}$ are also the
eigenvalues of $\mathbf{L}$, and the eigenvalues $\lambda_i$ of  
$\mathbf{L}^{\dagger}$ and $\mu_i$ of $\mathbf{L}$ are related by
\begin{equation}
  \label{eq:53}
  \lambda_i = \begin{cases}
    1/\mu_i & \text{if $\mu_i \not = 0$} \\
    0 & \text{if $\mu_i = 0$}
    \end{cases}
\end{equation}
the embedding for $v_i \in V$ can be written using the eigenvalues and
eigenvectors of $\mathbf{L}$. The above embedding of $G$ using the
eigenvalues and eigenvectors of $\mathbf{L}$ had appeared in the
literature under the guise of spectral clustering.
\citep{yen07:_graph,luxburg07:_tutor_spect_clust}.
\section{Embedding by eigensystem of P}
\label{sec:embedd-eigensyst-p}
Let $\Delta = \kappa(f(\mathbf{P} - \mathbf{Q})\bm{\Pi}^{-1})$ be
EDM-2. We have seen how to embed $\Delta$ using classical MDS in
\S~\ref{sec:embedd-class-mds}. We will now discuss the embedding of
$\Delta$ using the eigenvalue and eigenvectors of
$\mathbf{P}$. Because $\mathbf{P}$ is time-reversible, $\bm{\Pi}^{1/2}
\mathbf{P} \bm{\Pi}^{-1/2}$ is symmetric and so
$\bm{\Pi}^{1/2}(\mathbf{P} - \mathbf{Q})\bm{\Pi}^{-1/2}$ is also
symmetric. Let $\mathbf{U} \bm{\Sigma} \mathbf{U}^{T}$ be the
eigen-decomposition of $\bm{\Pi}^{1/2}(\mathbf{P} -
\mathbf{Q})\bm{\Pi}^{-1/2}$. Then $\mathbf{U} f(\bm{\Sigma})
\mathbf{U}^{T}$ is the eigen-decomposition of $\bm{\Pi}^{1/2}
f(\mathbf{P} - \mathbf{Q}) \bm{\Pi}^{-1/2}$. Because $f(\mathbf{P} -
\mathbf{Q})$ is similar to $\bm{\Pi}^{1/2}(\mathbf{P} -
\mathbf{Q})\bm{\Pi}^{-1/2}$, $\bm{\Pi}^{-1/2}\mathbf{U}$ is the matrix
of (right) eigenvectors of $f(\mathbf{P} - \mathbf{Q})$, which is also
the matrix of eigenvectors of $\mathbf{P} - \mathbf{Q}$. Furthermore,
because the eigenvectors of $\mathbf{P}$ are also the eigenvectors of
$\mathbf{Q}$, $\bm{\Pi}^{-1/2}\mathbf{U}$ is the matrix of
eigenvectors of $\mathbf{P}$. From the eigen-decomposition
$\mathbf{U}\bm{\Sigma}\mathbf{U}^{T} =
\bm{\Pi}^{1/2}f(\mathbf{P}-\mathbf{Q})\bm{\Pi}^{-1/2}$, we have
\begin{equation*}
    \kappa(f(\mathbf{P} - \mathbf{Q})\bm{\Pi}^{-1})=
    \kappa(\bm{\Pi}^{-1/2}\mathbf{U}f(\bm{\Sigma})\mathbf{U}^{T}\bm{\Pi}^{-1/2})
\end{equation*}
and so the embedding of a $v_i \in V$ into $\mathbb{R}^{d}$ using
$\Delta$ and the eigensystem of $\mathbf{P}$ is given by
\begin{equation}
  \label{eq:120}
   \frac{1}{\sqrt{\pi(i)}} (\sqrt{f(\mu_1)} \mathbf{u}_1(i),
    \dots, \sqrt{f(\mu_d)} \mathbf{u}_{d}(i))
\end{equation}
The embedding as given by Eq.~\eqref{eq:120} used the eigenvalues
$\mu_i \not= 1$ and eigenvectors $\mathbf{u}_i$ of $\mathbf{P}$,
sorted in decreasing order of magnitude of $\sqrt{f(\mu_i)}$. We
ignore $\mu_i = 1$ since the corresponding eigenvector $\mathbf{u}_i$
is constant. The eigenvectors $\mathbf{u}_i$ of $\mathbf{P}$ are not
orthonormal with respect to the normal inner product on Euclidean
space. However, they are orthonormal with respect to the inner product
$<\cdot,\cdot>_{\bm{\pi}}$ defined by
\begin{equation}
  \label{eq:121}
  <\mathbf{u},\mathbf{v}>_{\bm{\pi}} =
  \sum_{i}{\mathbf{u}(i)\mathbf{v}(i) \bm{\pi}(i)}
\end{equation}
As a first example, we consider the problem of embedding a graph $G$
using $\Delta_{\rho_t^2}$, the matrix of squared diffusion distances
at time $t$. $\Delta_{\rho_t^2} = \kappa((\mathbf{P} -
\mathbf{Q})^{2t} \bm{\Pi}^{-1})$, and so by Eq.~(\ref{eq:120}) with
$f(x) = x^{2t}$, the embedding of $v_i \in V$ into $\mathbb{R}^{d}$
using $\Delta_{\rho_t^2}$ and the eigensystem of $\mathbf{P}$ is  
\begin{equation}
  \label{eq:124}
  v_i \mapsto (\mu_{2}^{t} \mathbf{f}_{2}(i), \mu_{3}^{t}
  \mathbf{f}_{3}(i), \dots, \mu_d{d+1}^{t} \mathbf{f}_{d+1}(i))
\end{equation}
where $1 > |\mu_2| \geq |\mu_3| \geq \dots$ and
$\mathbf{u}_2, \mathbf{u}_3, \dots$ are the eigenvalues and
corresponding eigenvectors of $\mathbf{P}$ (the eigenvalue and
eigenvector pair corresponding to $\mu_i = 1$ is ignored). This is the
definition of diffusion maps as given by
\citet{coifman06:_diffus_maps}.  As another example, we consider the
problems of embedding a graph $G$ using $\Delta_\delta$, the matrix of
expected commute time.  $\Delta_\delta = \kappa((\mathbf{I} -
\mathbf{P} + \mathbf{Q})^{-1}\bm{\Pi}^{-1})$, and so by
Eq.\eqref{eq:120} with $f(x) = 1/(1-x)$, the embedding of $v_i \in V$
into $\mathbb{R}^{d}$ using $\Delta_\delta$ and the eigensystem of
$\mathbf{P}$ is
\begin{equation}
  \label{eq:122}
   \frac{1}{\sqrt{\pi(i)}} \Bigl(\frac{1}{\sqrt{1 - \mu_1}} \mathbf{u}_1(i),
    \dots, \frac{1}{\sqrt{1 - \mu_d}} \mathbf{u}_{d}(i)\Bigr)
\end{equation}
The embedding as given by Eq.~\eqref{eq:122} is also a variation of
Laplacian eigenmaps. \\ \\
\noindent
An observation can be made about the embedding of $\Delta_{\delta}$
using the eigensystem of $\mathbf{P}$ as compared to the eigensystem
of the normalized Laplacian $\bm{\mathcal{L}}$. Because $\mathbf{I} -
\mathbf{P} = \bm{\Pi}^{-1/2}\bm{\mathcal{L}}\bm{\Pi}^{1/2}$, i.e.,
$\mathbf{I} - \mathbf{P}$ and $\bm{\mathcal{L}}$ are similar, hence,
if $\mathbf{f}$ is an eigenvector of $\mathbf{P}$ with eigenvalue
$\lambda$ then $\bm{\Pi}^{1/2}\mathbf{f}$ is an eigenvector of the
normalized Laplacian $\bm{\mathcal{L}}$ with eigenvalue $1 -
\lambda$. We can thus also view Eq.~\eqref{eq:122} as embedding of
$\Delta_\delta$ using the eigenvalues and \emph{scaled} eigenvectors
of $\bm{\mathcal{L}}$. Note that this is not equivalent to embedding
using the eigenvalues and eigenvectors of $\bm{\mathcal{L}}$, i.e.,
the embedding given by
\begin{equation}
  \label{eq:102}
  v_i \mapsto \Bigl( \frac{1}{\sqrt{\lambda_2}} \mathbf{g}_2(i),
  \frac{1}{\sqrt{\lambda_3}}\mathbf{g}_3(i), \dots, \frac{1}{\sqrt{\lambda_{d+1}}} \mathbf{g}_{d+1}(i) \Bigr)
\end{equation}
where $0 = \lambda_1 \leq \lambda_2 \leq \dots \leq \lambda_{n-1}$ and
$\bm{g}_1, \bm{g}_2, \dots, \bm{g}_{n-1}$ are the eigenvalues and
corresponding eigenvectors of $\bm{\mathcal{L}}$, is not an embedding
of $\Delta_{\delta}$ into $\mathbb{R}^{d}$. However, this embedding is
also shown to be useful in the context of spectral clustering. The
spectral clustering algorithm of \citet{ng02} embed the data points
using the eigenvectors of $\bm{\mathcal{L}}$ and the embedded data
points are then clustered using the K-means algorithm. \citet{ng02}
showed that, under some assumptions regarding the data points, such an
algorithm managed to find a meaningful clusters representation of the
data points.
%
\section{Comparing the embeddings}
\label{sec:comparing-embeddings}
\noindent
We have seen two different approaches to embedding $G$ via a Euclidean
distance matrix $\Delta = \kappa(f(\mathbf{P} -
\mathbf{Q})\bm{\Pi}^{-1})$. The first approach is by using classical
MDS and the second approach is by using the eigensystem of
$\mathbf{P}$.  Even though the two approaches embed the same $\Delta$,
they are not equivalent. The eigenvalues and eigenvectors of
$\tau(\Delta)$ is not related to the eigenvalues and eigenvectors of
$\mathbf{P}$. Furthermore, in constrast to the eigenvectors of
$\tau(\Delta)$, the eigenvectors of $\mathbf{P}$ are not orthogonal
with respect to the normal inner product on Euclidean space. Lastly,
the $d$-dimensional embedding of a $\Delta$ using classical MDS is the
best $d$-dimensional embedding with respect to the STRAIN criterion of
MDS, and thus it is expected that the resulting embedding explains the
variance of the data points better than the embedding using the
eigensystem of $\mathbf{P}$. \\ \\
\noindent
An interesting feature of the embedding of $\Delta =
\kappa(f(\mathbf{P} - \mathbf{Q})\bm{\Pi}^{-1}) $ using the
eigensystem of $\mathbf{P}$ is that the embeddings for different $f$
are intimately related. If $\Delta_1 = \kappa(f_1(\mathbf{P} -
\mathbf{Q})\bm{\Pi}^{-1})$ and $\Delta_2 = \kappa(f_2(\mathbf{P} -
\mathbf{Q})\bm{\Pi}^{-1})$, then the embedding for $\Delta_1$ and the
embedding for $\Delta_2$ only differs by the scaling factor
$f_1(\mu_i)$ for $\Delta_1$ and $f_2(\mu_i)$ for $\Delta_2$. Thus, if
$\bm{\xi}_i$ and $\bm{\zeta}_i$ are the embeddings of $v_i \in V$ into
$\mathbb{R}^{d}$ using $\Delta_1$ and $\Delta_2$, then there exists a
$d \times d$ diagonal matrix $\mathbf{T}$ such that
\begin{equation}
  \label{eq:123}
  \bm{\xi}_i = \mathbf{T} \bm{\zeta}_i, \quad \forall v_i \in V
\end{equation}
The embeddings $\bm{\xi}_i$ and $\bm{\zeta}_i$ are thus {\em
  coordinates rescaling}\/of one another. A special case of the above
observation is the following result on the relationship between
Laplacian eigenmaps \cite{belkin03:_laplac} and diffusion maps
\cite{coifman06:_diffus_maps}.
\begin{proposition}
  \label{prop:27}
  Let $G$ be a graph and $\mathbf{P}$ be the transition matrix on
  $G$. Suppose that $\mathbf{P}$ is irreducible and aperiodic. Let
  $\bm{\xi}_i$ be the embeddings of $v_i \in V$ using expected commute
  time on $G$ and the eigensystem of $\mathbf{P}$. Let $\bm{\zeta}_i$
  be the embeddings of $v_i \in V$ using diffusion maps. Then the two
  embeddings $\bm{\xi}_i$ and $\bm{\zeta}_i$ are anisotropic scaling
  of one another.
\end{proposition}
\section{Embeddings for directed graphs}
\label{sec:embedd-dist-direct}
We now turn to the problem of embedding a distance matrix $\Delta$,
constructed by considering random walks on some directed graphs
$G$. Consider, for example, the problem of embedding $\Delta_{\delta}$,
a matrix of expected commute time, where the underlying graph $G$ is
directed. We know from \S~\ref{sec:expect-comm-time-1} that
$\Delta_{\delta}$ is a Euclidean distance matrix, and so embedding
$\Delta_\delta$ using classical MDS is natural and works
well. However, the embedding of $\Delta_\delta$ using the eigensystem
of $\mathbf{P}$ is not straightforward. The eigenvalues and
eigenvectors of $\mathbf{P}$ could be complex-valued, and is not
embeddable into Euclidean space. When $G$ is an undirected graph we
know from \S~\ref{sec:embedd-class-mds} that the
embedding of $\Delta_{\delta}$ using classical MDS is equivalent to
embedding using the combinatorial Laplacian. This equivalence breaks
down for the case where $G$ is directed. \citet{chung05:_laplac_cheeg}
investigated the notion of graph Laplacian for directed graphs, with the
resulting combinatorial Laplacian $\mathbf{L}$ being defined as
\begin{equation}
  \label{eq:125}
  \mathbf{L} = \bm{\Pi} - \frac{\bm{\Pi}\mathbf{P} + \mathbf{P}^{T}\bm{\Pi}}{2}
\end{equation}
$\mathbf{L}$ as defined is positive semidefinite, however,
$\Delta_{\delta}$ is no longer the $\kappa$ transform of
$\mathbf{L}^{\dagger}$. The symmetrization done in
constructing $\mathbf{L}^{\dagger}$ is equivalent to defining 
expected commute time in terms of $(\mathbf{P} + \hat{\mathbf{P}})/2$,
i.e., the symmetrization is done at a much earlier stage compared to
the symmetrization done in constructing expected commute time on $G$.
The embedding of $\Delta_{\delta}$ through
$\mathbf{L}$ is therefore not straightforward. \\ \\
%
\noindent
The above observations extend to general $\Delta$ constructed by
random walks on directed graphs. We held the view that embedding by
classical MDS is the natural way to embed these kind of distance
matrix. Furthermore, one might want to use the embedding to train a
classifier. See, for example, the embeddings of the MNIST data set in
\S~\ref{sec:embedding-examples}. Out-of-sample extensions for MDS
exist and would be useful for this situation. 
%
\section{Embedding examples}
\label{sec:embedding-examples}
\begin{figure}[htbp]
  \begin{center}
    \subfigure[][]{
      \label{fig:mnist01_ect}
      \includegraphics[width=8cm]{graphics/mnist/mnist01_small.pdf}
    }
    \subfigure[][]{
      \label{fig:mnist01_pca}
      \includegraphics[width=8cm]{graphics/mnist/zero_one_pca.pdf}
    }
  \caption{Embedding of the digits 0 and 1 from the MNIST data
    set. \subref{fig:mnist01_ect} is the embedding obtained by
    classical MDS with $\Delta$ being the matrix of expected commute
    time. \subref{fig:mnist01_pca} is the embedding obtained by
    PCA where each data point is viewed as a $784$ dimensional vector.
    }
  \label{fig:mnist01}
  \end{center}
\end{figure}    

\begin{figure}[htbp]
  \begin{center}
    \subfigure[][]{
      \label{fig:mnist08_ect}
      \includegraphics[width=8cm]{graphics/mnist/mnist08_small.pdf}
    }
    \subfigure[][]{
      \label{fig:mnist08_pca}
      \includegraphics[width=8cm]{graphics/mnist/zero_eight_pca.pdf}
    }
  \caption{Embedding of the digits 0 and 8 from the MNIST data
    set. \subref{fig:mnist08_ect} is the embedding obtained by
    classical MDS with $\Delta$ being the matrix of expected commute
    time. \subref{fig:mnist08_pca} is the embedding obtained by
    PCA where each data point is viewed as a $784$ dimensional vector.
    }
  \label{fig:mnist08}
  \end{center}
\end{figure}    

\begin{figure}[htbp]
  \begin{center}
    \subfigure[][]{
      \label{fig:mnist39_ect}
      \includegraphics[width=8cm]{graphics/mnist/mnist39_small.pdf}
    }
    \subfigure[][]{
      \label{fig:mnist39_pca}
      \includegraphics[width=8cm]{graphics/mnist/three_nine_pca.pdf}
    }
  \caption{Embedding of the digits 3 and 9 from the MNIST data
    set. \subref{fig:mnist39_ect} is the embedding obtained by
    classical MDS with $\Delta$ being the matrix of expected commute
    time. \subref{fig:mnist39_pca} is the embedding obtained by
    PCA where each data point is viewed as a $784$ dimensional vector.
    }
  \label{fig:mnist39}
  \end{center}
\end{figure}    

\begin{figure}[htbp]
  \begin{center}
    \subfigure[][]{
      \label{fig:mnist26_ect}
      \includegraphics[width=8cm]{graphics/mnist/mnist26_small.pdf}
    }
    \subfigure[][]{
      \label{fig:mnist26_pca}
      \includegraphics[width=8cm]{graphics/mnist/two_six_pca.pdf}
    }
  \caption{Embedding of the digits 2 and 6 from the MNIST data
    set. \subref{fig:mnist26_ect} is the embedding obtained by
    classical MDS with $\Delta$ being the matrix of expected commute
    time. \subref{fig:mnist26_pca} is the embedding obtained by
    PCA where each data point is viewed as a $784$ dimensional vector.
    }
  \label{fig:mnist26}
  \end{center}
\end{figure}    

\begin{figure}[htbp]
  \begin{center}
    \subfigure[][]{
      \label{fig:mnist45_ect}
      \includegraphics[width=8cm]{graphics/mnist/mnist45_small.pdf}
    }
    \subfigure[][]{
      \label{fig:mnist45_pca}
      \includegraphics[width=8cm]{graphics/mnist/four_five_pca.pdf}
    }
  \caption{Embedding of the digits 4 and 5 from the MNIST data
    set. \subref{fig:mnist45_ect} is the embedding obtained by
    classical MDS with $\Delta$ being the matrix of expected commute
    time. \subref{fig:mnist45_pca} is the embedding obtained by
    PCA where each data point is viewed as a $784$ dimensional vector.
    }
  \label{fig:mnist45}
  \end{center}
\end{figure}    

\begin{figure}[htbp]
  \begin{center}
    \subfigure[][]{
      \label{fig:mnist17_ect}
      \includegraphics[width=8cm]{graphics/mnist/mnist17_small.pdf}
    }
    \subfigure[][]{
      \label{fig:mnist17_pca}
      \includegraphics[width=8cm]{graphics/mnist/one_seven_pca.pdf}
    }
  \caption{Embedding of the digits 0 and 1 from the MNIST data
    set. \subref{fig:mnist17_ect} is the embedding obtained by
    classical MDS with $\Delta$ being the matrix of expected commute
    time. \subref{fig:mnist17_pca} is the embedding obtained by
    PCA where each data point is viewed as a $784$ dimensional vector.
    }
  \label{fig:mnist17}
  \end{center}
\end{figure}    
\begin{figure}[htbp]
  \begin{center}
    \includegraphics[width=8cm]{graphics/mnist/out_of_sample_mnist45.pdf}
    \caption{Out of sample embedding of the digits 4 and 5 from the MNIST data
    set. The out of sample embedding was obtained by computing the
    similarities between the new points to the existing sampled points
    and then embedding the new points through the out-of-sample extension of
    classical MDS as in \cite{bengio04:_out_lle_isomap_mds_eigen}.}  
  \label{fig:out_of_sample_mnist45}
  \end{center}
\end{figure} 
The MNIST data set \citep{lecun98:_gradien} is a data set for
characters recognition. There's a total of $60000$ labeled images of
the digits $0$ through $9$, with $50000$ of those being training
instances and the remaining $10000$ being testing instances. Each
image is $28 \times 28$ pixels, with each pixel having integer values
between $0$ and $255$. Figure~\ref{fig:mnist01} through
Figure~\ref{fig:mnist17} illustrate the embeddings of several pairs of
digits using expected commute time via classical MDS and the
embeddings using principal component analysis. For each digit, we
sampled at random $12$\% of the training instances to use in our
construction of the embeddings. The similarities between instances are
Gaussian similarities with $\sigma^2 = 5 \times 10^5$. This value of
$\sigma$ was chosen so that the similarities between all instances are
not concentrated around a small subinterval of $(0,1)$. We see from
the figures that the points belonging to different digits classes are
well separated by the embeddings. From the figures we see that,
compared to the embeddings using principal components, the embeddings
obtained by expected commute time via classical MDS have better
separation between points in different classes.  Furthermore, the use
of a linear classifier in the embeddings using expected commute time
via classical MDS will work well in discriminating the classes.  This
is illustrated in Figure~\ref{fig:out_of_sample_mnist45}. The circled
points are from Figure~\ref{fig:mnist45_ect} and represent the
original set of sampled digits. An additional 201 points were randomly
chosen from the testing set for the digits 4 and 5, with 110 points
being the digits 4 and the remaining 91 points being digits 5. The
points are then embedded as colored triangles in a similar manner to
the out-of-sample extension of
\citet{bengio04:_out_lle_isomap_mds_eigen}. The figure indicates that
a linear classifier trained on the sampled points will be a good
discriminator for the out-of-sample points. \\ \\
\begin{figure}[htbp]
  \centering
  \subfigure[][]{
    \label{fig:embed2-a}
    \includegraphics[width=55mm]{graphics/twosteps_data.pdf}
    }
    \hspace{8pt}
    \subfigure[][]{
      \label{fig:embed2-b}
      \includegraphics[width=55mm]{graphics/twosteps_diffusion1.pdf}
      }
      \subfigure[][]{
        \label{fig:embed2-c}
        \includegraphics[width=55mm]{graphics/twosteps_diffusion2.pdf}
        }
        \caption{Embedding of an artificial data set
          \subref{fig:embed2-a} using diffusion distances. The data
          points are colored from left to right along the $x$
          axis. \subref{fig:embed2-b} gave the embedding of the data
          point using diffusion distances with Gaussian similarities
          and $\sigma^{2} = 0.002$. The points in the embedding are
          colored using their original color in
          \subref{fig:embed2-a}. \subref{fig:embed2-c} gave the
          embedding of the data point using diffusion distances and
          Gaussian similarities, this time with $\sigma^{2} = 0.01$. }
  \label{fig:embed2}
\end{figure}
\noindent We mentioned previously in \S~\ref{sec:diffusion-distances} that
diffusion distances only take into account paths of even length. This
sometime leads to unexpected results. Consider for example the
contrived data set in Figure~\ref{fig:embed2-a}. Let $\mathbf{W}_1$ be
the matrix of Gaussian similarities between the data points with
$\sigma^{2} = 0.002$ and $\mathbf{P}_1$ be the resulting probability
transion matrix. $\mathbf{W}_1$ is constructed so that each row of
$\mathbf{P}_1$ have a small number of non-diagonal entries that are
significantly different from $0$. Let $\Delta_{1}$ be the matrix of
diffusion distance at time $t = 5$ with respect to
$\mathbf{P}_1$. Figure~\ref{fig:embed2-b} gives the two dimensional
embedding of $\Delta_{1}$ using the eigensystem of $\mathbf{P}_1$. Let
$\mathbf{W}_2$ be the matrix of Gaussian similarities between the data
points, this time with $\sigma^{2} = 0.01$, and $\mathbf{P}_2$ be the
resulting probability transition matrix. Each row of $\mathbf{P}_2$
now contains a sizable number of entries that are significantly
different from $0$. Let $\Delta_{2}$ be the matrix of diffusion
distance at time $t = 5$ with respect to
$\mathbf{P}_2$. Figure~\ref{fig:embed2-c} gives the two dimensional
embedding of $\Delta_{2}$ using the eigensystem of $\mathbf{P}_2$. In
Figure~\ref{fig:embed2-b}, we see that the (almost) sparseness of
$\mathbf{P}_1$ leads to data points that are adjacent in the ambient
space being embedded into different sides of the embedding. The
situation is much less severe in Figure~\ref{fig:embed2-c} in that
only the distances between some of the cyan and black data points in
the embedded space is smaller than the distances between some of the
cyan and green/red data points. We think that in general, because of
the two-step nature of diffusion distances, diffusion maps will work
better on graphs that are densely connected, in comparison with graphs
that are sparsely connected. 
% \section{Embedding expected commute time for undirected graphs}
% \label{sec:embedd-expect-comm}
% Let $G = (V,E,\omega)$ be an undirected graph. Suppose that
% $\Delta_{\delta}$ is the matrix of expected commute time between the
% vertices of $G$. We recall below the formula for $\Delta_{\delta}$ from \S
% \ref{sec:expect-comm-time} 
% \begin{equation}
%   \label{eq:101}
%   \Delta_{\delta} = \kappa(\mathbf{Z}\bm{\Pi}^{-1}) = \mathrm{Vol}(G)
%   \kappa(\mathbf{L}^{\dagger})
% \end{equation}
% where $\mathbf{Z} = (\mathbf{I} - \mathbf{P} + \mathbf{Q})^{-1}$ and
% $\mathbf{L}^{\dagger}$ is the Moore-Penrose pseudoinverse of the
% combinatorial Laplacian $\mathbf{L}$.  \\ \\
% %
% %
% \noindent 
% The matrix $\Delta_{\delta}$ of expected commute time on $G$ can be
% used to define embeddings of the vertices $V$ of $G$ into Euclidean
% space. The first and most straightforward embedding is by classical
% MDS using $\Delta_{\delta}$ as the squared dissimilarity matrix. Since
% $\mathbf{L}^{\dagger}$ is double centered, $\tau(\Delta_{\delta}) =
% \mathrm{Vol}(G)\tau(\kappa(\mathbf{L}^{\dagger})) = \mathrm{Vol}(G)
% \mathbf{L}^{\dagger}$. If $\lambda_1 \geq \lambda_2 \geq \dots \geq
% \lambda_N$ are the eigenvalues of $\mathbf{L}^{\dagger}$ and
% $\bm{\nu}_1, \bm{\nu}_2, \dots, \bm{\nu}_N$ are the corresponding
% eigenvectors, then the embedding of $v_i \in V$ into $\mathbb{R}^{d}$
% using classical MDS is identical to 
% \begin{equation}
%   \label{eq:98}
%   v_i \mapsto \sqrt{\mathrm{Vol}(G)} 
% \Bigl(\sqrt{\lambda}_1 \bm{\nu}_1(i), \sqrt{\lambda_2}
%   \bm{\nu}_2(i), \dots, \sqrt{\lambda}_d \bm{\nu}_d(i) \Bigr)
% \end{equation}
% Eq.~\eqref{eq:98} can also be written in terms of the eigenvalues of
% $\mathbf{L}$. The eigenvectors of $\mathbf{L}^{\dagger}$ and
% $\mathbf{L}$ coincide and the eigenvalues of
% $\mathbf{L}^{\dagger}$ can be mapped to the eigenvalues of
% $\mathbf{L}$ as
% \begin{equation}
%   \label{eq:99}
%   h(\lambda_i) = \begin{cases}
%     1/\lambda_i & \text{if $\lambda_i \not = 0$} \\
%     0 & \text{if $\lambda_i = 0$}
%     \end{cases}
% \end{equation} \\ \\
% %
% %
% \noindent Another embedding of $\Delta_\delta$ can be found by using
% the eigenvalues and eigenvectors of $\mathbf{P}$. We know that
% $\mathbf{P} = \bm{\Pi}^{-1}\mathbf{P}^{T}\bm{\Pi}$.
% $\bm{\Pi}^{1/2}\mathbf{P}\bm{\Pi}^{-1/2}$ is thus symmetric.
% $\bm{\Pi}^{1/2}(\mathbf{P} - \mathbf{Q})\bm{\Pi}^{-1/2}$ is therefore
% also symmetric. Let $\mathbf{U}\bm{\Sigma}\mathbf{U}^{T}$ be the
% spectral decomposition of $\bm{\Pi}^{1/2}(\mathbf{P} -
% \mathbf{Q})\bm{\Pi}^{-1/2}$. Then $\bm{\Pi}^{1/2}(\mathbf{I} -
% \mathbf{P} + \mathbf{Q})^{-1}\bm{\Pi}^{-1/2} = \mathbf{U}(\mathbf{I} -
% \bm{\Sigma})^{-1}\mathbf{U}^{T}$. We thus have
% \begin{equation}
%   \label{eq:105}
%   \begin{split}
%   \mathbf{Z}\bm{\Pi}^{-1} &=  (\mathbf{I} - \mathbf{P} +
%   \mathbf{Q})^{-1}\bm{\Pi}^{-1} \\ 
%   &= \bm{\Pi}^{-1/2} \mathbf{U}(\mathbf{I} -
%   \bm{\Sigma})^{-1}\mathbf{U}^{T}\bm{\Pi}^{-1/2} 
%   \end{split}
% \end{equation}
% Since $(\mathbf{P} - \mathbf{Q})$ is similar to
% $\bm{\Pi}^{1/2}(\mathbf{P} - \mathbf{Q})\bm{\Pi}^{-1/2}$, we have
% that $\bm{\Pi}^{-1/2}\mathbf{U}$ is the matrix of (right) eigenvectors of
% $\mathbf{P} - \mathbf{Q}$. Furthermore, the eigenvectors of
% $\mathbf{P}$ are also the eigenvectors of $\mathbf{P} - \mathbf{Q}$. 
% The embedding of $V$ into $\mathbb{R}^{d}$ using the eigenvalues and
% eigenvectors of $\mathbf{P}$ is then given as
% \begin{equation}
%   \label{eq:104}
%   v_i \mapsto \Bigl( \frac{1}{\sqrt{1 - \lambda_2}} \mathbf{f}_2(i),
%     \frac{1}{\sqrt{1 - \lambda_3}}\mathbf{f}_3(i), \dots, \frac{1}{\sqrt{1 -
%           \lambda_{d+1}}} \mathbf{f}_{d+1}(i) \Bigr)
% \end{equation}
% where $\lambda_1 = 1 \geq \lambda_2 \geq \dots \geq \lambda_N$ are the
% eigenvalues of $\mathbf{P}$. The embedding as given by
% Eq.~\eqref{eq:104} is therefore an anisotropic scaling of the
% Laplacian eigenmaps as given by Eq.~\eqref{eq:92} in \S
% \ref{sec:laplacian-eigenmaps}. Now, $\mathbf{f}$ is an eigenvector of
% $\mathbf{P}$ with eigenvalue $\lambda$ implies that
% $\bm{\Pi}^{1/2}\mathbf{f}$ is an eigenvector of the normalized
% Laplacian $\bm{\mathcal{L}}$ with eigenvalue $1 - \lambda$. We can
% thus also view Eq.~\eqref{eq:104} as embedding of $\Delta_\delta$
% using the eigenvalues and \emph{scaled} eigenvectors of
% $\bm{\mathcal{L}}$. Note that this is not equivalent to
% embedding using the eigenvalues and eigenvectors of
% $\bm{\mathcal{L}}$, i.e., the embedding given by
% \begin{equation}
%   \label{eq:102}
%   v_i \mapsto \Bigl( \frac{1}{\sqrt{\lambda_2}} \mathbf{g}_2(i),
%   \frac{1}{\sqrt{\lambda_3}}\mathbf{g}_3(i), \dots, \frac{1}{\sqrt{
%       \lambda_{d+1}}} \mathbf{g}_{d+1}(i) \Bigr)
% \end{equation}
% where $0 = \lambda_1 \leq \lambda_2 \leq \dots \leq \lambda_{n-1}$ and
% $\bm{g}_1, \bm{g}_2, \dots, \bm{g}_{n-1}$ are the eigenvalues and
% corresponding eigenvectors of $\bm{\mathcal{L}}$, is not an embedding
% of $\Delta_{\delta}$ into $\mathbb{R}^{d}$. However, this embedding is
% also shown to be useful in the context of spectral clustering. The
% spectral clustering algorithm of \citet{ng02} embed the data points
% using the eigenvectors of $\bm{\mathcal{L}}$ and the embedded data
% points are then clustered using the K-means algorithm. \citet{ng02}
% showed that, under some assumptions regarding the data points, such an
% algorithm managed to find a meaningful clusters representation of the
% data points.
% \\ \\
% %
% %
% \noindent
% We comment briefly on the embeddings as given by Eq.~\eqref{eq:98} and
% Eq.~\eqref{eq:104}. Eq.~\eqref{eq:98} embeds $\Delta_{\delta}$ using
% the eigenvalues and eigenvectors of $\mathbf{L}$ while
% Eq.~\eqref{eq:104} embeds using the eigenvalues and eigenvectors of
% $\mathbf{P}$. The two embeddings as given by Eq.~\eqref{eq:98} and
% Eq.~\eqref{eq:104} are not equivalent. The eigenvalues and
% eigenvectors of $\mathbf{L}$ is not related to the eigenvalues and
% eigenvectors of $\mathbf{P}$. Thus, the claim in \citet{saul06:_semis}
% that the Laplacian eigenmaps of Eq.~\eqref{eq:92} is given by MDS
% using expected commute times is inaccurate. Note that, in constrast to
% the eigenvectors of $\mathbf{L}$, the eigenvectors of $\mathbf{P}$ are
% not orthogonal. Furthermore, the $d$-dimensional embedding as given by
% Eq.~\eqref{eq:98} is the best $d$-dimensional embedding of
% $\Delta_{\delta}$ with respect to the STRAIN criterion
% (Eq.~\eqref{eq:87}) and thus it's expected that the embedding as given
% by Eq.~\eqref{eq:98} explains the variance of the data points better
% than the embedding as given by Eq.~\eqref{eq:104}. However, since the
% $d$-dimensional embedding as given by Eq.~\eqref{eq:98} is not
% necessarily the best $d$-dimensional embedding of $\Delta_{\delta}$
% with respect to the STRESS criterion, it's not guaranteed that the
% embedding as given by Eq.~\eqref{eq:98} is a better approximation to
% $\Delta_\delta$ than the embedding given by Eq.~\eqref{eq:104}.

% \subsection{The MNIST data set and embeddings}
% The MNIST data set \citep{lecun98:_gradien} is a data set for
% characters recognition. There's a total of $60000$ labeled images of
% the digits $0$ through $9$, with $50000$ of those being training
% instances and the remaining $10000$ being testing instances. Each
% image is $28 \times 28$ pixels, with each pixel having integer values
% between $0$ and $255$. Figures \ref{fig:mnist01} through
% \ref{fig:mnist17} illustrate the embeddings of several pairs of digits
% using expected commute time via classical MDS and the embeddings using
% principal component analysis. For each digit, we sampled at random
% $12$\% of the training instances to use in our construction of the
% embeddings. The similarities between instances are Gaussian
% similarities with $\sigma^2 = 5 \times 10^5$. This value of $\sigma$
% was chosen so that the similarities between all instances are not
% concentrated around a small subinterval of $(0,1)$. We see from the
% figures that the points belonging to different digits classes are well
% separated by the embeddings. From the figures we see that, compared to
% the embeddings using principal components, the embeddings obtained by
% expected commute time via classical MDS have better separation between
% points in different classes.  Furthermore, the use of a linear
% classifier in the embeddings using expected commute time via classical
% MDS will work well in discriminating the classes. This is illustrated
% in Figure \ref{fig:out_of_sample_mnist45}. The circled points are from
% Figure \ref{fig:mnist45} and represent the original set of sampled
% digits. An additional 201 points were randomly chosen from the testing
% set for the digits 4 and 5, with 110 points being the digits 4 and the
% remaining 91 points being digits 5. The points are then embedded as
% colored triangles in a similar manner to the out-of-sample extension
% of \citet{bengio04:_out_lle_isomap_mds_eigen}. We can see that the
% out-of-sample points are embedded in such a way that the linear
% classifier trained on the sampled points is a good discriminator for
% the out-of-sample points.
% \begin{figure}[htbp]
%   \begin{center}
%     \subfigure[][]{
%       \label{fig:mnist01_ect}
%       \includegraphics[width=8cm]{graphics/mnist/mnist01_small.pdf}
%     }
%     \subfigure[][]{
%       \label{fig:mnist01_pca}
%       \includegraphics[width=8cm]{graphics/mnist/zero_one_pca.pdf}
%     }
%   \caption{Embedding of the digits 0 and 1 from the MNIST data
%     set. \subref{fig:mnist01_ect} is the embedding obtained by
%     classical MDS with $\Delta$ being the matrix of expected commute
%     time. \subref{fig:mnist01_pca} is the embedding obtained by
%     PCA where each data point is viewed as a $784$ dimensional vector.
%     }
%   \label{fig:mnist01}
%   \end{center}
% \end{figure}    

% \begin{figure}[htbp]
%   \begin{center}
%     \subfigure[][]{
%       \label{fig:mnist08_ect}
%       \includegraphics[width=8cm]{graphics/mnist/mnist08_small.pdf}
%     }
%     \subfigure[][]{
%       \label{fig:mnist08_pca}
%       \includegraphics[width=8cm]{graphics/mnist/zero_eight_pca.pdf}
%     }
%   \caption{Embedding of the digits 0 and 8 from the MNIST data
%     set. \subref{fig:mnist08_ect} is the embedding obtained by
%     classical MDS with $\Delta$ being the matrix of expected commute
%     time. \subref{fig:mnist08_pca} is the embedding obtained by
%     PCA where each data point is viewed as a $784$ dimensional vector.
%     }
%   \label{fig:mnist08}
%   \end{center}
% \end{figure}    

% \begin{figure}[htbp]
%   \begin{center}
%     \subfigure[][]{
%       \label{fig:mnist39_ect}
%       \includegraphics[width=8cm]{graphics/mnist/mnist39_small.pdf}
%     }
%     \subfigure[][]{
%       \label{fig:mnist39_pca}
%       \includegraphics[width=8cm]{graphics/mnist/three_nine_pca.pdf}
%     }
%   \caption{Embedding of the digits 3 and 9 from the MNIST data
%     set. \subref{fig:mnist39_ect} is the embedding obtained by
%     classical MDS with $\Delta$ being the matrix of expected commute
%     time. \subref{fig:mnist39_pca} is the embedding obtained by
%     PCA where each data point is viewed as a $784$ dimensional vector.
%     }
%   \label{fig:mnist39}
%   \end{center}
% \end{figure}    

% \begin{figure}[htbp]
%   \begin{center}
%     \subfigure[][]{
%       \label{fig:mnist26_ect}
%       \includegraphics[width=8cm]{graphics/mnist/mnist26_small.pdf}
%     }
%     \subfigure[][]{
%       \label{fig:mnist26_pca}
%       \includegraphics[width=8cm]{graphics/mnist/two_six_pca.pdf}
%     }
%   \caption{Embedding of the digits 2 and 6 from the MNIST data
%     set. \subref{fig:mnist26_ect} is the embedding obtained by
%     classical MDS with $\Delta$ being the matrix of expected commute
%     time. \subref{fig:mnist26_pca} is the embedding obtained by
%     PCA where each data point is viewed as a $784$ dimensional vector.
%     }
%   \label{fig:mnist26}
%   \end{center}
% \end{figure}    

% \begin{figure}[htbp]
%   \begin{center}
%     \subfigure[][]{
%       \label{fig:mnist45_ect}
%       \includegraphics[width=8cm]{graphics/mnist/mnist45_small.pdf}
%     }
%     \subfigure[][]{
%       \label{fig:mnist45_pca}
%       \includegraphics[width=8cm]{graphics/mnist/four_five_pca.pdf}
%     }
%   \caption{Embedding of the digits 4 and 5 from the MNIST data
%     set. \subref{fig:mnist45_ect} is the embedding obtained by
%     classical MDS with $\Delta$ being the matrix of expected commute
%     time. \subref{fig:mnist45_pca} is the embedding obtained by
%     PCA where each data point is viewed as a $784$ dimensional vector.
%     }
%   \label{fig:mnist45}
%   \end{center}
% \end{figure}    

% \begin{figure}[htbp]
%   \begin{center}
%     \subfigure[][]{
%       \label{fig:mnist17_ect}
%       \includegraphics[width=8cm]{graphics/mnist/mnist17_small.pdf}
%     }
%     \subfigure[][]{
%       \label{fig:mnist17_pca}
%       \includegraphics[width=8cm]{graphics/mnist/one_seven_pca.pdf}
%     }
%   \caption{Embedding of the digits 0 and 1 from the MNIST data
%     set. \subref{fig:mnist17_ect} is the embedding obtained by
%     classical MDS with $\Delta$ being the matrix of expected commute
%     time. \subref{fig:mnist17_pca} is the embedding obtained by
%     PCA where each data point is viewed as a $784$ dimensional vector.
%     }
%   \label{fig:mnist17}
%   \end{center}
% \end{figure}    

% \begin{figure}[htbp]
%   \begin{center}
%     \includegraphics[width=8cm]{graphics/mnist/out_of_sample_mnist45.pdf}
%     \caption{Out of sample embedding of the digits 4 and 5 from the MNIST data
%     set. The out of sample embedding was obtained by computing the
%     similarities between the new points to the existing sampled points
%     and then embedding the new points through the out-of-sample extension of
%     classical MDS as in \cite{bengio04:_out_lle_isomap_mds_eigen}.}  
%   \label{fig:out_of_sample_mnist45}
%   \end{center}
% \end{figure}    
% \section{Embedding diffusion distances for undirected graphs}
% \label{sec:embedd-diff-dist}
% Let $G = (V,E,\omega)$ be an undirected graph. Let
% $\Delta_{\rho_{t}^{2}}$ be the matrix of squared diffusion distances
% between the vertices of $G$. From Proposition \ref{prop:12},
% $\Delta_{\rho_{t}^{2}} = \kappa(\mathbf{P}^{2t} \bm{\Pi}^{-1})$. The
% matrix $\Delta_{\rho_{t}^{2}}$ of diffusion distances on $G$ can then
% be used to define embeddings of the vertices $V$ of $G$ into Euclidean
% space. The first embedding is by classical MDS using
% $\Delta_{\rho_{t}^{2}}$. However, contrary to the case of expected
% commute time, the classical MDS embedding doesn't seem to correspond to
% eigenvalues and eigenvectors of either the Laplacian matrices or the
% probability transition matrix. \\ \\
% %
% %
% \noindent
% We can also embed $\Delta_{\rho_{t}^{2}}$ using the eigenvalues and
% eigenvectors of the probability transition matrix. This is the
% diffusion maps of \citet{coifman06:_diffus_maps}. Similar to our
% discussion of the embedding of expected commute time into Euclidean
% space in \S \ref{sec:embedd-expect-comm}, let
% $\mathbf{U}\bm{\Sigma}\mathbf{U}^{T}$ be the spectral decomposition of
% $\bm{\Pi}^{1/2}\mathbf{P}\bm{\Pi}^{-1/2}$. Then
% $\mathbf{P}^{2t}\bm{\Pi}^{-1} =
% \bm{\Pi}^{-1/2}\mathbf{U}\bm{\Sigma}^{2t}\mathbf{U}^{T}\bm{\Pi}^{-1/2}$
% and the diffusion maps of \citet{coifman06:_diffus_maps} is given by
% \begin{equation}
%   \label{eq:106}
%   v_i \mapsto (\lambda_{2}^{t} \mathbf{f}_{2}(i), \lambda_{3}^{t}
%   \mathbf{f}_{3}(i), \dots, \lambda_{d+1}^{t} \mathbf{f}_{d+1}(i))
% \end{equation}
% where $\lambda_1 = 1 \geq |\lambda_2| \geq |\lambda_3| \geq
% |\lambda_{N}|$ are the eigenvalues of $\mathbf{P}$ in non-increasing
% order of modulus and $\mathbf{f}_i$ are the corresponding
% eigenvectors. Comparing Eq.~\eqref{eq:106}, Eq.~\eqref{eq:104} and
% Eq.~\eqref{eq:92} one see that diffusion maps is an anisotropic
% scaling of Laplacian eigenmaps. The following proposition is a
% restatement of the above observations.
% \begin{proposition}
%   \label{prop:22}
%   Let $G = (V,E,\omega)$ be an undirected graph and $\mathbf{P}$ be
%   its transition matrix. $\Delta_{\rho_{t}^{2}}$ defines an embedding
%   of the vertices of $G$ into $\mathbb{R}^{d}$ by
%   \begin{equation*}
%     v_i \mapsto (\lambda_{2}^{t} \mathbf{f}_{2}(i), \lambda_{3}^{t}
%     \mathbf{f}_{3}(i), \dots, \lambda_{d+1}^{t} \mathbf{f}_{d+1}(i))
%   \end{equation*}
%   where $\lambda_1 = 1 \geq |\lambda_2| \geq \dots \geq |\lambda_N|$
%   are the eigenvalues of $\mathbf{P}$
%   and $\mathbf{f}_{i}$ are the corresponding eigenvectors. The above
%   embedding is an anistropic scaling of Laplacian eigenmaps. It's also an
%   anisotropic scaling of the embedding using the
%   expected commute time $\Delta_{\delta}$ of Eq.~\eqref{eq:104}.
% \end{proposition}
% \subfiglabelskip = 0pt
% \begin{figure}[htbp]
%   \centering
%   \subfigure[][]{
%     \label{fig:embed1-a}
%     \includegraphics[width=55mm]{graphics/wellsdata.pdf}
%     }
%     \hspace{8pt}
%     \subfigure[][]{
%       \label{fig:embed1-b}
%       \includegraphics[width=55mm]{graphics/resistance.pdf}
%       }
%       \subfigure[][]{
%         \label{fig:embed1-c}
%         \includegraphics[width=55mm]{graphics/wells1.pdf}
%         }
%       \subfigure[][]{
%         \label{fig:embed1-d}
%         \includegraphics[width=55mm]{graphics/wells10.pdf}
%         }
%         \caption{Embedding and clustering of an artificial data set
%           with three clusters \subref{fig:embed1-a} using diffusion
%           distances and expected commute time. \subref{fig:embed1-b}
%           gave the embedding of the data point using expected
%           commute. The data points are also colored according to the
%           clusters formed by hierarichal clustering using expected
%           commute time. In \subref{fig:embed1-c} and
%           \subref{fig:embed1-d}, the data points were embedded using
%           diffusion distance at time scale $t = 1$ and $t = 10$,
%           respectively, through Eq.~\eqref{eq:106}. Once again, the
%           data points are also colored according to the clusters
%           formed by hierarichal clustering.}
%   \label{fig:embed1}
% \end{figure}
% As an illustration of the above observation, we consider a data set
% like the one in Figure \ref{fig:embed1-a}. Figure \ref{fig:embed1-b}
% gives the embedding of the data points into two dimension using
% expected commute time. The embedding was done through the system of
% eigenvalues and eigenvectors of the probability transition matrix
% $\mathbf{P}$ as in Eq.~\eqref{eq:104}. The data points were also
% clustered by hierarichal clustering using expected commute time. We
% see that the two dimensional embedding is consistent with the
% hierarichal clustering in Figure \ref{fig:embed1-b}. Figure
% \ref{fig:embed1-c} and Figure \ref{fig:embed1-d} give the embeddings of
% the data points using diffusion distances at time $t = 1$ and $t =
% 10$, respectively. The embeddings were both done through the system of
% eigenvalues and eigenvectors of $\mathbf{P}$ as in
% Eq.~\eqref{eq:106}. The data points were also clustered by hierarichal
% clustering using diffusion distances. The clusters in Figure
% \ref{fig:embed1-d} seem more pronounced than those in Figure
% \ref{fig:embed1-c}. This is most likely due to the fact that at time
% scale $t = 1$, diffusion distances between the data points are more
% tightly concentrated. Note also that the two dimensional embeddings
% are very similar to each other. We suppose that this is because for a
% small number of dimensions, the scaling matrices between the
% embeddings are very close to being isotropic scaling matrices. The
% embeddings using classical MDS might not have such a
% phenomenon. \\ \\
% %
% %
% \begin{figure}[htbp]
%   \centering
%   \subfigure[][]{
%     \label{fig:embed2-a}
%     \includegraphics[width=55mm]{graphics/twosteps_data.pdf}
%     }
%     \hspace{8pt}
%     \subfigure[][]{
%       \label{fig:embed2-b}
%       \includegraphics[width=55mm]{graphics/twosteps_diffusion1.pdf}
%       }
%       \subfigure[][]{
%         \label{fig:embed2-c}
%         \includegraphics[width=55mm]{graphics/twosteps_diffusion2.pdf}
%         }
%         \caption{Embedding of an artificial data set
%           \subref{fig:embed2-a} using diffusion distances. The data
%           points are colored from left to right along the $x$
%           axis. \subref{fig:embed2-b} gave the embedding of the data
%           point using diffusion distances with Gaussian similarities
%           and $\sigma^{2} = 0.002$. The points in the embedding are
%           colored using their original color in
%           \subref{fig:embed2-a}. \subref{fig:embed2-c} gave the
%           embedding of the data point using diffusion distances and
%           Gaussian similarities, this time with $\sigma^{2} = 0.01$. }
%   \label{fig:embed2}
% \end{figure}
% We mentioned previously in \S \ref{sec:diffusion-distances} that
% diffusion distances only take into account paths of even length. This
% sometime leads to unexpected results. Consider for example the
% contrived data set in Figure \ref{fig:embed2-a}. Let $\mathbf{W}_1$ be
% the matrix of Gaussian similarities between the data points with
% $\sigma^{2} = 0.002$ (see Eq.~\eqref{eq:100}) and $\mathbf{P}_1$ be the
% resulting probability transion matrix. $\mathbf{W}_1$ is constructed
% so that each row of $\mathbf{P}_1$ have at most two non-diagonal
% entries that are significantly different from $0$. Let $\Delta_{1}$ be
% the matrix of diffusion distance at time $t = 5$ with $\mathbf{P}_1$
% as the transition matrix. Figure \ref{fig:embed2-b} gives the two
% dimensional embedding of $\Delta_{1}$. The embedding was done through
% the system of eigenvalues and eigenvectors of $\mathbf{P}_1$. Let
% $\mathbf{W}_2$ be the matrix of Gaussian similarities between the data
% points, this time with $\sigma^{2} = 0.01$, and $\mathbf{P}_2$ be the
% resulting probability transition matrix. Each row of $\mathbf{P}_2$
% now contains a sizable number of entries that are significantly
% different from $0$. Let $\Delta_{2}$ be the matrix of diffusion
% distance at time $t = 5$ with $\mathbf{P}_2$ as the transition
% matrix. Figure \ref{fig:embed2-c} gives the two dimensional embedding
% of $\Delta_{2}$. The embedding was also done using the system of
% eigenvalues and eigenvectors of $\mathbf{P}_2$. In Figure
% \ref{fig:embed2-b}, we see that the (almost) sparseness of
% $\mathbf{P}_1$ leads to data points that are adjacent in the ambient
% space being embedded into different sides of the embedding. The
% situation is much less severe in Figure \ref{fig:embed2-c}. However,
% since the data points are now embedded on a curve, the distances
% between some of the cyan and black data points in the embedded space
% is now smaller than the distances between some of the cyan and green/red
% data points. It's slightly amusing that sometime figures similar to
% Figure \ref{fig:embed2-c} are used as an illustration of the
% usefulness of non-linear dimensionality reduction. In that sense,
% diffusion maps had performed a non-linear transformation
% of the linear data in Figure \ref{fig:embed2-a}.

% \section{Embeddings of other graph metrics}
% \label{sec:embedd-other-graph}


%%% Local Variables: 
%%% mode: latex
%%% TeX-master: "dissertation.tex"
%%% End: 

\documentclass[11pt]{asaproc}
\usepackage{graphicx}
\usepackage[colorlinks=true,pagebackref,linkcolor=blue]{hyperref}
\usepackage[colon,sort&compress]{natbib}
\bibliographystyle{plainnat}
\usepackage{amsmath}
\usepackage{amssymb}
\usepackage{bm}

\title{Embedding Directed Proximity Data}
\author{Minh Tang \thanks{School of Informatics and Computing, Indiana
    University, Bloomington} \and Michael Trosset\thanks{Department of
    Statistics, Indiana University, Bloomington}}

\begin{document}
\maketitle
\begin{abstract}
Multidimensional scaling (MDS) constructs Euclidean configurations of
points from symmetric pairwise proximities, i.e., the edge weights of
an undirected graph. In some applications, however, proximity is
asymmetric, e.g., nearest neighbour graphs are directed. In such
cases, one might symmetrize the proximity matrix and apply traditional
MDS to the symmetrized proximities. Instead, we describe embedding
techniques that constructs representation of directed proximity data.
\begin{keywords}
Multidimensional scaling, directed proximity
\end{keywords}
\end{abstract}

\section{Introduction}
There are a variety of problems in diverse disciplines where the
observations of the relationship between a pair of objects are
asymmetric. These observations for a set of objects are usually called
asymmetric data or asymmetric structures. Consider for example the
case of brand switching among customers. \citet{desarbo84} looks at
the number of people who switched between various soft drinks. The
number of people who switched between two brands are not equal and the
data is thus asymmetric. Another example is the Morse code confusion
rate. In \citet{rothkopf57}, a number of individuals were asked
whether two Morse codes that were played in sequence are similar or
not. It was observed that the number of people who thought that 2 and
J are similar when the sequence 2J was played is more than when the
sequence J2 was played. It's the goal of asymmetric data analysis to
offer structured and systematic approaches to handling these data.

A large number of asymmetric data can be represented as a directed
graph. For example, given a set of dissimilarity between a set of
objects, a k-NN graph $G$ with vertices representing the objects is a
directed graph and the resulting adjacency matrix representation for
$G$ is an asymmetric data matrix. 

The structure of our paper is as follows. We review  existing
approaches to embedding asymmetric structures, namely 
three-way MDS and asymmetric MDS models, in \S
\ref{sec:three-way-mds}. \S
\ref{sec:indiv-scal-thro} describe our projected subspace algorithm for embedding
asymmetric structures. Our algorithm can be viewed as a hybrid of the
classes of three-way MDS and asymmetric MDS algorithms. \S
\ref{sec:from-simil-diss} discussed the problem of transforming 
similarities to dissimilarities on directed graphs. The
transformation is by a relaxed random walk on the directed graph. Some
examples illustrating the working of our approach are presented in \S
\ref{sec:examples}. We conclude the paper  with some discussion of the
utility and short-coming of our approach in \S \ref{sec:conclusions}. 

\section{Three-way MDS and asymmetric MDS models}
\label{sec:three-way-mds}
Given a set of $M$ dissimilarity matrices $\bm{\Delta}^{(1)}$,
$\bm{\Delta}^{(2)}$, through $\bm{\Delta}^{(M)}$, three-way MDS
algorithms attempt to find a common group space $\mathbf{G}$ and
individual transformation matrices $\mathbf{T}_k$ that minimize some kind
of loss function $L(\{\bm{\Delta}^{(i)}\}_{i=1}^{M}, \mathbf{G},
\{\mathbf{T}_i\}_{i=1}^{M})$. For example, the Stress loss function
\citep{kruskal64:_nonmet} can be written for the above problem as
\begin{equation}
  \label{eq:1}
  L(\mathbf{G}, \mathbf{T}_1, \mathbf{T}_{2}, \dots, \mathbf{T}_M) =
  \sum_{k = 1}^{M}\sum_{i < j} (\delta_{ij}^{(k)} -
  d_{ij}(\mathbf{G}\mathbf{T}_k))^2
\end{equation}
where $\delta_{ij}^{(k)}$ is the $ij$-th entry of $\bm{\Delta}^{(k)}$
and $d_{ij}(\mathbf{G}\mathbf{T}_k)$ is the Euclidean distance between
the $i$-th and $j$-th row of the matrix
$\mathbf{G}\mathbf{T}_k$. The analogue of Eq.~(\ref{eq:1}) using the
Strain loss function of classical MDS \citep{torgesen52:_multid,gower66:_some} is 
\begin{equation}
  \label{eq:2}
  L(\mathbf{G}, \mathbf{T}_1, \mathbf{T}_2, \dots, \mathbf{T}_M)
  = \sum_{k = 1}^{M}\| \mathbf{B}_{\bm{\Delta}^{(k)}} -
  \mathbf{G}\mathbf{T}_k \mathbf{T}_k' \mathbf{G}' \|^2 
\end{equation}
where $\mathbf{B}_{\bm{\Delta}^{(k)}}$ is the fallible inner-product
  matrix formed by taking the double centering of $\bm{\Delta}^{(k)}
  \ast \bm{\Delta}^{(k)}$, $\ast$ being the Hadamard product of
  matrices. \\ \\ 

  \noindent Numerous algorithms exist to find the group space $G$ and the
  transformation matrices $\mathbf{T}_1, \mathbf{T}_{2}, \dots,
  \mathbf{T}_M$ that minimize Eq. (\ref{eq:1}) and Eq. (\ref{eq:2})
  subjected to various constraints on the group space $\mathbf{G}$ and
  the transformation matrices $\mathbf{T}_k$. If we restricted the
  transformation matrices $\mathbf{T}_k$ to be diagonal matrices with
  positive diagonal entries, then the resulting model is known as
  INDSCAL \citep{carroll70:_analy_n_eckar_young}. \citet{carroll70:_analy_n_eckar_young}
  proposed the CANDECOMP algorithm to solve the INDSCAL problem with
  respect to the minimization of Eq. (\ref{eq:2}) while
  \citet{leeuw80:_multiv,leeuw09:_multid_scalin_using_major} solve the
  INDSCAL problem with respect to the minimization of Eq. (\ref{eq:1})
  using SMACOF, a majorization algorithm. If we allow each of the
  $\mathbf{T}_k$ to be an arbitrary transformation matrix, the
  resulting model is known as IDIOSCAL \citep{carroll74:_contem}. An
  analytic solution of the IDIOSCAL model when the
  $\mathbf{B}_{\bm{\Delta}^{(k)}}$ are inner product matrix is
  given by \citet{schonemann72} under the restriction that
  $\tfrac{1}{M}\sum_{k=1}^{M}{\mathbf{T}_k \mathbf{T}_k'} = \mathbf{I}$ where
  $\mathbf{I}$ is an appropriately-sized identity matrix. One of the
  main criticism against the IDIOSCAL model is its generality. The
  $\mathbf{T}_k$, being arbitrary transformation matrices, don't
  leads to results that are easily interpreted in general. \\ \\

  \noindent The main limitation of the INDSCAL model is that the most well-known
  algorithm that tried to solve Eq. (\ref{eq:2}) under the INDSCAL
  model has some undesirable features. The CANDECOMP algorithm
  \citep{carroll70:_analy_n_eckar_young} proceed by modifying
  Eq. (\ref{eq:2}) to 
  \begin{equation}
    \label{eq:2}
    L(\mathbf{G}, \mathbf{T}_1, \mathbf{T}_2, \dots, \mathbf{T}_M)
    = \sum_{k = 1}^{M}\| \mathbf{B}_{\bm{\Delta}^{(k)}} -
    \mathbf{G}\mathbf{T}_k \mathbf{T}_k' \mathbf{H}' \|^2 
  \end{equation}
  and alternatingly solving for $\mathbf{G}$, $\mathbf{T}_k$ and
  $\mathbf{H}$ in a least-square manner, keeping the other variables
  fixed. However, the least square solution of $\mathbf{T}_k$ for any
  iteration of the algorithm is not guaranteed to have non-negative
  diagonal entries. Secondly, the matrices $\mathbf{G}$
  and $\mathbf{H}$ are assumed to converge in some notion of
  equivalence. However, \citet{berge91:_some_candec_indsc} showed that
  this might not be true in general. There have been several proposed
  techniques to handle the positivity constraint of $\mathbf{T}_k$ but
  they employed an iterative approach in updating $\mathbf{T}_k$,
  for example by non-negative least squares
  \citet{berge93:_comput_indsc}. 
  
\section{Individual Scaling through Projected Subspace}
\label{sec:indiv-scal-thro}

\section{From Similarities to Dissimilarities on Directed Graphs}
\label{sec:from-simil-diss}

\section{Examples}
\label{sec:examples}

\section{Conclusions}
\label{sec:conclusions}

\bibliography{dissertation}

\end{document}

\appendix
\chapter{Mathematical Preliminaries}
\section{Graph Laplacians}
\label{sec:graph-laplacians}
We now introduce the concept of the Laplacian matrix of a graph. Our
exposition will be very superficial. For a more comprehensive account
of graph Laplacians, see
\citep{chung05:_laplac_cheeg,cvetkovic80:_spect_graph_theor_applic}.

Let $G = (V,E,\omega)$ be a simple, undirected graph with vertices set
$V$, edges set $E$ and similarity measure $\omega \colon E \mapsto
\mathbb{R}^{\geq 0}$. If $u$ and $v$ are vertices of $G$, we write $u
\sim v$ whenever $\{u,v\} \in E$. The degree of a vertex $v$ is
defined as $\deg(v) = \sum_{u \sim v}{\omega(\{u,v\})}$ and the volume
of $G$ is $\mathrm{Vol}(G) = \sum_{v \in V}{\deg(v)}$.  We denote by
$N$ the number of vertices of $G$. We define $D = (d_{ij})$ as the $N
\times N$ diagonal matrix with diagonal entries $d_{vv} = \deg(v)$.

\begin{definition}
  \label{def:1}
  Let $G = (V,E,\omega)$ be a simple, undirected graph with similarity
  measure $\omega$. The {\em combinatorial} Laplacian of $G$ is the
  matrix $L = L(G)$ with entries
  \begin{equation}
    \label{eq:1}
    L_{uv} = \begin{cases}
      - \omega(\{u,v\}) & \text{if $u \not = v$ and $u \sim v$} \\
      \deg(u) & \text{if $u = v$} \\
      0 & \text{otherwise}
    \end{cases}
  \end{equation}
  The {\em normalized} Laplacian of $G$ is the matrix $\mathcal{L} =
  \mathcal{L}(G)$ with entries
  \begin{equation}
    \label{eq:2}
    \mathcal{L}_{uv} = \begin{cases}
      - \tfrac{\omega(\{u,v\})}{\sqrt{\deg(u)}\sqrt{\deg(v)}} & \text{if $u \not = v$ and $u \sim v$} \\
      1 & \text{if $u = v$} \\
      0 & \text{otherwise}
    \end{cases}
  \end{equation}
\end{definition}
The following proposition lists some simple properties of the
combinatorial and normalized Laplacians. 
\begin{proposition}
  \label{prop:1}
  Let $G = (V,E,\omega)$ be a simple, undirected graph and $L$ and
  $\mathcal{L}$ be its combinatorial and normalized Laplacians,
  respectively. We have
  \begin{itemize}
  \item $L$ and $\mathcal{L}$ are symmetric, positive
    semi-definite matrices.
  \item $\mathcal{L} = D^{-1/2} L D^{-1/2}$
  \item The number of connected components of $G$ is equal to the
    number of zero eigenvalues of either $L$ or $\mathcal{L}$.
  \item The eigenvalues of $\mathcal{L}$ is at most $2$. 
  \end{itemize}
\end{proposition}
\section{Finite Markov Chain}
\label{sec:finite-markov-chain}
\begin{definition}
  \label{def:6}
  Let $\Omega$ be a finite or countably infinite set and
  $\mathbb{Z}^{*}$ be the set of non-negative integers. A sequence
  $\mathbf{X} = (X_n)_{n \in \mathbb{Z}^{*}}$ of random variables with values in
  $\Omega$ is a {\em Markov chain} if
  \begin{equation}
    \label{eq:8}
    \mathbb{P}[X_{n+1} = j \, | \, X_n = i, X_{n-1} = i_{n-1},
    \dots, X_0 = i_0] = \mathbb{P}[X_{n+1} = j \, | \, X_n = i] =
    p_{ij}
  \end{equation}
  for all $n \geq 0$ and all states $i_0, i_1, \dots, i_{n-1}, i,
  j$. The matrix $\mathbf{P}$, possibly infinite, with entries
  $\mathbf{P}(i,j) = p_{ij}$ is then termed the transition matrix of
  $(X_n)_{n \in \mathbb{Z}^*}$.
\end{definition}
Let $\mathbf{X} = (X_n)_{n \in \mathbb{Z}^*}$ be a Markov
chain. Denote by $p_{ij}^{(n)}$ the probability of going from state
$i$ to state $j$ in $n$ steps, i.e.,
\begin{equation}
  \label{eq:11}
  p_{ij}^{(n)} = \mathbb{P}[ X_{n + m } = j \, | \, X_m = i]
\end{equation}
for all $i, j \in \Omega$, and $m,n \in \mathbb{Z}^{*}$. Then
$p_{ij}^{(n)}$ satisfy the {\em Chapman-Kolmogorov equation}
\begin{equation}
  \label{eq:12}
  p_{ij}^{(m+n)} = \sum_{k \in \Omega}{p_{ik}^{(m)}p_{kj}^{(n)}}
\end{equation}
for all $m,n \in \mathbb{Z}^{*}$. Thus if $\mathbf{P}^{(n)}$ is the
matrix with entries $p_{ij}^{(n)}$, then $\mathbf{P}^{(m+n)} =
\mathbf{P}^{(m)}\mathbf{P}^{(n)}$. Since $\mathbf{P}^{(1)} =
\mathbf{P}$, we have
\begin{equation}
  \label{eq:13}
  \mathbf{P}^{(n)} = \mathbf{P}^{n}
\end{equation}
The behaviour of a Markov chain $\mathbf{X} = (X_n)_{n \in
  \mathbb{Z}^{*}}$ is thus completely specified by its transition
matrix $\mathbf{P}$. We can therefore view a Markov chain as being a
sequence of random variables generated by a transition matrix
$\mathbf{P}$. This view will be most helpful in the context of this
dissertation. However, since the transition matrix $\mathbf{P}$ only describes
the conditional probabilities, in order for us to compute the marginal
probabilities $\mathbb{P}[X_n = j]$, we need to specify an initial
distribution for $X_0$.

\begin{definition}
  \label{def:5}
  Let $\mathbf{X}$ be a Markov chain with state space
  $\Omega$. The initial distribution $\mu$ of $\mathbf{X}$ is a probability
  distribution on $\Omega$ such that 
  \begin{equation}
    \label{eq:14}
    \mu(i) = \mathbb{P}[X_0 = i]
  \end{equation}
  for all $i \in \Omega$. 
\end{definition}

\begin{definition}
  \label{def:7}
  Let $\mathbf{X}$ be a Markov chain with state space $\Omega$. Let
  $i$ and $j$ be elements of $\Omega$. $j$ is
  {\em accessible} from $i$, denoted as $i \rightarrow j$, if there
  exists a $n \in \mathbb{Z}^{*}$ such that $p_{ij}^{(n)} > 0$. If $i
  \rightarrow j$ and $j \rightarrow i$, then we say that $i$ and $j$
  {\em communicate}, and we write $i \leftrightarrow j$. A Markov chain is
  {\em irreducible} if $i \leftrightarrow j$ for any $i,j \in \Omega$.
\end{definition}
\begin{definition}
  \label{def:2}
  The stationary distribution $\pi$ of
  $\mathbf{X}$, if it exists, is a probability distribution on
  $\Omega$ such that
  \begin{equation}
    \label{eq:15}
    \pi(j) = \sum_{i \in \Omega}{\pi(i) p_{ij}}
  \end{equation}
  for any $j \in \Omega$. 
\end{definition}

\begin{proposition}
  \label{prop:3}
  If $\mathbf{X}$ is an irreducible Markov chain with state space
  $\Omega$, then there exists a unique stationary distribution $\pi$
  of $\mathbf{X}$, and that $\pi(i) > 0$ for all $i \in \Omega$. 
\end{proposition}

\begin{definition}
  \label{def:3}
  Let $\mathbf{X}$ be a Markov chain
  with transition matrix $\mathbf{P}$. Define 
  \begin{equation}
    \label{eq:5}
    \tau_i = \min\{ t \geq 0 \colon X_t = i \}, \qquad \tau_i^{+} \min
    \{ t \geq 1 \colon X_t = i \}
  \end{equation}
  The expected first passage time from $i$ to $j$, denoted by
  $\mathbb{E}_{i}[\tau_j]$, is defined as
  \begin{equation}
    \label{eq:6}
    \mathbb{E}_{i}[\tau_j] = \sum_{t = 0}^{\infty}{t \, \mathbb{P}(\tau_j =
      t \,|\, X_0 = i)}
  \end{equation}
  The expected first return time from $i$ to $i$, denoted by
  $\mathbb{E}_{i}[\tau_i^{+}]$, is defined as
  \begin{equation}
    \label{eq:7}
    \mathbb{E}_{i}[\tau_i^{+}] = \sum_{t = 1}^{\infty}{t \,
      \mathbb{P}(\tau_v^{+} = t \,|\, X_0 = i)}
  \end{equation}
  $\tau_i$ and $\tau_{i}^{+}$ as declared above are examples of {\em
    stopping times}. 
\end{definition}

\begin{proposition}
  \label{prop:2}
  Let $\mathbf{X}$ be an irreducible
  Markov chain with transition matrix $\mathbf{P}$ and stationary
  distribution $\pi$. We then have that
  \begin{equation}
    \label{eq:9}
    \mathbb{E}_{i}[\tau_i^{+}] = \frac{1}{\pi(i)}
  \end{equation}
\end{proposition}

\begin{definition}
  \label{def:9}
  Let $\mathbf{X}$ be an irreducible Markov chain with transition
  matrix $\mathbf{P}$ and stationary distribution
  $\pi$. $\hat{\mathbf{P}} = (\hat{p}_{ij})$ is said to be the {\em
    time reversal} of $\mathbf{P}$ if, for all pairs $i,j \in \Omega$,
  one has
  \begin{gather}
    \label{eq:16}
    \pi(i) p_{ij} = \pi(j) \hat{p}_{ji} \\
    \shortintertext{or, in other words}
    \label{eq:78}
    \hat{\mathbf{P}} = \bm{\Pi}^{-1} \mathbf{P}^{T} \bm{\Pi} 
  \end{gather}
  $\mathbf{P}$ is said to be {\em time-reversible} if
  $\hat{\mathbf{P}} = \mathbf{P}$.
\end{definition}
Now $\hat{\mathbf{P}}$ also defines a Markov chain
$\hat{\mathbf{X}}$. $\hat{\mathbf{X}}$ will be termed the
time-reversed Markov chain with respect to $\mathbf{X}$. $\pi$ is also
the stationary distribution of $\hat{\mathbf{P}}$ and that
\begin{equation}
  \label{eq:17}
  \mathbb{P}[X_n = j, \dots, X_0 = i] = \mathbb{P}[\hat{X}_0 = i,
  \dots, \hat{X}_n = j] 
\end{equation}
where the initial distribution of $X_0$ and $\hat{X}_0$ are both
identical to the stationary distribution $\pi$.
\section{Random walks on graphs}
\label{sec:random-walks-graphs}
Let $G = (V,E,\omega)$ be a simple, undirected graph. We define the transition
matrix $\mathbf{P}_G = (p_{uv})$ of a Markov chain with state space $V$ as follows
\begin{equation}
  \label{eq:20}
  p_{uv} = \begin{cases}
    \tfrac{\omega(\{u,v\})}{\deg(u)} & \text{if $u \sim v$} \\
    0 & \text{otherwise}
  \end{cases}
\end{equation}
We now note some properties of the Markov chain $\mathbf{X}$ generated
by $\mathbf{P}_G$
\begin{proposition}
  \label{prop:15}
  Let $G$ be an undirected graph and $\mathbf{P}$ be the transition
  matrix on $G$. Let $\mathbf{X}$ be the Markov chain generated by
  $\mathbf{P}$. Then 
\begin{itemize}
\item $\mathbf{X}$ is irreducible if and only if $G$.
\item If $\mathbf{X}$ is irreducible, $\pi(v) =
  \tfrac{\deg(v)}{\mathrm{Vol}(G)}$ for all $v \in V$.
\item $\mathbf{P}$ is time-reversible. Therefore, $\bm{\Pi}\mathbf{P} = \mathbf{P}^{T}\bm{\Pi}$.
\end{itemize}
\end{proposition}
%
We can also define the transition matrix $\mathbf{P}_G$ when $G$ is
directed. $\mathbf{P}_G$ will have entries
\begin{equation}
  \label{eq:18}
  p_{uv} = \begin{cases}
    \tfrac{\omega(e)}{\deg(u)} & \text{if $e = (u,v) \in E$} \\
    0 & \text{otherwise}
  \end{cases}
\end{equation}
If $G$ is directed, then $\mathbf{X}$ is irreducible if and only if
$G$ is strongly connected. However, $\mathbf{P}$ is in general not
time reversible and there's no explicit expression for the
stationary distribution $\pi$ of $\mathbf{P}$. \\ \\
%
%
\noindent Let $G = (V,E)$ be a graph, directed or undirected, and $\mathbf{P}$
be its transition matrix. A function $f \colon V \mapsto \mathbb{R}$
is {\em harmonic} at $v \in V$ if
\begin{equation}
  \label{eq:10}
  f(v) = \sum_{w \in V}{\mathbf{P}(v,w) f(w)}
\end{equation}
$f$ is harmonic on $V$ if it's harmonic for all $v \in V$. If $\mathbf{P}$
is irreducible, we have a simple characterization for harmonic
functions on $V$. Specifically,
\begin{lemma}
  \label{lem:1}
  Suppose that $\mathbf{P}$ is irreducible. A function $f \colon V \mapsto
  \mathbb{R}$ is harmonic on $V$ if and only if $f$ is constant on
  $V$. 
\end{lemma}
\begin{proof}
  It's easy to see that if $f$ is constant on $V$ then it's also
  harmonic on $V$. Thus, let's assume that $f$ is harmonic on $V$.
  Let $v_*$ be a node such that $f(v_*) \geq f(w)$ for all $w \in
  V$. Since $f$ is harmonic, from Eq.~\eqref{eq:10} we have that $f(w)
  = f(v_*)$ for all $w$ such that $\mathbf{P}(v_*,w) > 0$. We thus see
  that every vertex $w$ that's accessible from $v_*$ will satisfy
  $f(w) = f(v_*)$. Since $\mathbf{P}$ is irreducible, $f(v_*) = f(w)$ for
  all $w \in V$. $f$ is thus constant on $V$.
\end{proof}

\begin{proposition}
  \label{prop:6}
  Let $G = (V,E)$ be a graph and $\mathbf{P}$ be its transition
  matrix. Suppose that the Markov chain defined by $\mathbf{P}$ is
  regular. Then there exists a unique stationary distribution $\pi$ of
  $\mathbf{P}$. Furthermore, if $\mathbf{Q} = \mathbf{1} \mathbf{\pi}^{T}$ is the
  matrix with each row being the stationary distribution, then
  \begin{equation}
    \label{eq:22}
    \lim_{k \rightarrow \infty}(\mathbf{P} - \mathbf{Q})^{k} = 0 
  \end{equation}
  Eq.~\eqref{eq:22} is equivalent to $\rho(\mathbf{P}-\mathbf{Q}) < 1$
  where $\rho(\mathbf{P}-\mathbf{Q})$ is the spectral radius of
  $\mathbf{P} - \mathbf{Q}$.
\end{proposition}

\begin{proposition}
  \label{prop:7}
  Let $G = (V,E)$ be a graph and $\mathbf{P}$ be its transition
  matrix. Suppose that $\mathbf{P}$ is regular. Then the matrix $\mathbf{Z} =
  (\mathbf{I} - \mathbf{P} + \mathbf{Q})^{-1}$ exists and is given by
  \begin{equation}
    \label{eq:28}
    \mathbf{Z} = \sum_{k=0}^{\infty}(\mathbf{P} - \mathbf{Q})^{k} = \mathbf{I} +
    \sum_{k=1}^{\infty}(\mathbf{P}^{k} - \mathbf{Q})
  \end{equation}
  
\end{proposition}
\begin{proof}
  Since $\mathbf{P}$ is regular, by Proposition \ref{prop:6}, $\lim_{k
    \rightarrow \infty}(\mathbf{P} - \mathbf{Q})^{k} = 0$. Thus, $\mathbf{Z}$ has
  an expansion in term of a Neumann series
  \begin{equation}
    \label{eq:29}
    \mathbf{Z} = \sum_{k=0}^{\infty}(\mathbf{P} - \mathbf{Q})^{k}
  \end{equation}
  Since $\mathbf{P}\mathbf{Q} = \mathbf{P}1^{T}\mathbf{\pi} = 1^{T}\mathbf{\pi} = 
  1^{T}\mathbf{\pi}\mathbf{P} = \mathbf{Q}\mathbf{P} = \mathbf{Q}$,  
  one has $(\mathbf{P} - \mathbf{Q})^{k} = \mathbf{P}^{k} - \mathbf{Q}$ for $k \geq
  1$. Eq.~\eqref{eq:28} thus follows. 
\end{proof}
The matrix $\mathbf{Z}$ is termed the {\em fundamental matrix}
\citep{kemeny83:_finit_markov_chain}. Some properties of
$\mathbf{Z}$ are given in the following proposition.
\begin{proposition}
  \label{prop:8}
  Let $\mathbf{P}$ be the transition matrix of a regular Markov chain and
  $\mathbf{Z}$ be its fundamental matrix. We have
  \begin{enumerate}[(i)]
  \item $\mathbf{P}\mathbf{Z} = \mathbf{Z} - \mathbf{I} + \mathbf{Q}$. 
  \item $(\mathbf{I} - \mathbf{P})\mathbf{Z} = \mathbf{I} - \mathbf{Q}$.
  \item $\mathbf{Z} \mathbf{J} = \mathbf{J}$. 
  \end{enumerate}
\end{proposition}
\begin{proof}
  $\mathbf{P}\mathbf{Z} = \mathbf{P} - \sum_{k=1}^{\infty}(\mathbf{P}^{k+1} - \mathbf{Q})
  = \mathbf{Z} - \mathbf{I} + \mathbf{Q}$. (i) and (ii) thus follows. For (iii),
  note that $\mathbf{P}^{k}\mathbf{J} = \mathbf{Q}\mathbf{J} = \mathbf{J}$. 
\end{proof}
\section{Distance Geometry}
\label{sec:distance-geometry}
We discuss in this section some notations and results regarding
distance matrices. The notion of an Euclidean distance matrix (EDM) is
of particular importance to our discussion and is defined in
Definition \ref{def:1}. We then introduce two linear transformations
between matrices, the $\kappa$ transform and the $\tau$ transform.
Schoenberg's characterization \citep{schoenberg38:_metric} of Euclidean
distance matrices in terms of positive semidefinite matrices are
stated in Theorem \ref{thm:5}. We also state some simple results
regarding the $\kappa$ transforms that are useful in the context of
this work.

\begin{definition}
  \label{def:10}
  Let $\Delta = (\delta_{ij}) \in M_n(\mathbb{R})$. $\Delta$ is a Type 1
  Euclidean distance matrix (EDM-$1$) if and only if there exists a
  positive integer $p$ and $x_1, x_2, \dots, x_n \in \mathbb{R}^{p}$
  such that $\delta_{ij} = \| x_i - x_j \|$. $\Delta$ is a Type 2
  Euclidean distance matrix (EDM-$2$) if and only if there exists a
  $p$ and $x_1, x_2, \dots, x_n \in \mathbb{R}^{p}$ such that
  $\delta_{ij} = \|x_i - x_j\|^{2}$. The {\em embedding dimension} of
  $\Delta$ is the minimum $p$ such that a configuration of points
  $x_1, x_2, \dots, x_n$ exists with the desired property.
\end{definition}

\begin{definition}
  \label{def:11}
  Let $\mathbf{A} \in M_n(\mathbb{R})$. Define a linear mapping $\tau \colon M_n(\mathbb{R})
  \mapsto M_n(\mathbb{R})$ by
  \begin{equation}
    \label{eq:81}
    \tau(\mathbf{A}) = - \frac{1}{2} \Bigl(\mathbf{I} -
    \frac{\mathbf{J}}{n}\Bigr)\mathbf{A} \Bigl(\mathbf{I} - \frac{\mathbf{J}}{n}\Bigr)
  \end{equation}
  If $a_{ij}$ are the entries of $\mathbf{A}$ then
  \begin{equation}
    \label{eq:56}
    b_{ij} = -\frac{1}{2}\Bigl(a_{ij} - \frac{1}{n}\sum_{j=1}^{n}a_{ij} -
    \frac{1}{n}\sum_{i=1}^{n}{a_{ij}} +
      \frac{1}{n^2}\sum_{i=1}^{n}\sum_{j=1}^{n}a_{ij}\Bigr)
  \end{equation}
  are the entries of $\mathbf{B} = \tau(\mathbf{A})$. $\tau$
  is a continuous mapping from $M_n(\mathbb{R})$ to
  $M_n(\mathbb{R})$. 
\end{definition}
%
The following result provides a necessary and sufficient condition for
$\Delta$ to be an EDM-2 matrix.
\begin{theorem}[\citep{schoenberg38:_metric}]
  \label{thm:5}
  $\Delta$ is an EDM-2 with embedding dimension $p$ if and only
  if $\mathbf{B} = \tau(\Delta)$ is positive semidefinite with rank
  $p$. 
\end{theorem}

\begin{definition}
  \label{def:12}
  Let $\mathbf{A} \in M_n(\mathbb{R})$. Define a linear mapping
  $\kappa \colon M_n(\mathbb{R}) \mapsto M_n(\mathbb{R})$ by
  \begin{equation}
    \label{eq:61}
    \kappa(\mathbf{A}) = \mathbf{J}\mathbf{A}_{\mathrm{dg}} -
    \mathbf{A} - \mathbf{A}^{T} + \mathbf{A}_{\mathrm{dg}}\mathbf{J}
  \end{equation}
  where $\mathbf{A}_{\mathrm{dg}}$ is the diagonal matrix obtained by
  setting the off-diagonal entries of $\mathbf{A}$ to $0$. If $a_{ij}$
  are the entries of $\mathbf{A}$ then
  \begin{equation}
    \label{eq:70}
    b_{ij} = a_{ii} - a_{ij} - a_{ji} + a_{jj}
  \end{equation}
  are the entries of $\mathbf{B} = \kappa(\mathbf{A})$. $\kappa$ is
  also a continuous mapping from $M_n(\mathbb{R})$ to
  $M_n(\mathbb{R})$. 
\end{definition}

\begin{proposition}
  \label{prop:16}
  The $\kappa$ transform has the following properties.
  \begin{itemize}
  \item[(A)] Let $\mathcal{C} = \{ \mathbf{A} \in S_n(\mathbb{R}) \colon
    \mathbf{C}\bm{1}_{n}^{T} = \bm{0} \}$ be the set of symmetric
    matrices with zero row sums and let $\mathcal{D} = \{ \Delta \in
    S_n(\mathbb{R}) \colon \Delta_{\mathrm{dg}} = 0 \}$ be the set of
    symmetric hollow matrices. Then $\kappa$ and $\tau$ are inverse
    mappings between $\mathcal{C}$ and $\mathcal{D}$, i.e.,
  \begin{gather}
    \label{eq:55}
    \mathbf{A} \in \mathcal{C}
    \Longrightarrow \Delta = \kappa(\mathbf{A}) \in \mathcal{D}, \,\,
    \mathbf{A} = \tau(\Delta) \\
    \Delta \in \mathcal{D} \Longrightarrow \mathbf{A} = \tau(\Delta)
    \in \mathcal{C}, \,\, \Delta = \kappa(\mathbf{A})
  \end{gather}
  \item[(B)] $\kappa(\mathbf{J}) = 0$. More generally,
    $\kappa(\bm{a}\bm{1}^{T}) = \kappa(\bm{1}\bm{b}^{T}) = 0$
    for any vector $\bm{a}$, $\bm{b}$.
  \item[(C)] Let $\tilde{\mathbf{X}}$ be the double centering of
    $\mathbf{X}$, i.e.,
  \begin{equation}
    \label{eq:71}
    \tilde{\mathbf{X}} = \Bigl(\mathbf{I} - \frac{\mathbf{J}}{n}\Bigr)\mathbf{X} \Bigl(\mathbf{I} - \frac{\mathbf{J}}{n}\Bigr)
  \end{equation}
  Then $\kappa(\tilde{\mathbf{X}}) = \kappa(\mathbf{X})$.
  \end{itemize}
\end{proposition}
Part (A) of Proposition \ref{prop:16} is from
\citep{critchley88:_certain_linear_mappin}. Part (B) and (C)
follows directly from the definition of the $\kappa$ transform. 
\begin{proposition}
  \label{prop:18}
  Let $\mathbf{A} \in S_n(\mathbb{R})$ be a positive semidefinite
  matrix. Then $\Delta = \kappa(\mathbf{A})$ is EDM-2.
\end{proposition}
\begin{proof}
  The double centering $\tilde{\mathbf{A}}$ of $\mathbf{A}$ is a
  matrix in $\mathcal{C}$. By Proposition
  \ref{prop:16}, $\Delta = \kappa(\mathbf{A}) =
  \kappa(\tilde{\mathbf{A}})$ and that $\tilde{\mathbf{A}} =
  \tau(\Delta)$. Now $\mathbf{A} \succeq 0$ implies
  $\tilde{\mathbf{A}} \succeq 0$. By Schoenberg's criterion, $\Delta =
  \kappa(\tilde{\mathbf{A}})$ is EDM-2. 
\end{proof}
%
\section{Matrix Analysis}
We listed here some results in matrix analysis that are useful within
the scope of this dissertation.
%
Let $\mathbf{A} = (a_{ij})$ be an $n \times n$ matrix with real
entries $a_{ij}$. Denote by $R_i$ the sum $\sum_{j \not = i}{|a_{ij}|}$ of
off-diagonal elements in row $i$. 
%
\begin{theorem}[\citep{gersgorin31:_uber_abgren_eigen_matrix}]
  \label{thm:1}
  Let $\mathbf{A}$ be an $n \times n$ matrix with off-diagonal row
  sums $R_i$. Then the eigenvalue of $\mathbf{A}$ lies in the set
  \begin{equation}
    \label{eq:23}
    \bigcup \{z \in \mathbb{C} \colon |z - a_{ii}| \leq R_i \}
  \end{equation}
\end{theorem}
\begin{definition}
  \label{def:4}
  The matrix $\mathbf{A}$ is said to be diagonally dominant if
  $|a_{ii}| \geq R_i$ for all $i$ and strictly diagonally dominant if
  $|a_{ii}| > R_i$ for all $i$.
\end{definition}
If $\mathbf{A}$ is diagonally dominant, then by Ger\u{s}gorin's
circle theorem, the eigenvalues of $\mathbf{A}$ has nonnegative real
parts. If $\mathbf{A}$ is strictly diagonally dominant, then the
eigenvalues of $\mathbf{A}$ has positive real parts. 
%
\begin{definition}
  \label{def:8}
  Let $Z_n \subset M_{n}(\mathbb{R})$ be the set of matrices with
  non-positive off-diagonal entries, i.e.,
  \begin{equation}
    \label{eq:24}
    Z_n = \{ \mathbf{A} = (a_{ij}) \in M_{n}(\mathbb{R}) \colon a_{ij}
    \leq 0 \,\, \text{if $i \not = j$} \}
  \end{equation}
 A matrix $\mathbf{A} \in Z_n$ is called an $M$-matrix if $A$ is
 positive stable, i.e., if the eigenvalues of $\mathbf{A}$ has
 positive real parts.
\end{definition}
A relationship between $M$-matrices and non-negative matrices is given
by the following result \citep[see][\S 2.5]{horn94:_topic_in_matrix_analy}.
\begin{theorem}
  \label{thm:2}
  $\mathbf{A} \in Z_n$ is a $M$-matrix if and only if $\mathbf{A}$ is
  non-singular and $\mathbf{A}^{-1} \geq 0$.  
\end{theorem}
%%% Local Variables: 
%%% mode: latex
%%% TeX-master: "dissertation"
%%% End: 

\bibliographystyle{plainnat}
\bibliography{dissertation.bib}
\newpage
\thispagestyle{empty}
\pagestyle{empty}
%\geometry{letterpaper,tmargin=1in,bmargin=1in,lmargin=1in,rmargin=1in,headheight=0in,headsep=0in,footskip=.3in}

\setlength{\parindent}{0in}
\setlength{\parskip}{0in}
\setlength{\itemsep}{0in}
\setlength{\topsep}{0in}
\setlength{\tabcolsep}{0in}

% Name and contact information



%%%%%%%%%%%%%%%%%%%%%%%%%%%%%%%%%%%%%%%%%%%%%%%%%%%%%%%%%
% Now for the actual document:

% Name with horizontal rule
\vspace{-1pt} {\begin{center} \Huge Curriculum Vitae \end{center}}

\bigname{\name}

\vspace{-8pt} \rule{\textwidth}{1pt}


\vspace{8 pt}

%%%%%%%%%%%%%%%%%%%%%%%%
\begin{ressection}{Education}

	\begin{ressubsec}{Indiana University, Bloomington}{Bloomington, IN}{Ph.D in
        Computer Science}
		\ressubitem{Graduation Date: October, 2010}
	\end{ressubsec}
	\begin{ressubsec}{University of Wisconsin, Milwaukee}{Milwauke,
        WI}{M.Sc in Computer Science}
		\ressubitem{Graduation Date: May 2004}
	\end{ressubsec}
	\begin{ressubsec}{Assumption Uiversity, Thailand}{Bangkok,
        Thailand}{B.Sc in Computer Science, Summa Cum Laude}
		\ressubitem{Graduation Date: October 2001}
	\end{ressubsec}
\end{ressection}

\begin{ressection}{Research Interests}

	\resitem{Combinatorics and graph theory. Current interests include
    analytical methods for combinatorial enumeration of discrete
    structures.}

	\resitem{Statistical machine learning. Current interests include
      clustering, dimension reduction, and learning theory.}

\end{ressection}
%%%%%%%%%%%%%%%%%%%%%%%%
\begin{ressection}{References}

	\begin{ressubsec}{Prof. Michael Trosset}{Indiana University,
        Bloomington}{Department of Statistics}
		\ressubitem{Email: mtrosset@indiana.edu}
	\end{ressubsec}
	\begin{ressubsec}{Prof. Dirk Van Gucht}{Indiana University,
        Bloomington}{School of Informatics and Computing} 
		\ressubitem{Email: vgucht@indiana.edu}
	\end{ressubsec}
	\begin{ressubsec}{Prof. Amol Mali}{University of Wisconsin,
        Milwaukee}{Department of Electrical Engineering and Computer
          Science}
		\ressubitem{Email: mali@cs.uwm.edu}
	\end{ressubsec}
	\begin{ressubsec}{Prof. Thitipong Tanprasert}{Assumption University, Thailand}{Faculty of Science and Technology}
		\ressubitem{Email: nui@scitech.au.edu}
	\end{ressubsec}
\end{ressection}


%%%%%%%%%%%%%%%%%%%%%%%%
\vfill\newpage
\end{document}
%%% Local Variables: 
%%% mode: latex
%%% TeX-master: t
%%% End: 
